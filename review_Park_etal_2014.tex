\documentclass[11pt, oneside]{article}   	% use "amsart" instead of "article" for AMSLaTeX format
\usepackage{geometry}                		% See geometry.pdf to learn the layout options. There are lots.
\geometry{letterpaper}                   		% ... or a4paper or a5paper or ... 
%\geometry{landscape}                		% Activate for for rotated page geometry
%\usepackage[parfill]{parskip}    		% Activate to begin paragraphs with an empty line rather than an indent
\usepackage{graphicx}				% Use pdf, png, jpg, or eps� with pdflatex; use eps in DVI mode
								% TeX will automatically convert eps --> pdf in pdflatex		
\usepackage{amssymb}

\title{Review: Extratropical SST key to understand the discrepancy in future Sahel rainfall projections}
\author{Jong-yeon Park, J{\"u}rgen Bader, Daniela Matei; Reviewed by Levi Silvers}
%\date{}							% Activate to display a given date or no date

\begin{document}
\maketitle

\section{General Comments}

\subsection{Overview}

This is a nice study that highlights, and seeks to resolve, a major discrepancy between various projections of rainfall in the Sahel region.  Trying to better understand what mechanisms can lead to precipitation changes as a result of changing greenhouse gas forcing is an inherently interesting subject.  From a technical point of view it is clearly important to understand why different models produce different results.  From an economic point of view the possibility of changing locations for deserts and regions of significant rainfall is also quite important.  

The sensitivity experiments that use ECHAM are very nice and add strength to your analysis.  I liked the fact that ECHAM seemed to have a relatively neutral prediction of future Sahel rainfall (S1).  This contrasts well with the GFDL and MIROC models.  It is helpful to show the reader that your hypothesis can be verified with the use of a separate model, and that simply by changing the forcing you can arrive at similar results to the GFDL and MIROC models.  

This will be nice paper and make a useful contribution towards a better understanding of both what the atmosphere does over Africa, and why the model ensemble mean is such a poor indicator of the individual models.  

\subsection{Questions}

Figure 2 seems to demonstrate your point very well and the contrast between 2.b and 2.d is striking.  It seems that you could almost tell the whole story with this figure.  I have a few questions regarding this figure, and the analysis of the indices in general.  After returning to Figure 2 another time I noticed that the correlation between 2.b and 2.d is similar, could one conclude from this simply that the Sahel rainfall is anti correlated with the tropical index?  Maybe this points to a clear connection between the tropics and extratropics.  Am I correct in thinking that the coefficients of the multi-linear model are computed from the historical simulations?  What does the same analysis show when the coefficients are computed from the RCP 4.5 simulations or from different time periods in the historical period?  For a topic such as future Sahel rainfall, the ensemble mean prediction of the models does not seem too useful, but a mechanistic understanding of what the possible future scenarios could lead to would be interesting and useful.  I think this is what you are trying to do.  

I don't think that I agree with the sentence that begins on line 69 with `However, they fail...'.   The fact that the indices differ from the MIROC-ESM does not make it clear to me that the index has failed, perhaps the model is misleading us here?  This raises the general question of which model simulations the authors tend to trust/believe and why.  Figure S1 shows the full range of possible projections.  Is there a physically motivated reason to trust, or to try and develop indices that fit with, the MIROC models?  Perhaps this reveals my lack of knowledge about the observational record, but what do the observations over the past decades in the Sahel tell us?  Is it clear that the rainfall in that region is increasing?        

Figure S1 is quite striking, and nicely shows the problem you are working on resolving.  It is noteworthy that all of the MIROC models are grouped together at the extreme end of the figure.  It looks like the Ensemble mean would be almost zero without them.  Figure S2 shows that several of the models which produce wet conditions  also have amplified NH extratropical SST warming.  This is  a good point, and used well in the paper but I am still curious what is really going on in the MIROC models that makes their response so different to the others.  It seems that this study is essentially a comparison of the predicted Sahel rainfall between the GFDL model and the MIROC model.  Do we have a good reason to think that one is a better representation of the Earth system than the other?  

I am not sure what the correct, or the most standard method in publications for expressing tenses is.  As a result, sometimes it seems that a statement should be presented in the past tense, and sometimes in the present tense.  An example would be line 14, should `found' be in the past tense as you wrote, or in the present tense?  My inclination is that it should be in the present tense, but I am not grammatically certain.  

Why were the names 20C and 21C chosen for the sensitivity experiments?  I am not saying the names need to be changed, but simply that the idea behind the names could be stated to help the reader.  These names could be confusing to the reader if they refer to experiments in which the SST are not constant.  If they indicate the mean values of SST than that should be stated.  Perhaps I missed it.  

Do you need to mention both CMIP3 and CMIP5 models?  It could simplify both your abstract and some of the text to simply state that the model discrepancies are present for both phases of the comparison rather than stating it first for CMIP3 and then also discussing the persistence of the model spread for CMIP5.  

The sentence which begins on line116 is a bit confusing to me.  Why does this choice of model emphasize what you claim?  In particular, you state that anomalies may be a more dominant driving factor than SST patterns, but this is not obvious to me.  Has this been shown to be the case in other literature?  In general I would guess that patterns would be more important than anomalies, but I do not now.  Are the patterns the same between the various models that you are comparing?  I realize this could open up an entire additional line of research and that it can't all be discussed in this paragraph, but I do think the sentence you have could be further clarified.  Maybe a future paper could examine the dynamical influences of SST patterns in contrast to anomalies?    

ln 63 of Supp. Material: I think ECHAM6 is being used to analyze predictions of Sahel rainfall, not the actual Sahel rainfall.  

As far as I could see in the paper, all of your analysis covered the months of July, August, and September.  What happens with rainfall in the Sahel during the rest of the year?  Is there as large of a discrepancy among the models for the rest of the year?  

Concerning the paragraph in the supplementary material that begins on line 111.  This argument seems reasonable to me, but at first I was confused by the use of the terms `reduced low-level pressure' and `reduction of low-level pressure'.  Initially I thought the magnitude of the low-level system was decreasing, which I believe is opposite your argument.  If you want to leave this the same it is probably ok.  One option for clarity would be to discuss strengthening and weakening of the system which seems to convey a sense of the change in magnitude well.  

\section{Spelling/grammatical corrections}

       {\bf  Primary Article} 

	\begin{enumerate}
	
	  \item ln 11, 17:  twenty should be twentieth
	  \item ln 15: I think that changing found to either find or show would be a more natural expression.  See comment above about tenses.  
	  \item ln 63: combination should be plural
	  \item ln 67: `Such statistical model and indices' should be either `Such statistical models and indices' or `Such a statistical model and indices'.  You could also simply say, `Such indices       are similar...' , but that changes the meaning a little bit.  
	  \item ln 70: The end of the sentence which ends on this line should read either `an opposite sign of the trend' or `opposite signs of the trend'.  
	  \item ln 95: I think that  `for future Sahel,'  should  be changed to `for future Sahel rainfall,'.  This could also be a natural breaking point if you want to break the long sentence into two, the second one starting with `The model discrepancy...'  but this is definitely a matter of personal preference.  
	  \item ln 111: `dry' and `wet' here are grammatically incorrect.  There are several possible options such as drying, moistening, becoming more dry etc.  
	  \item ln 117: There should be a `the' before CMIP5.
	  \item ln 118: emphasize should be emphasizes.
	  \item ln 119: This is probably correct as is, but in my opinion the sentence would be cleaner if both `rather' and `different' are dropped. 
	  \item ln 159: There should be a `the' before West.
	  \item ln 161: concerns should be concern.  
	  \item ln 162: This sentence is not so clear.  In particular, what is `its' referring to?  The way this reads, `its' could be the calamities or the rainfall projections.  I could guess that you mean rainfall projections, but I think it can be clearer.      
	  \item ln 166: This should either read `to get a more reliable estimate of' or `to get more reliable estimates'.  
	  \item ln 167: relative should be relatively
	  \item ln 177: `has' should be either `has been' or `was'.
	  \item ln 184: is should be are.
	  \item ln 187: `fitting' should be `fit', there should be a `the' before least-squares.  
	  \item ln 190: There should be a `The' at the beginning of this sentence. 
	  \item ln 192: `has' should be `have'.
	  \item ln 194: `is' should be `are'. 
	  \item ln 197: Do you mean `forced' instead of `prescribed' both here and in line 199?  If you do mean `prescribed' then I think the sentence needs to be reworked a little.  
	  \item ln 198: There should be a `the' before twentieth.  
	  \item ln 200: `is' should be `are'.  
	  
	\end{enumerate}
	
     {\bf Supplementary Information}

	\begin{enumerate}
	  \item ln 12: Should start with either `Although the multi-model ensemble mean shows a...' or `Although multi-model ensemble means show a...' 
	  \item ln 20: There should be a `the' before major challenges and `response of global' should be `response to global'.  
	  \item ln 23: I am pretty sure that Northern Hemisphere should be capitalized.  
	  \item ln 24:  I am a bit confused by this sentence.  Perhaps `stronger' should be replaced by `distinct'?  We can talk about this if you want.  
	  \item ln 61: `show'  should be `shows'.
	  \item ln 83: The symbol used between your two concentrations of $CO_2$ is not clear. 
	  \item ln 99-103: This sentence should be restructured, I think it is not so clear.  
	  \item ln 112: `the' before reduction should be an `a'. 
	  \item ln 116: `wind' should be `winds'.
	  \item ln 129: There should be a `the' before both Tropical and African.  
	  \item ln 142: The `a' before clear evidence should be omitted.    
	\end{enumerate}
	


\section{Figures}

\begin{enumerate}
  \item Figure 1, in the caption, `twenty-century' should be `twentieth-century'.  
  \item Figure 3, in the caption, `represent' should be changed to `represents an'. 
  \item Figure 4, ln 320 `centuries' should be `century'.  
  \item Figure 4, It looks like the color bars are identical.  I would then remove one (top), and then make 4.a and 4.b the same size as 4c-d.  Feel free to ignore this comment if you prefer otherwise.  
  \item Figure 5 ??  Is this figure 5?  There is no label and no caption.  You need a caption.  What region is being referred to in the title?  The line in the plot should be clearly marked.  You seem to have the same issue with the last figure of the supplementary material.  Is this just a software issue that didn't format you material correctly?  I also don't have a label or caption for the last figure of the supplementary material.  Neither of these figures seem to be discussed in the text and it is not clear what they add to the paper.  Perhaps they are not needed at all?  
  \item Table S1   In my opinion, `experiments prescribed by SSTs' would be clearer if it was ` experiments using prescribed SSTs'.  
  \item Figure S3 ln 233 rainfalls should be rainfall
  \item Figure S5 ln 256 `twentieth' should be followed by `century'.
  \item Figure S10  I think that `Closed' should be `filled'.  All of the circles are closed.  
\end{enumerate}

%\subsection{}



\end{document}  