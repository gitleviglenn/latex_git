\documentclass[11pt]{amsart}   	% use "amsart" instead of "article" for AMSLaTeX format
\usepackage{geometry}                		% See geometry.pdf to learn the layout options. There are lots.
\geometry{letterpaper}                   		% ... or a4paper or a5paper or ... 
%\geometry{landscape}                		% Activate for for rotated page geometry
%\usepackage[parfill]{parskip}    		% Activate to begin paragraphs with an empty line rather than an indent
\usepackage{graphicx}				% Use pdf, png, jpg, or eps� with pdflatex; use eps in DVI mode
								% TeX will automatically convert eps --> pdf in pdflatex		
\usepackage{amssymb}
\usepackage{gensymb}

\title{Using the Walker Circulation to connect low-level cloud changes to deep convective entrainment}
\author{Levi G. Silvers}
%\date{2016}							% Activate to display a given date or no date

\begin{document}
\maketitle
\section{Key Points}
\begin{itemize}
  \item{How do clouds influence the Walker Circulation?}
  \item{To what extent are the deep convective clouds and the low-level clouds in the Walker Cell coupled?}
  \item{When simulating the Walker Circulation, how well does a GCM compare to a CRM?}  
\end{itemize}

\section{Streamfunction}

For a two-dimensional flow in the $\lambda,p$ plane we can write the mass conservation equation in terms of a 
mass streamfunction $\psi$ as: 
\begin{equation}
u =  -\frac{\partial \psi}{\partial p}   \,\, {\rm and} \,\, \omega =\frac{\partial \psi}{a \rm{cos} \phi \partial \lambda}
\end{equation}
with $a$ as the radius of Earth.
This satisfies the continuity eqation: 
\begin{equation}
\frac{\partial}{a \rm{cos} \phi\partial\lambda}\left(-\frac{\partial \psi}{\partial p}\right)+\frac{\partial}{\partial p}\left(\frac{\partial \psi}{a \rm{cos}\phi \partial \lambda}\right)=0.
\end{equation}
The streamfunction is therefore defined in two ways and can be solved with either equation.  The choice is often made based on the 
boundary conditions.  Dimensionally, $\psi$ must have units of velocity multiplied by pressure, or $kg/s^3$.  Often, the streamfunction
is computed as a mass streamfunction with units of $kg/s$.  Density does not appear above
because pressure is being used as the vertical coordinate.   With height as the vertical coordinate, mass is given by $\rho dxdydz$, 
assuming a hydrostatic atmosphere and rewriting in terms of pressure gives mass given by $-(1/g) dxdydp$.
Scaling the equations above by $a/g$ will result in $\psi$ having units of $kg/s$.  

Solving for $\psi$ by integrating the vertical velocity is possible, but requires a boundary condition along one edge of the domain between the surface and the 
top of the atmosphere.  In contrast, solving for $\psi$ by integrating the horizontal velocity allows us to set $\psi=0$ along the upper edge of the domain.  
This is both simple, and physically motivated.  

Below solves for $\psi$ by integrating the vertical velocity $w$ over $x$ yields
\begin{equation}
\int_0^{x'} \rho(x,p)w(x,p)d{x'}=\int_0^{x'}\frac{\partial \psi(x,p)}{\partial x'}d{x'}
\end{equation}
\begin{equation}
\int_0^{x'} \rho(x,p)w(x,p)d{x'}= \psi(x,p)-\psi(0,p)
\end{equation}
\begin{equation}
\psi(x,p)=\int_0^{x'} \rho(x,p)w(x,p)d{x'} + \psi(0,p)
\end{equation}

We now need an expression for $\psi(0,p)$, this can be found by integrating the zonal velocity $u$ in $p$.
\begin{equation}
\int_{p_{sfc}}^{p'} u(x,p') d{p'} = g\int_{p_{sfc}}^{p'}\frac{\partial\psi}{\partial p'}d{p'}
\end{equation}
\begin{equation}
\int_{p_{sfc}}^{p'} u(x,p') d{p'} = g\psi(x,p)-g\psi(x,p_{sfc}).
\end{equation}
So at $x=0$ we have: 
\begin{equation}
\psi(0,p)=\frac{1}{g}\int_{p_{sfc}}^{p'} u(0,p') d{p'}+\psi(0,p_{sfc}),
\end{equation}
which leads to 
\begin{equation}
\psi(x,p)=\int_0^{x'} \rho(x,p)w(x,p)d{x'} + \frac{1}{g}\int_{p_{sfc}}^{p'} u(0,p') d{p'}+\psi(0,p_{sfc}).
\end{equation}
The term $\psi(0,p_{sfc})$ is simply a constant boundary condition at the atmospheric level closest to the surface and at the first horizontal grid point.   By initializing the $\psi$ field as 0 everywhere I have chosen this to be zero.  Note that while the $u(0,p')$ term is essentially a boundary condition for the computation of $\psi(x,p)$ by integrating $w(x,p)$, it is still dependent on pressure and therefore will change at each level of the atmosphere.      Writing this out in terms of grid points using $i,j$ for $x,p$ we have:
\begin{equation}
\psi_{i,j}= \rho_{i,j} w_{i,j}\Delta x+\sum_{j=p_{sfc}}^j \left(\frac{1}{g}u_{i,j}(p_j-p_{j-1})\right).
\end{equation}
In the previous equation, $w, u,$ and $\rho$ have all been averaged in space and time and $p$ is the pressure on full model levels.  The density, $\rho$ is computed as $ \rho_{i,j}=\frac{p_j}{R T_{v;i,j}}  $  where $R$ and $T_v$ are the gas constant for dry air ($287 \rm{J/kg K}$) and the virtual temperature ($T_v=T(1+q/\epsilon)/(1+q)$), respectively.  The specific humidity is given by $q$ and $\epsilon = 0.622$.

\section{Figures}

% looks like there is an Overfull \hbox here and the figure is not printing....
\begin{figure}[htb]
  \includegraphics{randomfigs/condensate_3mods.eps}
\end{figure}

 \begin{figure}[htb] %  figure placement: here, top, bottom, or page
   \centering
        %\includegraphics[width=5.0in]{agu/silvers_etal_2016/rce_20km_tqv_20mmto80mm_cathy.eps}
        \includegraphics[width=5.0in]{/Users/silvers/code/matlabscripts/WalkerCirc/figs_ctl/psi_log_1km.pdf}
   \caption{Column integrated water vapor ($\rm{mm}$).  The SST$=301$ K for all domains. 
   %and the time of the snapshots is the same as for Fig. \ref{fig:precvsdom}, but the color scale differs.
   For comparison, the black bar in the 3M, 12M, 50M, and 200M domains represents a length of 830km, corresponding to the short edge length of the 3/4M domain.}
   \label{fig:tqv5dom}
\end{figure}




\end{document}  