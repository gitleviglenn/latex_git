%%%%%%%%%%%%%%%%%%%%%%%%%%%%%%%%%%%%%%%%%%%%%%%%%%%%%%%%%%%%%%%%%%%%%
% this is the version Modified on 1-27-2012 and then sent to Wayne 
%%%%%%%%%%%%%%%%%%%%%%%%%%%%%%%%%%%%%%%%%%%%%%%%%%%%%%%%%%%%%%%%%%%%%
%
% The following two commands will generate a PDF that follows all the requirements for submission
% and peer review.  Uncomment these commands to generate this output (and comment out the two lines below.)
%
% DOUBLE SPACE VERSION FOR SUBMISSION TO THE AMS
\documentclass[12pt]{article}
\usepackage{ametsoc}
\usepackage{lineno}
\linenumbers
%
% The following two commands will generate a single space, double column paper that closely
% matches an AMS journal page.  Uncomment these commands to generate this output (and comment
% out the two lines above. FOR AUTHOR USE ONLY. PAPERS SUBMITTED IN THIS FORMAT WILL BE RETURNED
% TO THE AUTHOR for submission with the correct formatting.
%
% TWO COLUMN JOURNAL PAGE LAYOUT FOR AUTHOR USE ONLY
%%%%\documentclass[10pt]{article}
%%%%\usepackage{ametsoc2col}
%
%%%%%%%%%%%%%%%%%%%%%%%%%%%%%%%%%%%%%%%%%%%%%%%%%%%%%%%%%%%%%%%%%%%%%
% ABSTRACT
%
% Enter your Abstract here
%%%%%%%%%%%%%%%%%%%%%%%%%%%%%%%%%%%%%%%%%%%%%%%%%%%%%%%%%%%%%%%%%%%%%
\newcommand{\myabstract}{The subject of this study is topographically bound low level jets, such 
as the South American summertime low level jet on the east side of the Andes 
and its companion, the Chilean low level jet on the west side of the Andes. 
These jets are interpreted as balanced flows that obey the potential vorticity 
invertibility principle. This invertibility principle is expressed in 
isentropic coordinates, and the mathematical issue of isentropes that intersect 
the topography is treated by the method of a massless layer. In this way, the 
low level jets on the west and east sides of the Andes can both be attributed to 
the infinite potential vorticity that lies in the infinitesimally thin massless 
layer on the topographic feature. To obtain a cyclonic flow centered on the 
topographic feature, the mountain crest must have been heated enough to draw 
down the overlying isentropic surfaces; otherwise, isentropic surfaces bend
upward at the mountain crest and an anticyclonic flow is produced. Both 
anticyclonic and cyclonic solutions are obtained here using analytical and 
numerical methods to solve the invertibility principle. The summertime, 
topographically bound flows discussed here are quite distinct from the 
wintertime Rossby wavetrain patterns that occur when strong westerlies impinge 
on the topography. 
}
%
\begin{document}
%
%%%%%%%%%%%%%%%%%%%%%%%%%%%%%%%%%%%%%%%%%%%%%%%%%%%%%%%%%%%%%%%%%%%%%
% TITLE
%
% Enter your TITLE here
%%%%%%%%%%%%%%%%%%%%%%%%%%%%%%%%%%%%%%%%%%%%%%%%%%%%%%%%%%%%%%%%%%%%%
\title{\textbf{\large{A Theory of Topographically Bound Balanced Motions 
                      and Application to Atmospheric Low-Level Jets}}}
%
% Author names, with corresponding author information. 
% [Update and move the \thanks{...} block as appropriate.]
%
\author{\textsc{Levi G. Silvers}
				\thanks{\textit{Corresponding author address:} 
				Levi Silvers, Colorado State University, 
				Fort Collins, CO 80523. 
				\newline{E-mail: levi@atmos.colostate.edu}}\quad\textsc{and Wayne H. Schubert}\\
\textit{\footnotesize{Colorado State University, Fort Collins, Colorado}}
\and 
%\centerline{\textsc{Extra Author}}\\% Add additional authors, different insitution
%\centerline{\textit{\footnotesize{Affiliation, City, State/Province, Country}}}
}
%
% Formatting done here...Authors should skip over this.  See above for abstract.
\ifthenelse{\boolean{dc}}
{
\twocolumn[
\begin{@twocolumnfalse}
\amstitle

% Start Abstract (Enter your Abstract above.  Do not enter any text here)
\begin{center}
\begin{minipage}{13.0cm}
\begin{abstract}
	\myabstract
	\newline
	\begin{center}
		\rule{38mm}{0.2mm}
	\end{center}
\end{abstract}
\end{minipage}
\end{center}
\end{@twocolumnfalse}
]
}
{
\amstitle
\begin{abstract}
\myabstract
\end{abstract}
\newpage
}
%%%%%%%%%%%%%%%%%%%%%%%%%%%%%%%%%%%%%%%%%%%%%%%%%%%%%%%%%%%%%%%%%%%%%
% MAIN BODY OF PAPER
%%%%%%%%%%%%%%%%%%%%%%%%%%%%%%%%%%%%%%%%%%%%%%%%%%%%%%%%%%%%%%%%%%%%%
\section{Introduction}

     South American mean January 2003 winds at 925 mb are shown in Fig.~1. 
A striking feature is the strong cyclonic flow centered on the Andes.  
The South American low-level jet (SALLJ) is evident in the northerly flow 
east of the Andes, while the Chilean coastal low-level jet is evident in the
southerly flow over the eastern Pacific Ocean.  These Southern Hemisphere 
low level jets are analogous to the summertime 
low level jets around the mountainous regions of North America, i.e., the 
Great Plains low level jet (GPLLJ) and the California coastal low level jet.   

     Figures 2 and 3 show east-west cross sections of the mean meridional 
wind fields and the mean isentropes observed during the Year of Tropical 
Convection (YOTC).  These are two-month mean warm season wind fields averaged using the 
times 00, 6, 12, and 18 UTC.  The 
resolution of these ECMWF analysis fields is $0.5^{\circ}\times 0.5^{\circ}$ 
with 11 irregularly spaced vertical levels.  The YOTC analysis is available for the two 
year period between May 2008 and April 2010.  Although originally proposed to be 
a one year research program YOTC was extended for an additional year in order to 
capture a cycle of both La Nina and El Nino.  Wind fields around 
the Andes are shown at 21S and 30S.  The SALLJ often maximizes near 20S, 
while, as seen in Fig.~2, the coastal LLJ is stronger farther south. Cross 
sections of the wind fields around the Rocky Mountains are shown at 30N and 35N.  
The GPLLJ maximizes at about 25N, but as seen in Fig.~3, it is still quite
strong at 30N and 35N.  Due to 
the relatively cold eastern Pacific, the coastal LLJs tend to have a wind 
maximum that is closer to the surface than the jets to the east of 
the mountain ranges. The coastal jets are also broader than the 
plains jets because of the influence of the Pacific anticyclonic 
circulations to the west. Isentropes are generally drawn down over the 
mountain ranges, thereby intersecting the Earth's surface along
the sides and crests of the mountains.   
It should be noted that, on individual days, these low level jets can be 
considerably stronger than shown on two-month mean cross sections. 
For example, Vera et al.~(2006, their Fig.~8b) show a specific SALLJ 
event in February 2003 with a maximum wind of 25 m s$^{-1}$ at a height 
of 800-700 hPa.  
The cyclonic motion centered on the mountain 
ranges is the most obvious feature of the wind fields in these figures.  The 
North American Regional Reanalysis also clearly shows a dominant 
cyclonic circulation (see Jiang et al. 2007, their Fig.~1) over the Rocky Mountains.  
   
One of the first climatologies of LLJs was given by Bonner (1968), along with an often 
used set of wind shear criteria to define LLJs.   Stensrud (1996) provided a
review of LLJs and in particular, their significant influence on the global climate.  
An excellent review 
of the literature relevant to LLJs and analysis of the dynamical mechanisms that have been 
historically proposed to explain the existence and variability of LLJs was given by Jiang et al. 2007.
Much of the previous research on LLJs can be loosely grouped into two categories.  
The first includes boundary layer processes and forcing on relatively short (diurnal-a few 
days) times scales.  Relevant studies include Blackadar (1957), Holton (1967), Jiang 
et al. (2007), and Rife et al. (2010).  These studies were particularly interested in 
determining the physical mechanisms behind the diurnal oscillation and the phase that 
is observed in many LLJs.  The second category deals mostly with the synoptic-scale, 
monthly or seasonal mean structure of LLJs and includes studies such as Wexler (1961), 
Byerle and Paegle (2002, 2003), and Ting and Wang (2006).  These two groups certainly 
have some overlap.  The research presented here examines the influence of 
topographically bound balanced motions on the synoptic-scale, warm season mean structure of LLJs.

    The theoretical study most relevant to the present work is that of 
Eliassen (1980), who investigated the steady, topographically bound, balanced 
response of a rotating, stratified fluid to orography. Although Eliassen did 
not specifically apply his theory to LLJs, his results provide 
physical insight into the dynamical characteristics of these phenomena.  
Eliassen defined isentropic obstacles as those for which  
the lowermost isentrope continuously follows the geopotential surface.  
A non-isentropic obstacle is one whose geopotential surface punctures 
the lowermost isentropes. Eliassen showed that solutions of the potential 
vorticity invertibility principle for an isentropic ridge were only possible 
if the height of that ridge did not exceed a critical value. This `subcritical 
ridge' case is illustrated by the left panel of Fig.~4, which shows that   
the isentropes are compressed as they bend up over the 
mountain crest, thereby producing a balanced anticyclonic flow above the crest. 
If the `critical crest height' is exceeded, the ridge must 
puncture the lowermost isentropes. Two other cases not investigated by 
Eliassen are shown in the middle and right panels of Fig.~4.           
The middle panel shows a surface of constant geopotential that has 
been locally heated. In this case the isentropes bend downward and a 
balanced, cyclonic flow is produced.  
The surface potential temperature anomaly shown in the center panel of Fig. 4 represents 
a similar balanced flow structure as those studied by Thorpe (1986).  His work examined 
synoptic disturbances in gradient wind balance, this study looks at synoptic disturbances in 
geostrophic balance.  A combination of the two cases 
shown in the left and middle panels results in the heated obstacle shown 
in the right panel. Obviously, the right panel is the case most relevant 
to the observations shown in Figs.~2 and 3.
  
    The purpose of the present paper is to extend the analysis of Eliassen to 
include all three cases shown in Fig.~4, and to thereby demonstrate that many LLJs 
can be interpreted as balanced flows attributed, through the potential 
vorticity invertibility principle, to the infinite potential vorticity 
that resides in the massless layer over locally heated topography 
such as illustrated in the right panel of Fig.~4. 
A complete potential vorticity based analysis of LLJs would have to include both the 
potential vorticity at the surface that results from a gradient of potential temperature 
and the potential vorticity of the interior flow.  The surface component is related to 
radiative heating and the atmospheric stratification while the interior component 
is due to convective activity, absolute vorticity, or the stratification.  The interaction 
between the interior quasigeostrophic potential vorticity and the surface gradient of 
potential temperature was reviewed by Hoskins et al. (1985).  A nongeostrophic 
generalization was applied to a wake circulation on the lee side of a mountain by 
Schneider et al. (2003).  We here focus on the surface potential vorticity.  We do not 
assume that the affect of the surface potential vorticity is greater than that of the 
interior potential vorticity, we are simply demonstrating that LLJs are impacted 
by this surface potential vorticity.
The outline of the paper 
is as follows. Section 2 presents the $f$-plane invertibility principle in 
isentropic coordinates.  Section 3 derives analytical solutions for the  
subcritical isentropic ridge case. An analytical formula for the critical 
crest height is then derived.  The analytical methods used in section 3 
yield closed form solutions only in cases where there is no massless 
layer (i.e., the lower boundary is an isentropic surface). In order to compute 
the wind and mass fields that result from locally heated lower boundaries,  
section 4 solves the invertibility principle using a finite-difference 
approximation.  An analysis is then made of the results and how they 
compare to observations of LLJs.  Concluding remarks, including a discussion 
of a generalization from $f$-plane theory to the sphere, are given in section 5.  

\section{Invertibility principle}          %%%%%%%%%   Section 2   %%%%%%%%%%%%%%

     Consider hydrostatic, geostrophic, $y$-independent motions of a compressible 
stratified fluid on an $f$-plane. Using the potential temperature $\theta$ 
as the vertical coordinate, the potential vorticity is given by 
\begin{linenomath*}
\begin{equation}                        % Equation (1)
        P = \left(f + \frac{\partial v}{\partial x}\right)
	    \left(-\frac{1}{g}\frac{\partial p}{\partial\theta}\right)^{-1},  
\end{equation} 
\end{linenomath*}
with $f$ denoting the constant Coriolis parameter, $g$ the acceleration of 
gravity, $v(x,\theta)$ the meridional component of the geostrophic flow, 
and $p(x,\theta)$ the pressure. Expressing the density by 
$\rho=p/(RT)=c_p p/(R\theta\Pi)$, it is easily shown that 
$\theta\rho(d\Pi/dp)=1$, where $\Pi=c_p(p/p_0)^{R/c_p}$, 
with $p_0$ denoting the constant reference pressure, 
$R$ the gas constant, and $c_p$ the specific heat at constant pressure. 
This allows (1) to be written in the form 
\begin{linenomath*}
\begin{equation}                       % Equation (2)
    \frac{g}{\theta\rho P}\frac{\partial(v+fx)}{\partial x}
                        + \frac{\partial\Pi}{\partial\theta} = 0.   
\end{equation} 
\end{linenomath*}
Using the geostrophic relation $fv=(\partial M/\partial x)$ and 
the hydrostatic relation $\Pi=(\partial M/\partial\theta)$, where 
$M=\theta\Pi+\phi$ is the Montgomery potential and $\phi$ is the 
geopotential, we can write the thermal wind relation in the form 
\begin{linenomath*}
\begin{equation}                       % Equation (3)
       f\frac{\partial v}{\partial\theta} 
      - \frac{\partial\Pi}{\partial x} = 0.   
\end{equation} 
\end{linenomath*}

With proper boundary conditions and with $P(x,\theta)$ specified, 
(2) and (3) constitute an invertibility problem for the unknown 
functions $v(x,\theta)$ and $\Pi(x,\theta)$.  
Because of the dependence of $\rho$ on $\Pi$, (2) is nonlinear. However, 
the nonlinearity is weak, and for the analytical solutions 
presented in section 3, it will be removed by replacing $\rho(x,\theta)$ 
with the specified far-field profile $\tilde{\rho}(\theta)$. For the 
numerical solutions presented in section 4, this nonlinearity will 
be retained.  

\section{Analytical solution for the case of an isentropic mountain}   
 %%%%  Section 3  %%%%

     For the analytical results presented in this section it is preferable 
to work with deviations from the far-field values. In the far-field the flow 
vanishes, and the pressure and potential vorticity take on the horizontally 
homogeneous values $\tilde{p}(\theta)$ and $\tilde{P}(\theta)$, which are 
related by 
 \begin{linenomath*}
\begin{equation}                       % Equation (4)
     \tilde{P} = f\left(-\frac{1}{g}\frac{\partial\tilde{p}}{\partial\theta}\right)^{-1}. 
\end{equation}
 \end{linenomath*}
Denoting the far-field density by $\tilde{\rho}(\theta)$, it is easily shown 
that $\theta\tilde{\rho}(d\tilde{\Pi}/d\tilde{p})=1$, where 
$\tilde{\Pi}=c_p(\tilde{p}/p_0)^{R/c_p}$. This allows (4) to be written in the form 
\begin{linenomath*}
\begin{equation}                      % Equation (5)
        \frac{gf}{\theta\tilde{\rho}\tilde{P}}
      + \frac{\partial\tilde{\Pi}}{\partial\theta} = 0.   
\end{equation} 
\end{linenomath*}
Taking the difference of (2) and (5), we obtain 
\begin{linenomath*}
\begin{equation}                      % Equation (6)
        \frac{\tilde{\rho}\tilde{P}}{\rho P}\frac{\partial v}{\partial x} 
      + \left(\frac{f\theta^2N^2}{g^2}\right)
        \frac{\partial\Pi'}{\partial\theta} = f\left(1 - \frac{\tilde{\rho}\tilde{P}}{\rho P}\right),  
\end{equation} 
\end{linenomath*}
where $\Pi'(x,\theta)=\Pi(x,\theta)-\tilde{\Pi}(\theta)$ is the Exner function anomaly, 
and where the buoyancy frequency $N(\theta)$ is defined by 
\begin{linenomath*}
\begin{equation}                      % Equation (7)
    N^2(\theta) = \frac{g^2}{\theta^2}\left(-\frac{d\tilde{\Pi}}{d\theta}\right)^{-1}.  
\end{equation} 
\end{linenomath*}

    In the analysis of this section, four simplifying assumptions are made: (i) the lower 
boundary is assumed to be an isentropic surface;  (ii) the factors $(\tilde{\rho}/\rho)$ 
in (6) are approximated by unity; (iii) on each isentropic surface the potential 
vorticity $P$ is assumed to be equal to its far-field value $\tilde{P}$, so that the 
right hand side of (6) vanishes and the coefficient of $(\partial v/\partial x)$ 
becomes unity; (iv) the reference state buoyancy frequency 
$N(\theta)$ is assumed to be inversely proportional to $\theta$, i.e., 
$N(\theta)=N_B \theta_B/\theta$, where $N_B$ and $\theta_B$ are constants, 
which results in a constant coefficient for the second term in (6).    
Assumptions (i), (ii), and (iv) will be relaxed in section 4. The relaxation 
of assumption (i) will prove crucial for the simulation of topographically bound 
cyclonic flows such as those illustrated in Figs.~1--3. 

    For the particular reference state $N(\theta)=N_B \theta_B/\theta$, 
the definition (7) can be integrated to obtain 
\begin{linenomath*}
\begin{equation}                    % Equation (8)
   \tilde{\Pi}(\theta) = c_p - \frac{g^2}{\theta_B^2 N_B^2}\left(\theta - \theta_B\right), 
\end{equation}
\end{linenomath*}
where $\tilde{p}(\theta_B)=p_0$.  The reference state hydrostatic equation $d\tilde{M}/d\theta=\tilde{\Pi}$ 
can then be integrated to obtain 
\begin{linenomath*}
\begin{equation}                    % Equation (9)
   \tilde{M}(\theta) = c_p \theta - \frac{g^2}{2\theta_B^2 N_B^2}
                             \left(\theta - \theta_B\right)^2,
\end{equation}
\end{linenomath*}
where we have assumed $\tilde{\phi}(\theta_B)=0$.
Since $\tilde{\phi}(\theta)=\tilde{M}(\theta)-\theta\tilde{\Pi}(\theta)$, we can 
use (8) and (9) to obtain 
\begin{linenomath*}
\begin{equation}                    % Equation (10)
   \tilde{\phi}(\theta) = \frac{g^2}{2\theta_B^2 N_B^2}\left(\theta^2 - \theta_B^2\right). 
\end{equation}
\end{linenomath*}

The relationship of pressure and potential temperature for the reference 
state, as determined from (8), is plotted in Fig.~5, where we have chosen 
$c_p=1004.5$ J kg$^{-1}$ K$^{-1}$, $g=9.8$ m s$^{-2}$, $\theta_B=295$K, 
and $(\theta_B/g)N_B^2=5.1373$ K km$^{-1}$.  This choice of $\theta_B$ 
and $N_B$ results in a profile with $\tilde{p}(\theta)=150$ hPa at $\theta=360$ K. 
The associated value of $N_B$ is $1.3064 \times 10^{-2}$ s$^{-1}$ and $\theta_T=360$K
is used throughout this study. 
Although idealized, this profile can be considered typical of the 
subtropics in the warm season.

     With the four simplifying assumptions listed above, the Cauchy-Riemann conditions 
(6) and (3) simplify to (11) and (12) below. We shall require that (11) and 
(12) hold in a region that includes an underlying 
topographic feature whose geopotential is specified by $\phi_S(x)$.  
As for boundary conditions, we require that 
$v$ and $\Pi'$ approach zero in the far-field, which is expressed in (13). 
We also require that the upper boundary is both an isentropic ($\theta=\theta_T$) 
and isobaric surface, which is expressed as (14). 
To formulate the lower boundary condition we combine the $x$-derivative of 
$M-\theta\Pi=\phi$ with the geostrophic and hydrostatic relations to obtain 
$f[v-\theta(\partial v/\partial\theta)]=(\partial\phi/\partial x)$, which, 
when applied at $\theta=\theta_B$ yields (15). In summary, the elliptic problem is 
\begin{linenomath*}
\begin{equation}                    % Equation (11)
        \frac{\partial v}{\partial x} 
      + \left(\frac{f\theta_B^2 N_B^2}{g^2}\right)
        \frac{\partial \Pi'}{\partial\theta} = 0, 
\end{equation} 
\end{linenomath*}
\begin{linenomath*}
\begin{equation}                    % Equation (12)
       f\frac{\partial v}{\partial\theta} 
      - \frac{\partial\Pi'}{\partial x} = 0,  
\end{equation} 
\end{linenomath*}
with boundary conditions
\begin{linenomath*} 
\begin{equation}                     % Equation (13)
        v \to 0 \,\,\, {\rm and}  \,\,\,   \Pi' \to 0   \,\,\,  {\rm as} \,\,\,   x \to \pm\infty,  	
\end{equation}           
\end{linenomath*}
\begin{linenomath*}
\begin{equation}	             % Equation (14)
        \Pi' = 0                \,\, {\rm at} \,\,\, \theta = \theta_T, 
\end{equation}
\end{linenomath*}
\begin{linenomath*}
\begin{equation}                     % Equation (15)            
        f\left(v - \theta\frac{\partial v}{\partial\theta}\right) = \frac{d\phi_S(x)}{dx}    
	                       \,\, {\rm at} \,\,\, \theta = \theta_B.          
\end{equation} 
\end{linenomath*}
This elliptic problem can be solved using Fourier integral transforms.  The details can be 
found in the appendix.
For all the calculations presented in this paper  
we have chosen the geopotential on the lower boundary as 
\begin{linenomath*}
\begin{equation}                           % Equation (16)
   \phi_S(x) = gH e^{-x^2/a^2},  
\end{equation} 
\end{linenomath*}
where the constants $H$ and $a$ respectively specify the mountain height and width. Using (A9), (A10), and (A11) in (A2) and (A4), we obtain 
\begin{linenomath*}
\begin{equation}                    % Equation (17)
    v(x,\theta) = -\frac{gHa}{f\sqrt{\pi}}\int_0^\infty  k e^{-a^2 k^2/4}
                \left(\frac{e^{-\kappa(\theta-\theta_B)} + e^{-\kappa(2\theta_T-\theta_B-\theta)}}
		 {1 + \kappa\theta_B + (1 - \kappa\theta_B)e^{-2\kappa(\theta_T-\theta_B)}}\right)\sin(kx) \, dk,   
\end{equation}  
\end{linenomath*}


\begin{linenomath*}
\begin{equation}                    % Equation (18)
    \Pi'(x,\theta) = -\frac{g^2Ha}{f\theta_B N_B\sqrt{\pi}} \int_0^\infty k e^{-a^2 k^2/4}
                \left(\frac{e^{-\kappa(\theta-\theta_B)} - e^{-\kappa(2\theta_T-\theta_B-\theta)}}
	         {1 + \kappa\theta_B + (1 - \kappa\theta_B)e^{-2\kappa(\theta_T-\theta_B)}}\right)\cos(kx) \, dk.    
\end{equation}  
\end{linenomath*}

Equations (17) and (18) are the Fourier integral representations of the 
solutions to the invertibility problem (11)--(15). Plots of the solutions 
$v(x,\theta)$ and $\Pi'(x,\theta)$ are easily constructed through numerical 
evaluation of the Fourier integrals at each point of an array of points in 
$(x,\theta)$-space. In the construction of several of the figures shown here, 
we have chosen to display isolines of $v$ and $\theta$ in $(x,p)$-space  
in addition to isolines of $v$ and $p$ in $(x,\theta)$-space. Conversion between 
these two representations is simply a matter of interpolation. 

     Figure 6 shows a plot of the solutions $v(x,p)$ and $\theta(x,p)$ for 
$f=7.3\times 10^{-5}$ s$^{-1}$, $a=500$ km, and $H=1500$ m. Figure 7 shows a similar 
plot with $H=2500$ m. As can be seen from Fig.~7, when $H=2500$ m the layer 
$295 \le \theta \le 296$ K is almost massless near the mountain crest. In fact, 
for $H$ greater than 2560 m, this layer does become massless at the mountain crest. 
In other words, for this particular mountain shape and far-field reference state, 
the critical crest height is approximately $H_{\rm crit}=2560$ m.  Following an 
argument similar to that of Eliassen (1980), a formula for the critical height 
can be obtained as follows. When $H=H_{\rm crit}$, 
$(\partial\Pi/\partial\theta)=(\partial\tilde{\Pi}/\partial\theta)+(\partial\Pi'/\partial\theta)=0$ 
at the crest, or equivalently, using (8),  
\begin{linenomath*}
\begin{equation}                           % Equation (30)
      \frac{\partial\Pi'}{\partial\theta} = \frac{g^2}{\theta_B^2 N_B^2}
          \,\,\, {\rm at} \,\,\, x=0 \,\,\, {\rm and} \,\,\, \theta=\theta_B.   
\end{equation} 
\end{linenomath*}
Using (29) for $\Pi'$, with $H$ replaced by $H_{\rm crit}$, we can rewrite (30) as 
\begin{linenomath*}
\begin{equation}                          % Equation (31)
    H_{\rm crit} = \frac{fa}{N_B I(a)},  	
\end{equation} 
\end{linenomath*}
where  
\begin{linenomath*}
\begin{equation}                          % Equation (32)
    I(a) = \frac{1}{\sqrt{\pi}} \int_0^\infty  
                    \left(\frac{\hat{a}^2 \hat{\kappa}^2
		   \left\{1 + \exp[-2\hat{\kappa}(\theta_T-\theta_B)/\theta_B]\right\}} 
		           {1 + \hat{\kappa} + (1 - \hat{\kappa}) 
	                      \exp[-2\hat{\kappa}(\theta_T-\theta_B)/\theta_B]}\right) 
			    \exp\left(-\hat{a}^2\hat{\kappa}^2/4\right) \, d\hat{\kappa}, 	
\end{equation} 
\end{linenomath*}
with $\hat{a}=(fN_B/g)a$ denoting the dimensionless mountain width and with 
$\hat{\kappa}=\kappa\theta_B$. Figure 8 shows $H_{\rm crit}$ as a function 
of $a$ for five different values of $f$. These results indicate that broad mountains 
can be relatively high but still below the critical crest height.  They also indicate that the crest 
height of the central Andes is supercritical. 

     In concluding this section we note that the wind and mass fields 
displayed in Figs.~6 and 7 can also be interpreted as solutions of 
the following geostrophic adjustment problem (Eliassen 1980).   
Initially, a stably stratified fluid is in a state of rest over 
a level bottom surface on an $f$-plane. Over some arbitrary time 
interval, the bottom topography is raised to its final shape and 
any transient inertia-gravity waves are allowed to disperse away. 
The final adjusted wind and mass fields are then determined via the 
potential vorticity invertibility principle (11)--(15), with the 
bottom topography appearing in the lower boundary condition of this 
invertibility principle. If the crest height of the topography is 
not raised above the critical value, the lowest isentropic surface 
remains attached to the topography. However, if the crest of the 
topography is raised above the critical value, the topography punctures 
the lowest isentropic surfaces, so that potential temperature now varies 
along the lower boundary. In either case, the flow around the obstacle 
is anticyclonic, a result of Coriolis turning of fluid that flows 
away from the rising mountain crest. This anticyclonic, topographically 
bound, balanced flow is not useful for describing cyclonic 
flows such as those shown in Figs.~1--3. Thus, in the  
next section we consider the related problem of the balanced wind 
and mass fields near a topographic feature that has been heated 
by radiative processes, resulting in a cyclonic flow around the obstacle. 
This will allow us to interpret the South American low level jet 
and the Chilean low level jet as parts of the topographically bound 
motion associated with the PV anomaly produced by solar heating of the Andes.      

\section{Numerical solution for the case of a nonisentropic lower boundary}   %%%  Section 4  %%%

     The analytical solutions discussed in section 3 are for the special 
case in which the topographic surface is also an isentropic surface. 
When isentropic surfaces intersect the topographic surface, they can 
be considered to run just under the topographic surface with a pressure 
equal to the surface pressure, thereby forming a massless layer with 
infinite potential vorticity 
(i.e., $\partial p/\partial\theta \to 0$ and $P \to \infty$). 
Lorenz (1955) was the first to define such a massless layer just under the surface 
of the ground.  The inhomogenous potential temperature at a boundary was also 
treated by Bretherton (1966).  His approach was different from that of Lorenz in that 
Bretherton showed that a potential temperature gradient along a boundary could 
be replaced by a boundary with constant potential temperature provided a 
concentration of potential vorticity very close to the surface is included.  Bretherton�s 
conception of the massless layer was used by Hoskins et al. (1985), Thorpe (1985, 1986), and 
Schneider et al. (2003).  The approach of Lorenz was used by Andrews (1983), 
Fulton and Schubert (1991), and Schneider (2005).  Both approaches are equally 
valid and here we choose to follow the convention defined by Lorenz.
The isentropic surface that is just below the earth's surface 
over the topographic feature and is at the earth's surface in the 
far-field is labeled $\theta=\theta_B$. Then, defining $\theta_S(x)$ as 
the value of potential temperature on the topographic surface,  
the region $\theta_B < \theta < \theta_S(x)$ is the massless layer. 
When there is a massless layer, the $1/P$ factor 
in the first term of (2) takes on the value of zero (i.e., $P \to \infty$) 
in the massless layer. This introduces a variable coefficient effect that 
removes the advantages of using Fourier transforms. Thus, in this section, 
we solve a discretized (finite difference) version of the invertibility 
problem using an iterative method. In particular, we shall obtain solutions 
for the second and third cases shown schematically in Fig.~4.

     When the invertibility principle is solved using finite difference methods  
rather than Fourier transform methods, it is unnecessary to separate the fields 
into a far-field part and a deviation part, as was done in section 3.  
Thus, returning to (2) and making use of the geostrophic  
relation $fv=(\partial M/\partial x)$ and the hydrostatic relation  
$\Pi=(\partial M/\partial\theta)$, we obtain (33) below, where the 
density $\rho$ is given in terms of $M$ by (34). Note that the appearance 
of $\rho$ (as opposed to $\tilde{\rho}$) in the top line of (33) retains 
the previously discussed weak nonlinearity, which is easily incorporated 
into the iterative method. The top line of (33) holds above the 
massless layer, while in the massless layer $P \to \infty$ and the top 
line reduces to the bottom line of (33). To obtain the lateral 
boundary conditions (35) we have assumed that the far-field (i.e., $x=\pm L$)  
value of $M$ is equal to the specified function $\tilde{M}(\theta)$.  
To obtain the upper boundary condition (36) we have assumed that the 
upper isentropic surface ($\theta = \theta_T$) is also an isobaric surface
with a constant Exner function $\Pi_T$.  To formulate the lower boundary 
condition (37) we have applied the general relation 
$M-\theta(\partial M/\partial\theta)=\phi$ at $\theta=\theta_B$.   
In summary, the elliptic problem is 
\begin{linenomath*}
\begin{equation}                          % Equation (33)
   \begin{split}
        \frac{g}{f\theta\rho P}
        \left(f^2 + \frac{\partial^2 M}{\partial x^2}\right) 
     &+ \frac{\partial^2 M}{\partial\theta^2} = 0  
              \,\,\,\, {\rm for} \,\,\,\, \theta_S(x) < \theta \le \theta_T,  \\
     &  \frac{\partial^2 M}{\partial\theta^2} = 0  
	      \,\,\,\, {\rm for} \,\,\,\,   \theta_B \le \theta < \theta_S(x),
   \end{split}
\end{equation} 
\end{linenomath*}
%
\begin{linenomath*}
\begin{equation}                         % Equation (34)
   \rho = \frac{p_0}{R\theta}                             
         \left(\frac{1}{c_p}\frac{\partial M}{\partial\theta}\right)^{c_v/R},     
\end{equation} 
\end{linenomath*}
%
\begin{linenomath*}
\begin{equation}                         % Equation (35)
                M = \tilde{M}(\theta)             \,\,\, {\rm at} \,\,\,   x = \pm L,    
\end{equation}
\end{linenomath*}
%
\begin{linenomath*}
\begin{equation}                         % Equation (36)
        \frac{\partial M}{\partial\theta} = \Pi_T  \,\, {\rm at} \,\,\, \theta = \theta_T, 
\end{equation}
\end{linenomath*}
%
\begin{linenomath*}
\begin{equation}                         % Equation (37)
        M - \theta\frac{\partial M}{\partial\theta} = \phi_S(x)    
	                                         \,\, {\rm at} \,\,\, \theta = \theta_B,          
\end{equation}
\end{linenomath*}
where $c_v=c_p-R$ is the specific heat at constant volume. 

     To solve (33)--(37) for $M(x,\theta)$, we must specify the constants $\theta_B$, 
$\theta_T$, $\Pi_T$, $L$, and the functions $P(x,\theta)$, $\theta_S(x)$ $\phi_S(x)$, 
and $\tilde{M}(\theta)$.  In section 3 we assumed that $\theta_S(x)$ was equal to 
the constant $\theta_B$, so there was no massless layer. We also assumed that 
$P(x,\theta)=\tilde{P}(\theta)$, so the potential vorticity did not vary along 
$\theta$-surfaces. With these assumptions, the interior anticyclonic flow was 
determined entirely by $\phi_S(x)$ and $\tilde{M}(\theta)$. 

     Now consider the more 
general case where $\theta_S(x) \ne \theta_B$, $\phi_S(x) \ne 0$, 
$P(x,\theta)=\tilde{P}(\theta)$ for $\theta_S(x) < \theta \le \theta_T$, 
but $P(x,\theta) \to \infty$ for $\theta_B \le \theta < \theta_S(x)$. In 
this case there is a massless layer on the mountain slope.  However, even 
in this case there is a resting, horizontally homogeneous solution if the 
functions $\theta_S(x)$, $\phi_S(x)$, and $\tilde{M}(\theta)$ are specified 
in a particular way. This particular specification can be obtained by simply 
noting that $M(x,\theta)=\tilde{M}(\theta)$ for 
$\theta_S(x) \le \theta \le \theta_T$ is a resting solution. The Exner function 
and the geopotential associated with this solution are 
$\Pi(x,\theta)=\tilde{\Pi}(\theta)$ and $\phi(x,\theta)=\tilde{\phi}(\theta)$. 
If this geopotential field is to match the topography, we must have  
\begin{linenomath*}
\begin{equation}                             % Equation (38)
       \phi_S(x) = \tilde{\phi}(\theta_S(x)).  
\end{equation}
\end{linenomath*}
In other words, if $\phi_S(x)$, $\theta_S(x)$, and $\tilde{\phi}(\theta)$ 
are specified in such a way that (38) is satisfied, the solutions of the 
elliptic problem (33)--(37) are $v(x,\theta)=0$ and 
$\Pi(x,\theta)=\tilde{\Pi}(\theta)$ for $\theta_S(x) \le \theta \le \theta_T$. 
For the far-field profile specified by (10) and the Gaussian mountain 
specified by (26), the constraint (38) can be rearranged to 
\begin{linenomath*}
\begin{equation}                            % Equation (39)
   \theta_S(x) = \theta_B \left(1 + \frac{2N^2 H}{g} e^{-x^2/a^2}\right)^{1/2}. 
\end{equation}
\end{linenomath*}
This can serve as a useful check on any numerical procedure for solving 
(33)--(37).  This case is shown in Fig.~9.  It is interesting to note that the wind field in the massless 
layer does not vanish.  The wind field in the massless layer will not, in general 
vanish.   

     To discretize the problem (33)--(37) we introduce the grid points 
$(x_j,\, \theta_k)=(-L+j\Delta x,\, \theta_B+k\Delta\theta)$ with 
$j=0,1,\ldots,J$ and $k=0,1,\ldots,K$, where $\Delta x=2L/J$ and 
$\Delta\theta=(\theta_T-\theta_B)/K$. We then seek an approximate 
solution with gridpoint values $M_{j,k}$ satisfying the discrete equation 
\begin{linenomath*}
\begin{equation}                          % Equation (40)
   \begin{split}
     & A_{j,k}\left[(f\Delta x)^2 + M_{j-1,k} -2M_{j,k} + M_{j+1,k}\right]    \\ 
     &                  \qquad    + M_{j,k-1} -2M_{j,k} + M_{j,k+1} = 0, 
   \end{split}
\end{equation}
\end{linenomath*}
where the dimensionless coefficient $A_{j,k}$ is defined by 
\begin{linenomath*}
\begin{equation}                          % Equation (41)
   A_{j,k} = \frac{g(\Delta\theta)^2}{f\theta_k \rho_{j,k} P_{j,k}(\Delta x)^2}.   
\end{equation}
\end{linenomath*}
The discretized versions of (36) and (37) are
\begin{linenomath*} 
\begin{equation}                          % Equation (42)
     M_{j,K} - M_{j,K-1} = \Pi_T \Delta\theta, 
\end{equation}
\end{linenomath*}
%
\begin{linenomath*}
\begin{equation}                          % Equation (43)
     M_{j,0} - \frac{\theta_B}{\Delta\theta} 
	       \left(M_{j,1} - M_{j,0}\right) = \phi_S(x_j).
\end{equation}
\end{linenomath*}
%
Note that (40) applies both outside and inside the massless layer, with 
$A_{j,k} \ne 0$ outside the massless layer and $A_{j,k}=0$ inside the massless 
layer. Also note that the problem is nearly 
isotropic on the grid outside the massless layer if $A_{j,k}\approx 1$, which 
can serve as a rough guide for the choice of the ratio $\Delta\theta/\Delta x$. 
Using typical values of the quantities on the right hand side of (41), 
we obtain the rough guide 
$(\Delta\theta/\Delta x) \approx (1 \, {\rm K})/(57 \, {\rm km})$. For 
the numerical solutions presented here, we have chosen $\Delta x = 8$ km 
and $\Delta\theta = 0.14$ K, resulting in a grid with $J=750$ and $K=450$ 
for the domain $-4000 \le x \le 4000$ km and $295 \le \theta \le 360$ K. 
   

     We solve the discrete equations (40)--(43) using 
the following successive over-relaxation (SOR) procedure. 
Denoting the current estimate of $M_{j,k}$ by $\hat{M}_{j,k}$  
(not to be confused with the `hat' notation in 
section 3 to denote the Fourier component of a variable), 
and sweeping through the grid in lexicographic order, we first 
compute the current estimate of density from 
\begin{linenomath*}
\begin{equation}                          % Equation (44)
   \hat{\rho}_{j,k} = \frac{p_0}{R\theta_k} 
       \left(\frac{\hat{M}_{j,k+1} - \hat{M}_{j,k-1}}
                  {c_p 2\Delta\theta}\right)^{c_v/R},       
\end{equation}
\end{linenomath*}
%
the current estimate of the dimensionless coefficient from 
\begin{linenomath*}
\begin{equation}                          % Equation (45)
   \hat{A}_{j,k} = 
  \begin{cases}
   \displaystyle{\frac{g(\Delta\theta)^2}{f\theta_k \hat{\rho}_{j,k} P_{j,k}(\Delta x)^2}}
                          & {\rm if} \,\,\, \theta_S(x_j) < \theta_k < \theta_T     \\[1.5ex]
              0           & {\rm if} \,\,\, \theta_B      < \theta_k < \theta_S(x_j) 
  \end{cases}   
\end{equation}
\end{linenomath*}
%
and then the current residual from   
\begin{linenomath*}
\begin{equation}                           % Equation (46)                      
   \begin{split}
   \hat{r}_{j,k} 
        &= \hat{M}_{j,k-1} + \hat{M}_{j,k+1} 
	 - 2\left(1 + \hat{A}_{j,k}\right)\hat{M}_{j,k}    \\
	&+ \hat{A}_{j,k}\left[(f\Delta x)^2 + \hat{M}_{j-1,k} + \hat{M}_{j+1,k}\right].     
   \end{split}
\end{equation}
\end{linenomath*}
%
The solution estimate on the interior is then updated by 
\begin{linenomath*}
\begin{equation}                           % Equation (47)
   \hat{M}_{j,k} \leftarrow \hat{M}_{j,k}
                      + \frac{\omega\hat{r}_{j,k}}{2\left(1+\hat{A}_{j,k}\right)}, 
\end{equation}
\end{linenomath*}
%
where $\omega$ is the overrelaxation factor and (46) and (47) are computed at 
the grid points $1 \le j \le J-1, \,\, 1 \le k \le K-1$. Finally, the top 
and bottom boundary points are updated from the boundary conditions (42) 
and (43), written in the form 
\begin{linenomath*}
\begin{equation}                          % Equation (48)
     \hat{M}_{j,K} \leftarrow \hat{M}_{j,K-1} + \Pi_T \Delta\theta 
                        \quad {\rm for} \quad 1 \le j \le J-1,
\end{equation}
\end{linenomath*}
%
\begin{linenomath*}
\begin{equation}                          % Equation (49)
     \hat{M}_{j,0} \leftarrow \frac{(\theta_B/\Delta\theta)\hat{M}_{j,1} + \phi_S(x_j)} 
	                       {1 + (\theta_B/\Delta\theta)} 
	                 \quad {\rm for} \quad 1 \le j \le J-1.   
\end{equation}
\end{linenomath*}
Equations (44)--(49) are iterated, starting with the initial estimate 
$\hat{M}_{j,k}=\tilde{M}(\theta_k)$. This initial estimate does not change 
on the lateral boundaries $j=0$ and $j=J$.    
In order to gauge the convergence rate, we have monitored the norm of the residual 
as iteration proceeds. Experience shows that, when this norm has decreased by 
approximately two orders of magnitude, the wind field no longer significantly changes 
with further iterations.  All of the figures shown have 
been iterated until the residual has decreased by more than two orders of magnitude.     
Based on numerical tests, over-relaxation factors between $1.7$ and $1.95$ have 
been used.    

     The above iterative procedure determines the Montgomery potential in 
the entire domain for a given surface geopotential $\phi_S(x)$, a given 
surface potential temperature $\theta_S(x)$, and a given $\tilde{M}(\theta)$. 
From the Montgomery potential, the wind field can be easily recovered 
using geostrophic balance, and the pressure field can be computed using 
the hydrostatic approximation. As a ``null" test of this numerical procedure, 
we have specified $\tilde{M}(\theta)$ according to (9), $\phi_S(x)$ 
according to (26), and $\theta_S(x)$ according to (39). This combination 
should result in no flow and a horizontally homogeneous mass field outside 
the massless layer. 
The numerical solution is shown in Fig.~9, which agrees well with the 
expected result. Note that, when the solution is displayed in $(x,\theta)$-space 
(bottom panel of Fig.~9), there is an anticyclonic flow confined entirely to the massless layer 
so that this anticyclone does not appear when the solution is displayed in 
$(x,p)$-space (top panel).   

     Figure 10 shows the solution computed for an isentropic ridge of the 
same height (2500 m) as the ridge shown in Fig.~7.  A comparison of these 
two figures shows that the solution derived using Fourier transforms and the 
solution found with the above iterative procedure are similar, but 
the maximum winds in Fig.~10 are 1.8 m s$^{-1}$ stronger than those in Fig.~7.   
These figures are not expected to be identical because Fig.~7 was 
computed assuming the density was equal to the far-field density (which depends 
only on $\theta$) while Fig.~10 was computed with density as a function of 
$x$ and $\theta$.  The difference between Figs.~7 and 10 is primarily due to 
the difference in the right hand side of (6). Figure 7 results from the 
assumptions $(\tilde{\rho}/\rho) \to 1$ and $(\tilde{P}/P \to 1)$, so that 
the right hand side of (6) vanishes. Figure 10 results from the assumption 
$(\tilde{P}/P \to 1)$, so that the right hand side of (6) becomes 
$f(\tilde{\rho}/\rho)[(\rho/\tilde{\rho})-1]$, which is negative near 
the mountain crest since $\rho < \tilde{\rho}$ there.  This negative right 
hand side of (6) leads to a slightly stronger anticyclone in Fig.~10 
compared to Fig.~7. Note that the errors caused by the assumption  
$(\tilde{\rho}/\rho) \to 1$ are expected to be less for cases in which 
the crest height is much smaller than the critical crest height.     

     Figures 11--14 show the wind fields that result from the three 
simple cases represented schematically in Fig.~4.  An isentropic ridge with 
a crest height of 1800 m is shown in Fig.~11 and can be clearly identified 
by the fact that the isentropes do not intersect the ridge and the flow is 
anticyclonic.  A flat lower boundary with a potential temperature anomaly 
of 6 K is shown in Fig.~12, which results in a 12.7 m s$^{-1}$ cyclonic flow.  
Figure 13 combines these two forcing cases into a warm ridge with a crest 
height of 1800 m and a surface potential temperature anomaly of 6 K.  
Note that the isentropic lower surface (Fig.~11) has anticyclonic flow anchored over 
the ridge, but for the case of the flat lower boundary the flow is cyclonic. 
The warm ridge with a potential temperature anomaly of $6$ K has anticyclonic flow 
but when the potential temperature anomaly is increased to $12$ K (Fig.~14), the flow 
becomes cyclonic. This demonstrates the competing influences of the purely orographic 
forcing and the surface thermal forcing. Once the surface potential temperature anomaly 
has grown strong enough to overcome the anticyclonic flow associated with an 
isentropic obstacle, the wind field produced by a warm ridge 
becomes cyclonic and grows with the potential temperature anomaly.  
   

It can also be shown that as the width of the ridge
increases for a given potential temperature anomaly, the influence of the 
anomaly is spread out and effectively acts as a weaker anomaly.   

     It is also apparent in ($x,\theta$)-space that the isentropic ridge 
case does not contain a massless layer, but for the warm flat lower boundary 
and the warm ridge, a massless layer is present and marked as the area below 
the thick black line.  Pressure in the massless layer is independent of 
$\theta$, as is clear from the vertical isobars within the massless layers 
of Figs.~12--14.   
Although the pressure is independent of $\theta$ in the massless layer, 
it does vary with $x$, resulting in a nonzero wind field.  
The velocity is required to satisfy the lower boundary condition  
(37), which can be seen as the slope of the velocity increases with 
the magnitude of the velocity along the lower boundary.  The 
concept of nonzero velocity underneath the surface is 
admittedly strange, but is valid based on the governing equations.        

     For the case of an isentropic lower boundary the potential vorticity is uniform
on each $\theta$ surface.  This leads to an intuitive explanation 
for the wind field response to the mountain as seen in Figs.~6, 7, and 10.  
Recall that, in the absence of heating, mass cannot cross the isentropic 
surfaces.  This constrains how the mass field can adjust when the
isentropes are moved up or down.  One of the primary benefits to using 
isentropic coordinates is the simplicity of the
expression for potential vorticity.  The denominator of the potential 
vorticity is given by the pseudodensity $-(1/g)(\partial p/\partial\theta)$, 
which becomes smaller when the isentropes are compressed, as occurs over the crest 
of an isentropic ridge.  Because the potential vorticity is uniform,
the decrease in magnitude of the denominator implies the numerator also must 
decrease.  The only part of the numerator that can decrease in magnitude is 
$\partial v/\partial x$, so a negative isentropic relative vorticity  
is required, which is exactly what Figs.~6, 7, and 10 show.  It is apparent that the 
velocity gradient is greatest where the isentropes are the most compressed.  
Similar reasoning can be applied to the case of stretched isentropes with the only
difference being the sign of the isentropic relative vorticity.     

     The structure of the isentropes for the cases with a warm lower boundary is quite 
different.  The heating forces mass across isentropic surfaces.  As radiation 
heats the surface, an upslope motion is generated.  When the mass moves upward 
it is transferred from one isentropic layer to the next potentially warmer layer.  This is what 
physically causes the isentropes to bend down towards the heated surface.  The 
stretched isentropic layers generate a wind field that is oriented opposite 
to that which results from the isentropic case with compressed isentropes.

     Each of Figs.~11--14 show the wind maximum to be in the lowest layers of 
the fluid and to decay rapidly in the vertical and horizontal directions.  
This matches well with the basic characteristics of low-level jets.  In agreement
with the insights offered by the invertibility principle these figures show 
an increased gradient in the velocity field when the isentropes are either compressed 
or stretched. These figures clearly indicate that in the absence of other factors 
a sufficiently heated surface will result in a cyclonic wind field.

It is also clear that the jets of opposite sign on either side of the ridge are 
two components of the same response to the forcing.  A heated ridge does not lead 
to a single LLJ, but two LLJs that form a cyclonic or anticyclonic circulation.      


\section{Concluding remarks}
    
     The results presented here show that the surface potential vorticity 
singularity that is the result of the potential temperature anomalies 
along the heated topography plays a key role in the  
dynamics of atmospheric low-level jets.  This conclusion has been  
reached by solving the invertibility principle for the balanced 
response of a stratified fluid to forcing along the lower boundary.  
As shown in Fig.~4, we have studied three types of lower boundary forcing:  
an isentropic ridge along the lower boundary, in which case $\phi_S(x) \ne 0$
and $\theta_S(x)=\theta_B$; a locally heated, flat lower boundary, in which 
case $\phi_S(x)=0$ and $\theta_S(x) \ne \theta_B$; and a heated ridge, 
in which case $\phi_S(x) \ne 0$ and $\theta_S(x) \ne \theta_B$.  
Isentropic ridges produce an anticyclonic wind field. The closer the 
ridge height comes to the critical crest height, the tighter the isentropes 
are packed over the ridge and the stronger the corresponding wind field.  
In contrast, the case of a locally heated flat lower boundary results in a 
cyclonic wind field.  When these two cases are combined into the case 
of a heated ridge, the winds are reduced due to the competing effects 
of the orography ($\phi_S$) and the heating ($\theta_S$). For small 
potential temperature anomalies on the surface, the influence of $\phi_S$ 
is dominant and the wind field is anticyclonic.  However, for surface 
potential temperature anomalies that are larger and closer 
to those observed in YOTC, the heating along the lower boundary is the 
dominant effect, and the wind field is cyclonic.  As an example of 
these competing effects, the heated ridge 
shown in Fig.13, with a potential temperature anomaly of 6K, has an 
anticyclonic flow, while the heated ridge shown in Fig.~14, with a 
potential temperature anomaly of 12K, has a cyclonic flow.  Byerle and Paegle (2003) 
noted a change from cyclonic to anticyclonic flow during the North American winter. 
They attributed this to a change in the zonal flow and the interaction 
with the orography.  The results presented here suggest that this transition 
is due to the winter potential temperature anomaly dropping below the critical value 
that is needed for a cyclonic circulation.  
    
     Past studies of the South American LLJ and the Chilean Coastal LLJ 
have treated these as isolated phenomena. The results presented here clearly 
show that these two LLJs can be attributed to the PV anomaly that is generated 
by diabatic heating of the elevated terrain of the Andes. Although we have 
only treated $\phi_S(x)$ and $\theta_S(x)$ functions that are symmetric about 
$x=0$, the YOTC observations shown in Fig.~2 indicate that surface potential
temperatures are colder on the west side due to cold upwelling in the eastern 
Pacific Ocean. Incorporation of this asymmetry in the specified $\theta_S(x)$
leads to a corresponding asymmetry in the LLJs (i.e., a stronger Chilean LLJ). 

     Two limiting aspects of the present work are the $f$-plane and 
$y$-independent assumptions. This raises the obvious question as to 
whether it is possible to generalize the present theory to   
balanced flows on the sphere. Such a generalization has 
been given by Silvers (2011). The specified topography and the specified 
potential temperature on the lower boundary are then given by 
$\phi_S(\lambda,\mu)$ and $\theta_S(\lambda,\mu)$, where $\lambda$ is the 
longitude and $\mu$ is the sine of the latitude. The relation between the mass 
field and the wind field is assumed to be the local linear balance 
(Schubert et al.~2009) condition 
$M(\lambda,\mu,\theta)=2\Omega\mu \, \psi(\lambda,\mu,\theta)$, where 
$\psi(\lambda,\mu,\theta)$ is the streamfunction for the nondivergent 
(i.e., zero isentropic divergence) part of the flow. Then, the spherical 
generalization of (3) is  
\begin{linenomath*}
\begin{equation}                          % Equation (6.1)
   \begin{split}
        \frac{g}{\theta\rho P}
        \left(2\Omega\mu + \nabla^2 \psi\right) 
     &+ 2\Omega\mu\frac{\partial^2 \psi}{\partial\theta^2} = 0  
        \,\,\,\, {\rm for} \,\,\,\, \theta_S(\lambda,\mu) < \theta \le \theta_T,  \\
     &  2\Omega\mu\frac{\partial^2 \psi}{\partial\theta^2} = 0  
	\,\,\,\, {\rm for} \,\,\,\,   \theta_B \le \theta < \theta_S(\lambda,\mu),
   \end{split}
\end{equation} 
\end{linenomath*}
where $\nabla^2$ is the two-dimensional (fixed $\theta$) Laplacian operator 
on the sphere.  With appropriate boundary conditions, (50) constitutes an 
invertibility principle for topographically bound flows on the sphere. Solution 
of this three-dimensional elliptic problem allows treatment of much more realistic 
topography $\phi_S(\lambda,\mu)$ than was treated in sections 3 and 4. 



\begin{acknowledgment} 
We would like to thank Brian McNoldy, Scott Fulton, Thomas Birner, David Randall, Paul Ciesielski, 
Richard Johnson, Matthew Masarik,    
and Richard Taft for their valuable advice. This research has been supported 
by the National Science Foundation under Grant ATM-0837932 and 
under the Science and Technology Center for Multi-Scale Modeling of 
Atmospheric Processes, managed by Colorado State University 
through cooperative agreement No.~ATM-0425247.  Workstation computing 
resources were provided through a gift from the Hewlett-Packard Corporation.  
\end{acknowledgment}




% Use appendix}[A], {appendix}[B], etc. etc. in place of appendix if you have multiple appendixes.
%\ifthenelse{\boolean{dc}}
%{}
%{\clearpage}
\begin{appendix}
\section*{\begin{center}Analytic solution using Fourier integral transforms\end{center}}

      To solve the invertibility problem (11)--(15) we use Fourier integral transforms.  
For simplicity we assume that the specified function $\phi_S(x)$  
is symmetric in $x$, so that $\Pi'(x,\theta)$ is symmetric in $x$ and $v(x,\theta)$ is 
antisymmetric in $x$.  The Fourier sine transform pair for $v(x,\theta)$ is  
\begin{linenomath*}
\begin{equation}                    % Equation (A1)
   \hat{v}(k,\theta) = \frac{2}{\pi} \int_0^\infty  v(x,\theta)\sin(kx) \, dx, 
\end{equation}
\end{linenomath*}
\begin{linenomath*}
\begin{equation}                    % Equation (A2)
        v(x,\theta) = \int_0^\infty \hat{v}(k,\theta) \sin(kx) \, dk, 
\end{equation} 
\end{linenomath*}
while the Fourier cosine transform pair for $\Pi'(x,\theta)$ is 
\begin{linenomath*}
\begin{equation}                    % Equation (A3)
   \hat{\Pi}'(k,\theta) = \frac{2}{\pi} \int_0^\infty  \Pi'(x,\theta)\cos(kx) \, dx, 
\end{equation}
\end{linenomath*}

\begin{linenomath*}
\begin{equation}                    % Equation (A4)
         \Pi'(x,\theta) = \int_0^\infty \hat{\Pi}'(k,\theta) \cos(kx) \, dk.
\end{equation} 
\end{linenomath*}
A similar cosine transform pair exists for the surface geopotential $\phi_S(x)$ and its 
transform $\hat{\phi}_S(k)$. 

     We now wish to Fourier transform (11), (12), (14), and (15). To Fourier 
transform (11) and (14) we multiply them by $\cos(kx)$ and integrate over $x$ 
from $0$ to $\infty$, thereby obtaining (A5) and (A7) below.  To Fourier transform 
(12) and (15) we multiply them by $\sin(kx)$ and integrate over $x$ from $0$ to $\infty$, 
thereby obtaining (A6) and (A8) below. In summary, the Fourier transform of the elliptic 
problem (11)--(15) is
\begin{linenomath*} 
\begin{equation}                    % Equation (A5)
     k\hat{v} + \left(\frac{f\theta_B^2 N_B^2}{g^2}\right) \frac{d\hat{\Pi}'}{d\theta} = 0,    
\end{equation} 
\end{linenomath*}

\begin{linenomath*}
\begin{equation}                    % Equation (A6) 
         f\frac{d\hat{v} }{d\theta}  + k \hat{\Pi}'  = 0,  
\end{equation} 
\end{linenomath*}
with boundary conditions
\begin{linenomath*} 
\begin{equation}                    % Equation (A7)
        \hat{\Pi}' = 0               \,\,\, {\rm at}  \,\,\, \theta = \theta_T,  
\end{equation}
\end{linenomath*}

\begin{linenomath*}
\begin{equation}                    % Equation (A8)	
       \theta\frac{d\hat{v}}{d\theta} - \hat{v} = \frac{k}{f}\hat{\phi}_S(k)   
                                      \,\,\, {\rm at}  \,\,\, \theta = \theta_B.           
\end{equation}
\end{linenomath*}
    As can be confirmed by direct substitution, the solutions 
of (A5)--(A8) are\footnote{The solutions of (A5)--(A8) can also be expressed 
in terms of $\cosh(\kappa\theta)$ and $\sinh(\kappa\theta)$, but for large horizontal 
wavenumbers, numerical evaluation of the hyperbolic representation of the solution 
presents a practical difficulty since the ratio of very large numbers is involved. 
This difficulty does not occur in the exponential representation (A9)--(A10), 
since this representation avoids evaluation of exponential functions with large 
positive exponents.} 
\begin{linenomath*}      
\begin{equation}                    % Equation (A9)
    \hat{v}(k,\theta) = -\frac{k}{f}\hat{\phi}_S(k)
        \left(\frac{e^{-\kappa(\theta-\theta_B)} + e^{-\kappa(2\theta_T-\theta_B-\theta)}}
         {1 + \kappa\theta_B + (1 - \kappa\theta_B)e^{-2\kappa(\theta_T-\theta_B)}}\right),   
\end{equation}  
\end{linenomath*}

\begin{linenomath*}
\begin{equation}                    % Equation (A10)
    \frac{\theta_B N_B}{g}\hat{\Pi}'(k,\theta)   
     = -\frac{k}{f}\hat{\phi}_S(k)
        \left(\frac{e^{-\kappa(\theta-\theta_B)} - e^{-\kappa(2\theta_T-\theta_B-\theta)}}
         {1 + \kappa\theta_B + (1 - \kappa\theta_B)e^{-2\kappa(\theta_T-\theta_B)}}\right).    
\end{equation} 
\end{linenomath*}
where $\kappa(k) = gk/(f\theta_B N_B)$. 
The 
Fourier cosine transform of (16) yields
\begin{linenomath*} 
\begin{equation}                           % Equation (A11)
   \hat{\phi}_S(k) = \frac{gH a}{\sqrt{\pi}} e^{-a^2 k^2/4}.  
\end{equation} 
\end{linenomath*}

\end{appendix}
%\subsection{First appendix secondary heading}
%
%\subsection{Second appendix secondary heading}
%
%\subsubsection{First appendix tertiary heading}
%
%\subsubsection{Second appendix tertiary heading}
%
%\paragraph{First appendix quaternary heading}
%
%\paragraph{Second appendix quaternary heading}
%
%\end{appendix}

% Create a bibliography directory and place your .bib file there.
% -REMOVE ALL DIRECTORY PATHS TO REFERENCE FILES BEFORE SUBMITTING TO THE AMS FOR PEER REVIEW
%\ifthenelse{\boolean{dc}}
%{}
%{\clearpage}
%\bibliographystyle{ametsoc}
%\bibliography{references}

\section{References}
  
%Byerle, L.~A., and J. Paegle, 2002: Description of the seasonal cycle of low-level flows flanking the Andes
%and their interannual variability. {\sl Meteorologica}, {\bf 27}, 71--88.

Bonner, W.~D., 1968: Climatology of the low level jet. 
{\sl Mon.\ Wea.\ Rev.}, {\bf 96}, 833--850. 
  
Byerle, L.~A., and J. Paegle, 2003: Modulation of the Great Plains low-level jet and moisture transports 
by orography and large-scale circulations. {\sl J.\ Geophys.\ Res.}, {\bf 108}, 8611 %, doi:10.1029/2002JD003005.
  
Eliassen, A., 1980: Balanced motion of a stratified, rotating fluid induced by bottom topography.  
{\sl Tellus}, {\bf 32}, 537--547. 
    
Jiang, X., N.-C. Lau, I. M. Held, and J. J. Ploshay, 2007: Mechanisms of the Great Plains 
low-level jet as simulated in an AGCM. {\sl J.\ Atmos.\ Sci.}, {\bf 64}, 532--547.  
%doi: 10.1175/JAS3847.1  
  
Kanamitsu, M., W. Ebisuzaki, J. Woollen, S.-K. Yang, J. J. Hnilo, M. Fiorino, 
and G. L. Potter, 2002: NCEP-DOE AMIP-II Reanalysis (R-2). 
{\sl Bull.\ Amer.\ Meteor.\ Soc.}, {\bf 83}, 1631--1643. 
 
Nogu\'es-Paegle, J., J. Paegle, and Coauthors, 2001: American low level jets---A scientific 
prospectus and implementation plan. 
Available online at \\ http://www.clivar.org/organization/vamos/Publications/alls.pdf

Schubert, W.H., R.K. Taft, and L.G. Silvers, 2009: Shallow water quasi-geostrophic theory on the sphere.
{\sl J. Adv. Model. Earth Syst.},{\bf 1}(2),17 pp. % doi: 10.3894/JAMES.2009.1.2.
 	
Silvers, L.~G., 2011: A theory of topographically bound balanced motions and 
application to atmospheric low-level jets. Ph.D. Dissertation, Department 
of Atmospheric Science, Colorado State University. 

Stensrud, D. J., 1996: Importance of low-level jets to climate: A review. 
{\sl J.\ Climate}, {\bf 9}, 1698--1711. 	
  	
Tarasova, T. A., J. P. R. Fernandez, I. A. Pisnichenko, J. A. Marengo, J. C. Ceballos, and M. J. Bottino, 2006: 
Impact of new solar radiation parameterization in the Eta Model on the simulation of summer climate over South America. {\sl J. Applied Meteorology and Climatology}, {\bf 45}, 318--333. 
% doi: 10.1175/JAM2342.1

%Ting, M., and Hailan Wang, 2006: The role of the North American topography on the maintenance of the 
%Great Plains summer Low-Level Jet.  {\sl J. Atmos. \ Sci.}, {\bf 63}, 1056--1068.
  
Vera, C., J. Baez, M. Douglas, C.~B. Emmanuel, J. Marengo, J. Meitin, M. Nicolini, 
J. Nogues-Paegle, J. Paegle, O. Penalba, P. Salio, C. Saulo, M.~A. Silva Dias, 
P. Silva Dias, and E. Zipser, 2006: The South American low-level jet experiment.  
{\sl Bull.\ Amer.\ Meteor.\ Soc.}, {\bf 87}, 63--77. 
  
Waliser, D., M. Moncrieff, D. Burridge, and Coauthors, 2011: The `Year' of tropical convection 
(May 2008 to April 2010): Climate variability and weather highlights. {\sl Bull.\ Amer.\ Meteor.\ Soc.}, 
{\sl submitted}. 

  
%%%%%%%%%%%%%%%%%%%%%%%%%%%%%%%%%%%%%%%%%%%%%%%%%%%%%%%%%%%%%%%%%%%%%
% FIGURES-REMOVE ALL DIRECTORY PATHS TO FIGURE FILES BEFORE SUBMITTING TO THE AMS FOR PEER REVIEW
%%%%%%%%%%%%%%%%%%%%%%%%%%%%%%%%%%%%%%%%%%%%%%%%%%%%%%%%%%%%%%%%%%%%%
\begin{figure}[t]                             % Figure 1
  \noindent\includegraphics[width=19pc,angle=0]{meanwind925_combined.eps}
\caption{January 2003 mean 925 hPa wind (scale of 10 m s$^{-1}$ vector 
 indicated just below the colorbar) from the NCEP-DOE Reanalysis II data 
 (Kanamitsu et al. 2002), which has the horizontal/vertical 
 resolution T62/L28.  Vectors are not drawn in the region of the Andes 
 above 925 hPa (at T62 horizontal resolution).  The background map shows 
 topography (see colorbar, scale is 100's m), with the La Plata river 
 basin outlined in red.  Adapted from Tarasova et al.~(2006) and 
 Nogu\`es-Paegle et al.~(2001).}
\end{figure}


%
%
\begin{figure}[t]                             % Figure 2
  \noindent\includegraphics[width=19pc,angle=0]{andesxsec.eps}
\caption{Mean warm season analysis of meridional wind at 21 S (top) and 30 S 
(bottom) from the ECMWF  Year of Tropical Convection (YOTC) data.  The magnitude of the meridional wind is 
given by the colorbar; 
black contours are isentropes.}
\end{figure}   
%
\begin{figure}[t]                              % Figure 3 
  \noindent\includegraphics[width=19pc,angle=0]{rockiesxsec.eps}
\caption{Mean warm season reanalysis meridional wind field at 30 N (top) and 35 N 
(bottom) from the ECMWF Year of Tropical Convection (YOTC) data.  The magnitude of wind is given by 
the colorbar, black contours are isentropes.}
\end{figure} 

\begin{figure}[t]                 % Figure 4
  \noindent\includegraphics[width=39pc,angle=0]{schematicfinal.eps}
\caption{Schematic vertical cross sections representing the three 
idealized configurations studied here. Contours represent lines of 
constant potential temperature (isentropes), with the vertical axes 
interpreted as height. On the left is an isentropic obstacle, the 
middle is a heated flat lower surface, and the right is a heated obstacle.}
\end{figure}


\begin{figure}[t]                             % Figure 5 
\noindent\includegraphics[width=19pc,angle=0]{refstatB.eps}
\caption{Reference pressure $\tilde{p}$ as a function of $\theta$, computed from (8).}
\end{figure}

\begin{figure}[t]                      % Figure 6
\noindent\includegraphics[width=19pc,angle=0]{vim_px_H1500_lat30.eps}
\caption{Cross-section plot of meridional wind ($v$, colors) and potential temperature 
($\theta$, solid lines) near a `Gaussian isentropic mountain' with half-width $a=500$ km 
and crest height $H=1500$ m. The potential temperature on the mountain surface is 295 K, and the flow 
is anticyclonic. The contour intervals are 1 m s$^{-1}$ for $v$ and 2 K for $\theta$.  
The extreme values of $v$ are $+11.5$ m s$^{-1}$ on the west side (dark red) of the mountain 
and $-11.5$ m s$^{-1}$ on the east side (dark blue).}
\end{figure}


\begin{figure}[t]                      % Figure 7
\noindent\includegraphics[width=19pc,angle=0]{vim_px_H2500_lat30.eps}
\caption{Same as Fig.~6, but with crest height $H=2500$ m.   
The peak values of $v$ are $\pm19.2$ m s$^{-1}$.}
\end{figure}

\begin{figure}[t]                      % Figure 8 
\noindent\includegraphics[width=19pc,angle=0]{crit_Hvs_width.eps}
\caption{Critical height $H_{\rm crit}$ of a Gaussian mountain as a function of half-width 
at five latitudes.}
\end{figure}

\begin{figure}[t]                               % Figure 9 
\noindent\includegraphics[width=19pc,angle=0]{zerowind1.eps}
\caption{A null case (i.e., no flow), in which $\tilde{\phi}(\theta)$, 
$\phi_S(x)$, and $\theta_S(x)$ are specified such that (38) is satisfied. 
In this particular null case, $\tilde{\phi}(\theta)$ is given by (10), 
$\phi_S(x)$ by (26), and $\theta_S(x)$ by (39). Black contours are 
isentropes (top) and isobars (bottom). The anticyclone in the massless 
layer of the bottom panel is not apparent in the top panel since
the massless layer has ``zero thickness" in $(x,p)$-space.}
\end{figure}  

\begin{figure}[t]                     % Figure 10 
\noindent\includegraphics[width=19pc,angle=0]{SORtest.eps}
\caption{Meridional wind field (shading, warm colors are 
positive, cool colors negative, and the contour interval is 1 m s$^{-1}$) for an 
isentropic lower boundary computed using SOR.  The maximum 
height of the Gaussian mountain is $2500$ m and its halfwidth is 500 km. 
The maximum winds are 21.0 m s$^{-1}$.}
\end{figure}  

\begin{figure}[t]                               % Figure 11
\noindent\includegraphics[width=19pc,angle=0]{ist1c_lap.eps}
\caption{Meridional wind field (shading, 1 m s$^{-1}$ intervals) for an isentropic 
ridge with $H=1800$ m and a Gaussian half-width of $a=600$ km. The flow is anticyclonic with $v_{\rm max}=15.4$ m s$^{-1}$.  
Black contours are isentropes (top) and isobars (bottom).}
\end{figure}  

\begin{figure}[t]                          % Figure 12 
\noindent\includegraphics[width=19pc,angle=0]{flb4c.eps}
\caption{Meridional wind field (shading, 1 m s$^{-1}$ intervals) for a  
flat lower surface having a 6 K  warm temperature anomaly with a Gaussian halfwidth 
of $a=600$ km. The resulting winds have $v_{\rm max}=12.7$ m s$^{-1}$.  The massless 
layer in ($x,p$)-space (top) is indicated by black object along the lower boundary while 
in ($x,\theta$)-space (bottom) it is indicated by the thick black line.  
Contour spacing for the pressure field in the massless layer is 1 hPa.}
\end{figure} 

\begin{figure}[t]                         % Figure 13 
\noindent\includegraphics[width=19pc,angle=0]{hr11.eps}
\caption{Meridional wind field (shading, 1 m s$^{-1}$ intervals) for 
a heated ridge with a 6 K potential temperature anomaly, $H=1800$ m, 
$a=600$ km, $v_{\rm max}=11.13$ m s$^{-1}$.  The massless layer is 
indicated by the black object (top) and the thick black line (bottom).  
Halfwidth of ridge is 900 km. Contour spacing of isobars(bottom)  
is 50 hPa except for lowest contour of 975 hPa.}
\end{figure}    

%\begin{figure}[t]                        % Figure 14 
%\noindent\includegraphics[width=29pc,angle=0]{flb5b.eps}
%\caption{Same as Fig.~11 except with a 12 K surface potential temperature anomaly.  
%The flow is cyclonic with $v_{\rm max}=27.27$ m s$^{-1}$.  In the lower panel the 
%contour spacing for pressure is 50 hPa above the massless layer and 4 hPa in the 
%massless layer.}
%\end{figure}   

 \begin{figure}[t]                      % Figure 14
\noindent\includegraphics[width=19pc,angle=0]{hr6b.eps}
\caption{Same as Fig.~12 except with a 12 K surface potential temperature anomaly.  
The isentropes now bend down toward higher pressure and the flow is cyclonic 
with $v_{\rm max}=10.3$ m s$^{-1}$.}
\end{figure}



%%%%%%%%%%%%%%%%%%%%%%%%%%%%%%%%%%%%%%%%%%%%%%%%%%%%%%%%%%%%%%%%%%%%%
% TABLES
%%%%%%%%%%%%%%%%%%%%%%%%%%%%%%%%%%%%%%%%%%%%%%%%%%%%%%%%%%%%%%%%%%%%%
%\begin{table}[t]
%\caption{This is a sample table caption and table layout.  Enter as many tables as
%  necessary at the end of your manuscript. Table from Lorenz (1963).}\label{t1}
%\begin{center}
%\begin{tabular}{ccccrrcrc}
%\hline\hline
%$N$ & $X$ & $Y$ & $Z$\\
%\hline
% 0000 & 0000 & 0010 & 0000 \\
% 0005 & 0004 & 0012 & 0000 \\
% 0010 & 0009 & 0020 & 0000 \\
% 0015 & 0016 & 0036 & 0002 \\
% 0020 & 0030 & 0066 & 0007 \\
% 0025 & 0054 & 0115 & 0024 \\
%\hline
%\end{tabular}
%\end{center}
%\end{table}
%
\end{document}


%%%%%%%%%%%%%%   Extra Pieces   *******************

     The basic characteristics of the wind field from the YOTC data shown 
in Figs.~3--4 agrees well with the results shown in Fig.~14.  The orientation 
of the wind field (cyclonic)  is the same and the magnitude is similar to that 
shown by the YOTC data. Because a steady state has been assumed the time 
evolution of potential vorticity and wind can obviously not be seen.  This 
model only computes the wind field that results from the potential vorticity 
at an instant in time. Alternately, this could be interpreted as the potential 
vorticity averaged over a time interval.  The figures in section 5 can be 
interpreted as representing means of roughly 1-3 months.  If the temperature 
distribution along the lower boundary is made larger, say 20K 
(see Jiang et al.~2010), the wind speeds are closer to what would be expected 
for daily or weekly mean fields. The parameters used to compute all of the 
wind fields shown in section 5 were chosen to be comparable to Earth-like 
mountains with realistic temperature anomalies along the lower boundary.  
It is probably not appropriate to make direct comparisons between the results 
of section 5 and observations of specific jet events or even climatologies 
because the assumptions of a steady, nonviscous fluid will certainly lead to 
discrepancies between the computed and observed wind fields, especially along 
the lower boundary, which is precisely where the LLJ maxima in section 5 are 
located. Performing a comprehensive comparison between these results and 
observations of LLJs is beyond the scope of this work.  
  