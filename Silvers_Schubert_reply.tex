
\documentclass[11pt]{article}

\setlength{\headheight}{0in}
\setlength{\headsep}{0in}
\setlength{\footskip}{0.5in}
\setlength{\hoffset}{-1in}
\setlength{\voffset}{-1in}
\setlength{\oddsidemargin}{1.25in}
\setlength{\evensidemargin}{1.25in}
\setlength{\topmargin}{1.0in}
\setlength{\textwidth}{6.0in}
\setlength{\textheight}{9.0in}

\begin{document} 

\noindent Manuscript: ``A Theory of Topographically Bound Balanced Motions and Application to Atmospheric Low-Level Jets"

\noindent Authors:  L. G. Silvers and W. H. Schubert 

\noindent Date:  23 March 2012

\bigskip
\centerline{Response to Reviewer 1}

\vskip 0.5 cm


\noindent{\bf Major comments:} 

\medskip
\noindent{\bf (1)}  The reviewer makes some important points here. We have followed the reviewer's 
advice and moved to an appendix all the mathematical detail concerning the derivation 
of the analytical solution. We have retained in section 3 the important Figures 6--8 and 
the formulas (Equations (17), (18), (20)) from which they are constructed. This reorganization 
does streamline the body of the paper. Concerning the work of Eliassen, his argument is in 
density coordinates rather than isentropic coordinates, with assumptions that cause a loss of 
distinction between $p$ and $\Pi$.  The authors feel it is helpful to present the argument 
in isentropic coordinates because this provides a clarifying step towards 
understanding the dynamics of cyclonic flows around a mountain with a non-isentropic surface.  
  
\medskip
\noindent{\bf(2)}  We believe Reviewer 1 is basically correct in his statement about the 
fundamental assertions of the paper. Concering the first (that two dimensions are adequate), 
this seems acceptable to a first approximation, but note that, since real topography is very 
complicated, we included a brief discussion of the three dimensional, spherical case at the 
end of the paper. Concering the third (that surface potential temperature trumps interior PV), 
we have included some additional discussion (see also the response to major comment 3).    
We have spent the most time on the second point (that the time-mean low-level flows are 
balanced) and have followed the Reviewer's suggestion and computed the geostrophic winds 
from the YOTC geopotential field on isobaric surfaces. The result is that the actual winds 
and the geostrophic winds match well if a light horizontal smoother is applied to the 
geostropic winds. The fact that the geostrophic winds have some noise in them is
probably associated with the fact that the analysis fields are produced by interpolating 
(to isobaric surfaces) the results of the ECMWF hybrid vertical coordinate data. 
We have included some additional discussion of these points, but have tried to keep 
it brief, since it could easily take us on a tangent. 
  

\medskip
\noindent{\bf(3) }  The following text was added, starting in line 98: 
``A complete potential vorticity 
based analysis of LLJs would have to include both the 
potential vorticity at the surface that results from a gradient of potential temperature 
and the potential vorticity of the interior flow.  The surface component is related to 
radiative heating and the atmospheric stratification while the interior component 
is due to convective activity, absolute vorticity, or the stratification.  The interaction 
between the interior quasigeostrophic potential vorticity and the surface gradient of 
potential temperature was reviewed by Hoskins et al. (1985).  A nongeostrophic 
generalization was applied to a wake circulation on the lee side of a mountain by 
Schneider et al.~(2003).  We here focus on the surface potential vorticity.  We do not 
assume that the affect of the surface potential vorticity is greater than that of the 
interior potential vorticity, we are simply demonstrating that LLJs are impacted 
by this surface potential vorticity. ''

The following text was added, starting in line 61:
``Much of the previous research on LLJs can be loosely grouped into two categories.  
The first includes boundary layer processes and forcing on relatively short (diurnal-a few 
days) times scales.  Relevant studies include Blackadar (1957), Holton (1967), Jiang 
et al. (2007), and Rife et al. (2010).  These studies were particularly interested in 
determining the physical mechanisms behind the diurnal oscillation and the phase that 
is observed in many LLJs.  The second category deals mostly with the synoptic-scale, 
monthly or seasonal mean structure of LLJs and includes studies such as Wexler (1961), 
Byerle and Paegle (2002, 2003), and Ting and Wang (2006).  These two groups certainly 
have some overlap.  The research presented here examines the influence of 
topographically bound balanced motions on the synoptic-scale, warm season mean structure of LLJs.''

Also please see the response to reviewer 2 comments 6 and 15.  Text was added which references 
other appropriate studies and provides more historical background.  

Further concerning Ting and Wang (2006):
In the abstract of Ting and Wang (2006), they state: ``the effect of thermal forcing is negligible'' 
on the LLJ of the Great Plains.  However, at the end of section 3 in their paper they state, `both thermal 
forcing and transient eddy forcing due to the presence of North American topography play a secondary 
role in the generation and maintenance of the Great Plains LLJ.''  This is confusing to the reader because 
`negligible' is not generally interpreted as `secondary'.  We are not in this study claiming that the heating 
of mountains is the primary generator of LLJs, but simply that it is an additional mechanism that is 
important.  In this regard our conclusions do not seem to be in contradiction with the conclusions of 
Ting and Wang in their section 3.  We also agree with them with the result that one of the dominant 
effects of topography is to force a cyclonic center over the Rocky Mountains (see their pg 1061).  

However, the primary difficulty in direct comparisons of our work with other research on the topographic influences 
to the atmosphere is the fact that our model does not include a background flow, while the majority of other 
topographic studies (Ting and Wang, 2006 Rodwell and Hoskins 2001, etc.) do.  Often the fundamental results of 
these studies are dependent on the particular background flow that is assumed to be interacting with the 
topography.  We are simply computing the steady state response that must result from surface PV anomalies.  
The total atmospheric response to topography will include our results plus those of studies that include a nonzero 
background flow.  

\bigskip
\noindent{\bf Minor comments: }

Please note that several minor changes and corrections were made to the captions of the figures.  

\medskip
\noindent{\bf Figure 1.} Based on Reviewer 1's and Reviewer 2's comments, we have created 
a new Figure 1, using YOTC data.  

\medskip
\noindent{\bf Figure 1 caption.} The reanalysis data are every six hours, as is now noted in the caption. 

\medskip
\noindent{\bf Line 29. }  Added ``These are two-month mean warm season wind fields averaged using the 
times 0, 6, 12, and 18 UTC."  to lines 29 and 30. 

\medskip
\noindent{\bf Line 31} was changed from, ``The YOTC analysis is available from May 2008 through 
April 2010 (see Waliser et al.~2011).'' to, ``The YOTC analysis is available for the two year 
period between May 2008 and April 2010.  Although originally proposed to be a one year research 
program YOTC was extended for an additional year in order to capture a cycle of both La Ni\~na 
and El Ni\~no.'' The sentence,``Two-month means from the summer season in each hemisphere were 
chosen for display." was deleted because it is redundant after the above additions. 

\medskip 
\noindent{\bf Line 30. }  This was changed to ``with 15 irregularly spaced vertical levels''.  
This is the highest vertical resolution available when using YOTC data.  These levels are as 
follows in hPa: 1000, 950, 925, 900, 850, 800, 700, 600, 500, 400, 300, 250, 200, 150, and 100.  
We included the actual levels below 400 hPa in the caption of Fig.~2.

\medskip
\noindent{\bf Lines 36--38.}  We intended to imply that the colder ocean has an impact on the structure of the,
isentropes, not simply that the colder a surface is the lower the jet.  Clarifying this is a good suggestion.  
We removed the part ``Due to the relatively cold eastern Pacific" so 
that it now reads simply as ``The coastal LLJs tend to have a wind maximum that is closer to the 
surface than the jets to the east of the mountain ranges.''

\medskip
\noindent{\bf Lines 38--40.}   We have tried to soften this statement, and present our interpretation in a 
straightforward manner.  We are simply implying that the coastal LLJ is undoudtedly influenced by the 
Pacific anticyclone to the west.  This would also be true of the GPLLJ if it was nearer to the Atlantic 
anticyclone.

\medskip
\noindent{\bf Equation 2.}  We followed the reviewer's comment and left $f$ outside of the $x$-derivative.  

\medskip
\noindent{\bf Line 116.}  Equation (6) can be obtained by either taking the quotient or the difference of 
equations (2) and (5).  If the reviewer would like both of these possibilities mentioned in the 
text we could write, ``The following equation can be obtained by taking either the difference 
or the quotient of (2) and (5).'' in place of line 116.  

\medskip
\noindent{\bf Line 130ff.}  Added just after equation (8): where $\tilde{p}(\theta_B) = p_0$.

\medskip
\noindent{\bf Lines 142--143.} Good point. To better justify the reference state 
profile, we have compared it to Figs.~2 and 3, by adding the following. 
``Although idealized, this reference state profile, with $\theta=331$ K 
at 400 hPa and $\theta=295$ K at 1000 hPa, can be considered typical of the 
profiles found at different locations in Figs.~2 and 3, noting of course that 
the actual values of $\theta$ on the lower boundary in Figs.~2 and 3 have 
a relatively wide range because of the contrast between the cold ocean and 
the warm continent."


\medskip
\noindent{\bf Line 151.} We start with the definition $M-\theta\Pi=\phi$. We then differentiate 
this with respect to $x$ at fixed $\theta$, obtaining 
$(\partial M/\partial x)_\theta-\theta(\partial\Pi/\partial x)_\theta=(\partial\phi/\partial x)_\theta$. 
We then use the geostrophic equation and the thermal wind equation to obtain 
$f[v-\theta(\partial v/\partial\theta)]=(\partial\phi/\partial x)_\theta$, which applies on 
any $\theta$ surface. When it is applied at the $\theta=\theta_B$ surface, we get the lower 
boundary condition. We never get the term $\Pi(\partial\theta/\partial x)_\theta$, assuming 
that is what was meant by $\Pi(d\phi/dx)$ in the comment.  

\medskip
\noindent{\bf Lines 207--209.}  ``$\theta_T=360$K is used throughout this study."  was added to line 183.

\medskip
\noindent{\bf Lines 222-223.}  To better illustrate this we have added the following sentence. 
``For example, the actual crest heights shown in the two panels of Fig.~2 are approximately 
4200 m (at 21S) and 3700 m (at 30S), both of which are considerably higher than the critical 
crest heights from the two curves labeled $20^\circ$ and $30^\circ$ latitude in Fig.~8." 

\medskip
\noindent{\bf Line 247.}  Yes, we think your interpretation is better, and we 
have modified the text to that wording. 

\medskip
\noindent{\bf Line 322. }  To be more precise, the wording here was changed to 
$\Delta\theta=(\theta_T-\theta_B)/N_\theta=65/450=0.144\bar{4}$ K.   

\medskip
\noindent{\bf Lines 399--400.}  Concerning the pressure and wind fields in 
the massless layer, we have added the following near the end of section 4. 
``Obviously, the pressure and wind fields in the massless layer do 
not have direct physical meaning. They can be simply regarded as 
a consequence of formulating (and discretizing) the invertibility 
principle on a domain (and grid) that is uniform in $(x,\theta)$ space, 
even though $\theta$ is not uniform on the lower boundary."  


\medskip
\noindent{\bf Lines 416--419, and Lines 419--420. } The reviewer raises an important point here. 
Since our paper deals with the invertibility principle rather than the evolution equation 
for PV, we have rewritten this paragraph as follows. 
``The structure of the isentropes for the cases with a warm lower boundary is quite different.  
Near the mountain the isentropes bend downward toward the warm mountain crest. Isentropes that 
intersect the mountain can be considered to run along the mountain crest until they erupt on 
the other side of the mountain. With this interpretation, there is an infinitesimally thin
massless layer of infinite PV on the the mountain crest. Then, by the invertibility principle 
and with uniform PV along isentropes that do not intersect the mountain, there is a cyclonic 
flow with a low-level jet on each side of the mountain." 

\medskip
\noindent{\bf Lines 421--426.}   We have rewritten this paragraph as follows. 
``Each of Figs.~11--14 show the wind maxima to be in the lowest layers of 
the fluid and to decay rapidly in the vertical and horizontal directions.  
This matches fairly well with the basic characteristics of observed low-level jets, 
although inclusion of surface friction and boundary layer effects in the theoretical argument 
is probably necessary for accurate placement of the height of the jet maxima. 
In agreement with the insights offered by the invertibility principle, these 
figures show increased (decreased) mountain top vorticity when the isentropes 
are stretched (compressed). These figures clearly indicate that in the absence 
of other factors a sufficiently heated surface will result in a cyclonic wind field.
It is also clear that the jets of opposite sign on either side of the ridge are 
two parts of the response to the forcing, i.e., a sufficiently heated ridge leads  
to a pair LLJs that form a cyclonic couplet. Finally, it is important to remember that 
the balanced, potential vorticity arguments given here apply to the time-averaged 
flow and that there are important superimposed diurnal oscillations (not accurately 
captured by PV invertibility arguments) of the 
type studied by Blackadar (1957), Holton (1967), and Jiang et al.~(2007)."       


\medskip
\noindent{\bf Line 429. }  Yes, the term ``couplet" is a good one, and is now used. 

\medskip
\noindent{\bf Lines 431--432. } In response to the reviewer's comment we have changed  
``potential vorticity anomalies associated with heated topography'' 
to, ``thin sheet of surface potential vorticity that is the result of the potential temperature anomalies 
along the heated topography''

\medskip
\noindent{\bf Lines 451--453.} Good point. To clear this up we have modified this 
discussion as follows. ``Byerle and Paegle (2003) have discussed the seasonal 
transition from cyclonic to anticyclonic flow in the context of the Great Plains 
low-level jet.  The results presented here suggest that this transition 
is due to the winter potential temperature anomaly dropping below the 
critical value that is needed for a cyclonic circulation."
   

\bigskip
\bigskip

\noindent PS to Reviewer 1: Thank you for your detailed and insightful review, which has resulted in 
substantial improvements.   





\newpage

\noindent Manuscript: ``A Theory of Topographically Bound Balanced Motions and Application to Atmospheric Low-Level Jets"

\noindent Authors:  L. G. Silvers and W. H. Schubert  

\noindent Date:  23 March 2012

\bigskip

\centerline{Response to Reviewer 2}

\vskip 0.5 cm

\bigskip
\noindent{\bf Specific comments}

Please note that several minor changes and corrections were made to the captions of the figures.  

\medskip
\noindent{\bf 1.} In order to relate the present work with that of Hoskins et al.~(1985) 
and Thorpe (1986), the following text was inserted in the 
Introduction. ``The surface potential 
temperature anomaly shown in the center panel of Fig.~4 represents 
a similar balanced flow structure to those studied by Hoskins et al.~(1985) 
and Thorpe (1986). Their work solved the pressure coordinate form of the 
invertibility principle for synoptic disturbances in gradient wind balance, 
while the present study solves the isentropic coordinate form of the invertibility 
principle for synoptic disturbances in geostrophic balance."  The reference to 
Bretherton's work is included later (see number 6 below). 


\medskip
\noindent{\bf 2. and 3.} The reviewer makes an excellent point. We have produced a new 
Figure 1 that uses the higher resolution YOTC data.  ECMWF as the source of the YOTC data 
has now been added to the captions for figures 2 and 3. 

\medskip
\noindent{\bf 4.} Reviewer 2 raises an important point about interior convective 
heating and turbulence. To improve our presentation we have modified the discussion 
just under equation (7) by adding the following. 
 ``In the numerical calculations presented in section 4, only 
the assumption concerning the vertical profile of $N(\theta)$ will be
retained. In particular, the lower boundary will not be an isentrope, and the 
potential vorticity will not be uniform on those isentropes that intersect the
lower boundary, although we will still assume the potential vorticity is uniform 
on those isentropes that do not intersect the lower boundary. Thus, the
present work considers only those balanced flows associated with surface 
processes, and not balanced flows associated with interior potential vorticity anomalies.
However, it should be noted that cumulus convection and convective turbulence 
over elevated terrain will result in interior potential vorticity anomalies
and associated balanced flows that are not considerered here." 

\medskip
\noindent{\bf 5.} Both Reviewers have questioned the ``reality" of the isentropic 
lower boundary case. Following these suggestions, we have delegated most of
this discussion to the Appendix, so only a summary of the analysis now appears 
in section 3. This shortened section 3 is useful in understanding the
anticyclonic case and the concept of a critical height at which isentropes 
must be punctured by the mountain crest. 

\noindent{\bf 6.}  The following text was inserted in the first paragraph of section 4.  
``Lorenz (1955) was the first to define a massless 
layer, not in the context of potential vorticity dynamics but in the 
context of available potential energy. Later, Bretherton (1966) showed 
that a potential temperature gradient along a boundary could 
be replaced by a boundary with constant potential temperature provided a 
concentration of potential vorticity very close to the surface is included.  
Bretherton's concept of a dynamical massless layer has been used in 
many later studies, e.g., Hoskins et al.~(1985), Thorpe (1985, 1986), 
Andrews (1983), Fulton and Schubert (1991), Schneider et al.~(2003), 
and Schneider (2005). The isentropic surface that is just at the earth's 
surface over both the topographic feature and in the 
far-field is labeled $\theta=\theta_B$. Then, defining $\theta_S(x)$ as 
the actual value of potential temperature on the topographic surface,  
the region $\theta_B < \theta < \theta_S(x)$ is the massless layer." 
  

\medskip
\noindent{\bf 7.}  The sentence ``The wind field in the massless layer will not, 
in general, vanish when there is a nonvanishing pressure gradient in the massless 
layer." was changed to: ``The wind field in the massless layer will not, in general vanish." 

\medskip
\noindent{\bf 8. } The reviewer raises an excellent point. To remove attention 
from any fields in the massless layer, we have included the following statement 
near the end of section 4. ``Obviously, the pressure and wind fields in the massless layer do 
not have direct physical meaning. They can simply be regarded as 
a consequence of formulating (and discretizing) the invertibility 
principle on a domain (and grid) that is uniform in $(x,\theta)$ space, 
even though $\theta$ is not uniform on the lower boundary."   

\medskip
\noindent{\bf 9.} Good point. We considered making $f$ negative for application 
to the Andes. However, we decided to yield to the bias of northern hemisphere 
meteorologists and keep $f$ positive.  

\medskip
\noindent{\bf 10.}  We have tried to clarify the captions of several figures. 
The difference between above-ground and below-ground wind 
maxima certainly needed to be clarified. We have changed the caption of figure 
13 to point out that the maximum above-ground value is approximately 5 m s$^{-1}$. 

\medskip
\noindent{\bf 11.}  The reviewer raises a good point. We have adopted the reviewer's 
suggestion by eliminating the term ``orographically forced tendency for anticyclonic motion" 
from the somewhat altered discussion on page 20.

\medskip
\noindent{\bf 12.}  The following was added, beginning at the bottom of page 16: ``Based on the previous 
discussion of the critical crest height, and Fig.~2 it is clear that the 
Andes are not isentropic.  For a standard atmosphere, (8) can be used to
determine the appropriate range of surface potential temperature between 
the base and the peak of a mountain.  For a 2 km mountain $\Delta\theta$ can 
be shown to be about  7 K.  When the surface is heated due to radiation or 
other processes, as we expect for the Andes, $\Delta\theta$ will be larger.
Figure 2 can be used a rough estimate of how potential temperature varies 
along the surface.  At 30 S, $\Delta\theta$ on the west side is about 9 K and 
on the east side it is about 15 K.  From the null case shown in Fig.~9, it seems that a surface potential 
temperature anomaly of more than 9 K (i.e., an increase from 295 K in the far 
field to 304 K at the mountain crest) is required to produce a cyclone.  
The parameters used in Figures 11--14 were 
loosely based on this reasoning."       


\medskip
\noindent{\bf 13.}  In order to relate the model solutions more closely with 
the observations that were shown in Figs.~2 and 3, we have included the following. 
``Even though the result shown in Fig.~14 is from a highly idealized model, it is 
useful to make a rough comparison with the observations shown in Figs.~2 and
3, noting that the sign of $v$ must be reversed (because of the sign of $f$) 
when comparing Fig.~14 with Fig.~2 and that the mountain height and width used 
in Fig.~14 lie between those of the Andes and the Rockies. The strength and 
location of the low-level jets in Fig.~14 are in rough agreement with those in
Figs.~2 and 3, although the observed jets are slighly elevated off the surface, 
perhaps due to surface frictional effects that are not included in the model. 
Two interesting aspects of Figs.~2 and 3 are the complexity of the topography 
and the E-W asymmetry of the $\theta_S(x)$ field, which is related to the cold
sea surface temperatures west of the continents. Although these refinements 
of the $\phi_S(x)$ and $\theta_S(x)$ fields can be included in the model, they
lie somewhat outside the scope of the present paper."  


\medskip
\noindent{\bf 14.}  Yes, this refers to the ``surface" line in the lower panels, 
as is now made clear in the text. 

\medskip
\noindent{\bf 15.}  We added to the second paragraph of section 5: ``Past studies of the South American LLJ (Vera et al. 2006) and the Chilean 
Coastal LLJ (Jiang et al. 2010) 
have treated these as isolated phenomena. This tendency dominates studies of 
the LLJs in North America as well (see Holton, 1967, Blackadar, 1957, Wexler, 1962, 
Jiang et al. 2007, etc.).''

\medskip
\noindent{\bf 16.}  We dropped the diabatic from `diabatic heating' as recommended.  The sentence in question 
was modified as follows, ``The results presented here clearly show that these two LLJs can be 
attributed to the PV anomaly that is generated by heating of elevated terrain.''

\medskip
\noindent{\bf 17.}  Capitalized ``Southern Hemisphere" on line 24. 


\bigskip
\bigskip

\noindent PS to Reviewer 2: Thank you for your insightful review.



\end{document}


