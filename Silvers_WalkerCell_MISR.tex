\documentclass[varwidth, border=10pt]{standalone}

\usepackage{caption}
\usepackage{subcaption}
\usepackage[labelformat=parens,labelsep=quad,skip=3pt]{caption}
\usepackage{graphicx}

\begin{document}
% for some reason, including a blank line between the subfigure blocks leads to a vertically aligned figure...

\begin{figure}
  \centering
  %\hspace{\fill}
  \begin{subfigure}{0.5\textwidth}
  \centering
    \includegraphics[width=\linewidth]{/Users/silvers/code/ncl_git/Full_3doms_10lat35_120lon300_map-eps-converted-to.pdf}
 \end{subfigure}
 %\hspace{\fill}
%\vspace{2pt}
 %\hspace{\fill}
\begin{subfigure}{0.31\textwidth}
\centering
  \includegraphics[width=\linewidth]{//Users/silvers/code/ncl_git/Full_3doms_10lat35_120lon300_Prof-eps-converted-to.pdf}
\end{subfigure}
%\caption{Observations of cloud fraction in the tropical Pacific from MISR over the years 2000-2019. Panel on the right shows the cloud fraction as a function of height from 
%a region dominated by subsidence (solid), the mean of the full domain shown in left panel (long dashes), and the mean from a region with deep convection (small dashes).  
%The latitudinal extent follows that used by Schwendike et al., 2014 in their definition of the regional Walker circulation (see their figure 2).}
 \end{figure}

\end{document}