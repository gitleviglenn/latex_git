\documentclass[11pt]{report}
% this is a shell to write new latex programs in....

% additional packages
%--------------------latex
\usepackage{amsmath}
\usepackage{times}
\usepackage{indentfirst}
\usepackage[dvips]{graphicx}

% page layout stuff (set as desired)
%------------------
\newlength{\lpgmargin} \setlength{\lpgmargin}{1.0in}
\newlength{\rpgmargin} \setlength{\rpgmargin}{1.0in}
\newlength{\tpgmargin} \setlength{\tpgmargin}{1.0in}
\newlength{\bpgmargin} \setlength{\bpgmargin}{1.0in}

\setlength{\headheight}{0.0in}
\setlength{\headsep}{0.0in}
\setlength{\footskip}{0.5in}

% page layout stuff (leave as is)
%------------------
\setlength{\hoffset}{-1in}
\setlength{\voffset}{-1in}
\setlength{\oddsidemargin}{\lpgmargin}
\setlength{\topmargin}{\tpgmargin}
\setlength{\marginparsep}{0in}
\setlength{\marginparwidth}{0in}
\setlength{\textwidth}{\paperwidth}
   \addtolength{\textwidth}{-\lpgmargin}
   \addtolength{\textwidth}{-\rpgmargin}
\setlength{\textheight}{\paperheight}
   \addtolength{\textheight}{-\tpgmargin}
   \addtolength{\textheight}{-\bpgmargin}
   \addtolength{\textheight}{-\headheight}
   \addtolength{\textheight}{-\headsep}
   \addtolength{\textheight}{-\footskip}

% page style stuff (set as desired)
%-----------------
%\pagestyle{plain}
\pagestyle{plain}

%=============================================================================

\begin{document}

%\centerline{\LARGE \textbf{TeX Shell}}
Paper Summary

Title: \textbf{The Generalization of Semigeostrophic Theory to the 
               $\beta$-Plane}

Authors: Gudrun Magnusdottir and Wayne Schubert

J. Atmos. Sci., 1990

\vspace{0.25in}

%\chapter{1. Introduction}
\begin{enumerate}
\item{\textbf{Introduction}}

Brian Hoskins combined the geostrophic momentum approximation with the
geostrophic coordinate transformation to derive the system of semigeostrophic
equations.  These are almost as simple as the geostrophic equations, but they
succeed at representing fluid behavior (fronts, jets, etc.) that is beyond the
range of physical validity in the geostrophic system.  This derivation depends on
the assumption of a constant Coriolis parameter.

A completely different approach to the semigeostrophic equations was developed
by Salmon, Shutts, and several others.  They started by using Hamiltonian
methods and were able to include a variable Coriolis parameter.  The stated goal
of the this paper by Magnusdottir and Schubert is ``to examine this question of a
variable Coriolis parameter in semigeostrophic theory, not through the use of
Hamiltonian methods, but by more conventional and elementary methods of
analysis.''  The derivation of a $\beta$-plane version of semigeostrophc theory
depends on the use of isentropic and geostrophic coordinates.  This allows the
entire system to be reduced to one predictive and one diagnostic equation.  The
potential pseudodensity is predicted and the inversion of this quantity leads to
the recovery of the balanced wind and mass fields.  
	
\item{\textbf{Semigeostrophic equations on the $\beta$-plane}}

The governing equations are 
\begin{equation}
    \frac{Du_g}{Dt}-(f(Y)v+\beta(y-Y)v_g)+\frac{\partial M}{\partial x}=0,
\end{equation}   

\begin{equation}
    \frac{Dv_g}{Dt}+(f(Y)u+\beta(y-Y)u_g)+\frac{\partial M}{\partial y}=0,
\end{equation}

\begin{equation}
    \frac{\partial M}{\partial \theta}=\Pi,
\end{equation}

\begin{equation}
    \frac{D\sigma}{Dt}+\sigma\left(\frac{\partial u}{\partial x}+
    \frac{\partial v}{\partial y}+\frac{\partial\dot{\theta}}{\partial\theta}
    \right)=0,
\end{equation}
\begin{equation}
    (v_g,-u_g)=\frac{1}{f(Y)}\left(\frac{\partial M}{\partial x},
    \frac{\partial M}{\partial y}\right),
\end{equation}
and 
\begin{equation}
    (X,Y)=\left(x+\frac{v_g}{f(Y)},y-\frac{u_g}{f(Y)}\right).
\end{equation}

\item{\textbf{Vorticity, potential vorticity, and potential pseudodensity
equations}}

The predictive equation in flux form is: 
\begin{equation}
    \frac{\partial\sigma^*}{\partial T}+\frac{\partial}{\partial X}\left(
    -\frac{\sigma^*}{f}\frac{\partial M^*}{\partial Y}\right)+
    \frac{\partial}{\partial Y}\left(\frac{\sigma^*}{f}
    \frac{\partial M^*}{\partial X}\right)+
    \frac{\partial}{\partial\Theta}(\sigma^*\dot{\theta})=0
\end{equation}
where $\sigma^*=(f/\zeta)\sigma$ and the total derivative in geostrophic space
has been used.  

\item{\textbf{Invertibility principle}}

The potential pseudodensity ($\sigma^*$) is a combination of the mass field and the wind field, both
of these are
related to $M^*$ and as a result, $\sigma^*$ depends only on $M^*$.  To do this we need to use the
geostrophic, hydrostatic, and coordinate transformation relations.  The invertibility system is
given by eqs. 4.3a,b,c.  Has this been solved?  A multigrid solver developed by Fulton (1989) was
mentioned to solve the $f$-plane case.  Is there any chance of a similar multigrid solver being
developed for the spherical version of semi-geostrophic case?  

\item{\textbf{Concluding remarks}} 

A closed system for predicting the potential pseudodensity has been derived.  The advantage of this
system of the quasi-geostrophic system is the implicit inclusion of the ageostrophic advection in
the coordinate transformation.  A general method of derivation for similar semigeostrophic systems
emerges from this paper.  This system starts with the geostrophic relations, geostrophic
coordinates, and the 'canonical' momentum equations.  These relations are then used to derive the
appropriate momentum equations, and the ageostrophic advection will automatically be implicit.  This
has already been extended to the hemispheric case in spherical coordinates.  Can a different balance
condition be assumed that allows a fully global theory to be developed?  

Geostrophic Relations: 
\begin{equation}
    \left(v_g,-u_g\right)=\frac{1}{f(Y)}\left(\frac{\partial M}{\partial x},
    \frac{\partial M}{\partial y}\right).
\end{equation}
Geostrophic Coordinates: 
\begin{equation}
    \left(X,Y\right)=\left(x+\frac{v_g}{f(Y)},y-\frac{u_g}{f(Y)}\right).
\end{equation}
The `desired' canonical momentum equation: 
\begin{equation}
    f\frac{DY}{Dt}=\frac{\partial M^*}{\partial X}, \qquad 
    -f\frac{DX}{Dt}=\frac{\partial M^*}{\partial Y}
\end{equation}
These equations can then be used to derive the corresponding zonal and meridional momentum equations
for the semigeostrophic system.  
\end{enumerate}

Paper Summary

Title: \textbf{Semigeostrophic Theory on the Hemisphere}

Authors: Gudrun Magnusdottir and Wayne Schubert

J. Atmos. Sci., 1991

\vspace{0.25in}

\begin{enumerate}

\item{\textbf{Introduction}}
The original semigeostrophic theory developed by Hoskins, only applied to the $f$-plane.  Using an
approach based on Hamilton's principle, Salmon and Shutts were able to include the effects of a
variable Coriolis parameter.  Using a more `conventional' approach, outlined above, Magnusdottir
and Schubert (1990) were able to generalize the theory to the $\beta$-plane.  In this paper, the
authors further generalize the geostrophic balance and coordinates to derive a hemispheric version
of the theory.  This theory is appropriate for the study of midlatitude dynamics when the
sphericity of the Earth is an important feature of the system.  The primary weakness of this theory
is its lack of validity in the equatorial regions.

``Both the definitions of the geostrophic wind and the spherical geostrophic coordinates are
generalizations of the $\beta$-plane definitions.''  Is this problematic?  The $\beta$-plane is
mathematically a bit odd and doesn't correspond to a physically realizeable situation.  Should the
hemispheric version of semigeostrophic theory be derived from a more physically sound base?  

\item{\textbf{Generalizations of the geostrophic momentum approximation and geostrophic 
coordinates}}

The equations for the spherical geostrophic momentum approximation reduce to the primitive
equations when ($u_g,v_g$) are replaced by ($u,v$) and $\Phi$ is replaced by $\phi$.  Maybe they
aren't really based on the $\beta$-plane equations in a problematic way as much as a simplification
of the primitive equations.  

The geostrophic relations are: 
\begin{equation}
    \left(v_g\frac{\cos\Phi}{\cos\phi},-u_g\frac{\cos\phi}{\cos\Phi}\right)=
    \frac{1}{2\Omega\sin\Phi}\left(\frac{\partial M}{a\cos\phi\partial\lambda},
    \frac{\partial M}{a\partial\phi}\right), 
\end{equation}    
the geostrophic coordinates are:
\begin{equation}
    (\Lambda,\sin\Phi)=\left(\lambda+\frac{v_g}{a2\Omega\sin\Phi\cos\Phi},
    sin\phi-\frac{u_g\cos\Phi}{a2\Omega\sin\Phi}\right).
\end{equation}

These definitions are justified because they lead to the canonical momentum equations
\begin{equation}
    2\Omega\sin\Phi a\frac{D\Phi}{Dt}=\frac{\partial M^*}{a\cos\Phi\partial\Lambda},
\end{equation}
\begin{equation}
    -2\Omega\sin\Phi a\cos\Phi\frac{D\Lambda}{Dt}=\frac{\partial M^*}{a\partial\Phi}
\end{equation}

This system maintains several fundamental conservation principles of the primitive equations.  The
absolute angular momentum, the kinetic energy (evaluated geostrophically), potential vorticity, and
potential pseudodensity.  

\end{enumerate}

Paper Summary

Title: \textbf{Vorticity Coordinates, Transformed Primitive Equations, and a Canonical Form for
Balanced Models}

Authors: Wayne Schubert and Gudrun Magnusdottir

J. Atmos. Sci., 1994

\vspace{0.25in}

Following the general method of derivation outlined above, this paper derives the generalized 
Geostrophic relations as: 
\begin{equation}
    u=\frac{\partial \chi}{\partial x}+\frac{1}{2}f\left[(X-x)\frac{\partial Y}{\partial x}
    -(Y-y)\left(\frac{\partial X}{\partial x}+1\right)\right],
\end{equation}
and 
\begin{equation}
    v=\frac{\partial \chi}{\partial y}+\frac{1}{2}f\left[(X-x)\frac{\partial Y}{\partial y}
    -(Y-y)\left(\frac{\partial X}{\partial y}+1\right)\right].
\end{equation}
These can be reduced to the `normal' geostrophic relations as discussed on page 3311. 
These equations are also known as the Clebsch representations of the velocity field.  The view of
the Clebsch equations being generalizations of the geostrophic coordinates is realized by makeing
the geostrophic approximation, can the same thing be done with the rotational part of the wind?    






Notes to self:  There are at least two important elements of semigeostrophic theory.  First, the
implicit inclusion of the ageostrophic wind allowing the semigeostrophic system to use the total
wind for advection rather than just the geostrophic part of the wind.  The second component is the
nonlinear invertibility principle. It seems likely that the first of these is much more important,
if we can keep the ageostrophic wind in the advective terms, but we have to linearize the
invertibility priniciple to solve it, the model will still be much better than the geostrophic
system.  










\end{document}
