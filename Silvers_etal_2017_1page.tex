\documentclass[11pt]{amsart}   	% use "amsart" instead of "article" for AMSLaTeX format
\usepackage{geometry}                		% See geometry.pdf to learn the layout options. There are lots.
\geometry{letterpaper}                   		% ... or a4paper or a5paper or ... 
%\geometry{landscape}                		% Activate for for rotated page geometry
%\usepackage[parfill]{parskip}    		% Activate to begin paragraphs with an empty line rather than an indent
\usepackage{graphicx}				% Use pdf, png, jpg, or eps� with pdflatex; use eps in DVI mode
								% TeX will automatically convert eps --> pdf in pdflatex		
\usepackage{amssymb}
\usepackage{gensymb}

\title{The diversity of cloud responses to twentieth-century sea surface temperatures}
\author{Levi G. Silvers, David Paynter, and Ming Zhao}
%\date{2016}							% Activate to display a given date or no date

\begin{document}
\maketitle
%\section{}
%\subsection{}

%\keypoints{
%\item Radiative Convective Equilibrium (RCE) is a useful framework to compare, contrast, and harmonize the two extremes of explicit and parameterized convection.
%\item Consistent climates across a range of domain sizes encourage the use of RCE to enhance our ability to model clouds (or cloud systems).  
%\item The response of low clouds to warming, and hence estimation of the climate feedback parameter, is not robust across configurations.
%}

\section{Abstract}
Low-level clouds are shown to be the conduit between the observed sea surface temperatures (SST) 
and large decadal fluctuations of the top of the atmosphere (TOA) radiative imbalance.  The influence 
of low-level clouds on the climate feedback is shown for global mean time
series as well as particular geographic regions.  The changes of clouds are found to be 
important for a mid-century period of high sensitivity and a late century period of low 
sensitivity.  These conclusions are drawn from analysis of amip-piForcing simulations using three 
atmospheric general circulation models (AM2.1, AM3, and AM4.0).  All three models confirm the importance 
of the relationship between the global climate sensitivity and the eastern Pacific 
trends of SST and low-level clouds.  However, this work argues that the variability of the 
climate feedback parameter is not driven by stratocumulus dominated regions in the eastern ocean basins, 
but rather by the cloudy response in the rest of the tropics.


\section{Key Points}
\begin{itemize}
    \item Three climate models driven by observed sea surface temperatures increase then decrease climate sensitivity over the twentieth century
\item Substantial changes of the climate feedback parameter are mirrored by the global mean low-level cloud anomalies
\item This variability is connected to atmospheric stability and not dominated by stratocumulus regions, but by the broader tropics
\end{itemize}

\begin{figure}
   \includegraphics[width=5.in]{agu/silvers_etal_2017/figures/alpha_tseries_6pan.pdf}
  \caption{Time series analysis of AM4.0(black), AM3(green), and AM2.1(blue).  Thick lines are ensemble means, shading gives range of ensemble members.  A) Global climate feedback parameter $-\alpha$ ($\rm{W/m^2 K}$),  
  B) Same as  A) but with contributions from 4 tropical windows (small-dashed) and the rest of tropics (long-dashed),  C)  $\alpha$ ($\rm{W/m^2 K}$) decomposed into the CRE (lower) and CLR (upper) components,   D)  Global mean Low-Cloud Cover parameter $LCC_{LR}$ ($\rm{\% /K}$).  Global $\alpha_{CRE}$ ($\rm{W/m^2 K}$) is decomposed into the short-wave CRE (E) and the long-wave CRE (F).   All time series computed by regression against global, yearly mean surface temperature change.}
  \label{fig:alphacre}
\end{figure}




\end{document}  