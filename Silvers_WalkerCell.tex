%\documentclass[11pt]{article}   	% use "amsart" instead of "article" for AMSLaTeX format
\documentclass[draft]{agujournal2019}
\usepackage{url} %this package should fix any errors with URLs in refs.
\usepackage{lineno}
\usepackage[inline]{trackchanges} %for better track changes. finalnew option will compile document with changes incorporated.
\usepackage{soul}
\linenumbers
%\usepackage{geometry}                		% See geometry.pdf to learn the layout options. There are lots.
%\geometry{letterpaper}                   		% ... or a4paper or a5paper or ... 
%\geometry{landscape}                		% Activate for for rotated page geometry
%\usepackage[parfill]{parskip}    		% Activate to begin paragraphs with an empty line rather than an indent
%\usepackage{graphicx}				% Use pdf, png, jpg, or eps§ with pdflatex; use eps in DVI mode
								% TeX will automatically convert eps --> pdf in pdflatex		
%\usepackage{amssymb}
%\usepackage{gensymb}
%\usepackage{epstopdf}
%\usepackage{pdflatex}
%\usepackage{epsfig}

\draftfalse

\journalname{Journal of Advances in Modeling Earth Systems (JAMES)}

\begin{document}

			% Activate to display a given date or no date
%\title{Using the Walker Circulation to connect low-level cloud changes to deep convective entrainment}
\title{Clouds and Radiation in a mock-Walker Circulation}
\authors{Levi G. Silvers, Nadir Jeevanjee, and Thomas Robinson}

\correspondingauthor{Levi Silvers}{levi.silvers@stonybrook.edu}

%\textbf{Key Points: choose three}
\begin{keypoints}
  \item{At 25 km, simulations with parameterized shallow- and deep-convection exhibit asymmetries in the circulation, 
  precipitation, and humidity fields resulting in a steady-state overturning circulation that is not centered on the SST
  maximum.}
  \item{Interactions between clouds and radiation combined with parameterized convection act to shift the precipitation 
  maximum away from the SST maximum.}
  \item{The longwave cloud-radiative effect is a fundamental factor in establishing these asymmetries and leads to dramatic
  changes in the low-level clouds and boundary layer structure.}
  \item{Cloud resolving simulations result in stronger overturning, more condensate aloft, and a RH in the deep convective 
  region that is 30\% higher than for the coarser cases with parameterized convection.}
  \item{The mock-Walker Circulation can be thought of as RCE plus an imposed gradient in surface temperature.}
\end{keypoints}


\begin{abstract}

% What are climate modelers doing to reduce uncertainties about cloud processes?
The representation of clouds in climate models is the most stubborn contributor to uncertainty of the climate response to increased 
greenhouse gases.  A few of the reasons clouds are difficult to represent in models are imprecise definitions of clouds, an existential 
dependence on trace amounts of condensed water, the importance of microphysical processes at scales less than the grid-scale, 
and the close coupling between atmospheric radiative fluxes, clouds, and the large-scale dynamical circulations of Earth.  To elucidate this 
coupling between clouds, radiation, and the large-scale circulations of the Earth, we simulate an idealized equatorial tropical Pacific.   In 
the western Pacific warm sea surface temperatures prevail beneath rising atmospheric motion and deep convective storms.   In contrast, 
the eastern tropical Pacific is characterized by cooler sea surface temperatures and low-level, non-precipitating clouds with a dry 
atmosphere above the clouds.  These regions are connected by a large-scale circulation.  We analyze this scenario in a high-resolution 
and coarse resolution simulation and infer how deep convective heating in the west Pacific can influence the atmospheric cooling 
thousands of kilometers away in the east Pacific.  This cooling is a key ingredient in the determination of the Earth's sensitivity to 
changes in greenhouse gas concentrations.

A high resolution (1km) cloud resolving model is used here as a benchmark to compare with the same model at lower 
resolution (25km, 100km) configured 
as a fully parameterized general circulation model.  Both models use approximately the same spatial domain.  The climate model used is 
derived from the most recent version of the atmospheric GCM (AM4.0) developed at the Geophysical Fluid Dynamics Laboratory.  By 
dramatically decreasing the grid-spacing we resolve much more of the dynamical motion and eliminate some of the statistical 
approximations of clouds, and thus more explicitly simulate some of their impacts.  

\end{abstract}

%\section{Key Questions}


\section{Introduction}

%Without question, clouds are more than passive conglomerates of tracers in a turbulent fluid.  If an advance in our understanding of the role clouds play in our climate is desired, it is essential that we 
%learn to model the two way interaction between clouds and the circulation.  The radiative influence of clouds is one of the primary mechanisms by which they influence the circulation (imp of lwcre off).  
%Our scientific motivation for this study is to illuminate/clarify the ways in which the large-scale circulation connects regions of deep convective activity with the prevalent tropical shallow 
%convection and the radiatively critical stratocumulus cloud decks.  While the meridional temperature gradient is a fundamental driver of the large-scale circulation, and the Coriolis force is 
%critical in determining the characteristics of tropical waves and the tropical extra-tropical connections, we propose that all of the physics of importance to our particular questions are present 
%in the equatorial manifestation of the Walker circulation (perhaps better in conclusion).  
%
%Current computing capabilities cannot come close to resolving all of the spatial scales which are important in the lifetime of clouds.  
%As a result, coarse resolution climate models must use statistical representations of the influences from unresolved clouds. 
%Ideally the statistics of the coarse model should represent the explicitly resolved clouds of the higher resolution model and in particular as resolution is increased and the statistical approximations eliminated, the solution of the coarse model should be informed by the solution of the high resolution model.  This approach will help to demonstrate that the influence of clouds on climate is not an intractable problem.
%
%The paradigm of the Walker circulation includes many distinct manifestations of moist convection, interactions between radiation and clouds, as well as dynamical motions which range from
%sub-kilometer scales to roughly 10,000 km.     In addition to the scientific interest of the Walker circulation as a link between deep convection, shallow-convection, and stratocumulus clouds, the tropical 
%Pacific provides an ideal physical context with which to explore the sensitivities and strengths of theoretical models, high-resolution models and general circulation models.  
%The tropical system in the Pacific Ocean including deep, shallow, and mid-level convection, as well as stratocumulus cloud decks and sea-surface temperature gradients 
%represents a fascinating and useful microcosm of the ways in which clouds can interact with dynamic circulations.
%This region of the Earth system contains clues to the difficult questions of how clouds couple to the circulation and how the nonlinear interactions within the system contribute 
%to the cloud feedbacks to perturbations and the overall sensitivity of the system to anthropogenic forcing.
%
%Several studies (Grabowski et al.,2000, Bretherton et al., 2006; Wofsy and Kuang, 2012, Kuang, 2012) have examined a 
%mock Walker circulation in the context of cloud resolving 
%simulations, or idealized theoretical models.   The comparison between CRMs and simple theory revealed several important 
%similarities (like what? See Bretherton et al., 2006) and a multiplicity of differing results (like what) for different CRM studies. 
%It thus remains unclear what GCM like models should be expected to produce in similar configurations.  
%The unique contribution from this study is to take a new CMIP6 atmospheric global climate model and to use it in the context of a
%simple mock-Walker circulation configuration while changing as little as possible from the parent model.  The model we 
%use (GFDL's AM4.0) is flexible enough to operate at grid-spacings ranging from 100km down to 1km in a doubly periodic 
%domain that is configured similar to the equatorial Pacific.  This allows us to use a single code base to compare a model 
%run at GCM-like resolution with a full suite of subgrid-scale paremeterizations with a cloud-resolving version of the same 
%model in which all convective activity is strictly explicit and many more of the important dynamic motions are resolved. 
%"Effects of cloud on atmospheric radiative cooling must be considered to get a realistic Walker circulation over specified SST. (just before equation 8 of Bretherton et al. 2006)."    

%The atmosphere of Earth constantly radiates energy to space.  This cooling of the atmosphere is mostly balanced by warming from the release of latent energy by convective clouds and precipitation.  The details of this balance between cooling and warming determine the role that clouds play in setting the Earth's climate sensitivity.  However, clouds contribute to both warming and cooling of the atmosphere.  Deep convective storms and high clouds warm the atmosphere by trapping outgoing radiation and absorbing shortwave radiation while clouds at lower levels cool the atmosphere by reflecting the incoming radiation from the sun.  Previous research has repeatedly shown that low-level clouds are the single largest factor in the current uncertainty of the Earth's climate response to a doubling of $\rm{CO}_2$ concentration (Bony and Dufresne, 2005, Sherwood et al. 2014).  However, it has also been shown that the turbulent mixing of the atmosphere by deep convective clouds provides a strong control within climate models on the climate sensitivity (Zhao, 2014).  The goal of this project is to elucidate/explore the connection between these two pieces of evidence.
%
%Previous studies such as Held et al., 1993; Tompkins et la. 1998, Bretherton et al., 2005; Muller and Held, 2012; Wing and Emanual, 2014 have shown that idealized models are quite 
%sensitive to the details of the configuration such as domain size and resolution.  This lack of convergence among high resolution models makes it difficult to know what to expect from 
%lower resolution models with convective parameterizations which are relevant to our understanding of GCMs.  Part of the purpose of this paper is to use one modeling framework to
%illustrate just what is obtained from a coarse GCM-like model run with idealized boundary conditions.  We then use this modeling framework to compare the results obtained from simulations
%with a resolution of 100, 25, 2, and 1km.
%
%The goal of designing idealized numerical experiments is to develop a tool that is easier to understand than the system the idealization 
%is based on.  The goal of adding complexity to a climate model is to achieve a more `realistic' simulation, as compared to the observed 
%world.  These simulations are somewhere in the middle of those extremes.  We have decided to keep many of the complex elements of 
%the global climate model  so that we can hopefully better understand the parent model (AM4.0) while learning something concrete about 
%the observable world.  By retaining such complexities as interactive radiation, convective parameterization, and the Tiedtke large-scale 
%cloud scheme we run the risk creating an idealized model which is no easier to understand than the original parent
%model which compares more readily to observations.  However, it is important to use idealizations such as this to test our current 
%understanding and to illuminate some of the idiosyncrasies of the climate models that are currently looked to for information about the 
%future of our climate.  The mock-Walker Circulation can be thought of as RCE plus an imposed gradient in surface temperature.  This is 
%a simple design that incorporates additional complexity which is critical for the connection to observed atmospheric phenomena.
%
%Our simulations are based on the equatorial tropical Pacific.   In the western Pacific warm sea surface temperatures exist beneath rising atmospheric motion and deep convective storms.   In contrast, the eastern tropical Pacific is characterized by cooler sea surface temperatures and low level, non-precipitation clouds with a very dry atmosphere above the clouds.  These regions are connected by a large-scale circulation of winds.  By comparing this scenario in a high-resolution and coarse resolution simulation we can infer how deep convective heating in the west Pacific can influence the atmospheric cooling thousands of kilometers away in the east Pacific.
%
%Inevitable limitations in computing power necessitate climate studies with Earth-system models to use relatively coarse grid-spacing.  
%%Coarse, statistical Earth-system models which simulate long periods of the climate are necessitated by the limitations of current computers.  
%This research uses a high-resolution cloud-resolving model on a limited spatial domain as a benchmark to compare with the same 
%model at lower resolution configured as a fully 
%parameterized general circulation model.   The modeling approach presented here, made unique by its clean connection between a global climate 
%model and a regional cloud resolving model is  made possible by the flexible modeling system at GFDL.   The goal of this paper is to use this 
%capability to better quantify and understand the connection between tropical convective heating and changes in low clouds.    
\textbf{Paragraphs of Introduction}
\begin{itemize}
  \item{Within the tropical Pacific are strong overturning circulations, deep convective towers, abundant shallow cumulus/congestus, and congestus clouds.  The Hadley circulation connects the tropical pacific with subtropics and has been extensively studies (review paper?).  The circulation first 
  noted by Sir Gilbert Walker, and described by \citeA{Bjerknes1969} connects the Indonesian region that is dominated by deep convection to the 
  eastern Pacific region which tends to be populated more by shallow cumulus and, in the subtropics, stratocumulus clouds.  This circulation
   has come to be 
  known as the Walker Circulation and has received much less attention than the Hadley circulation (Geisler, 1981; Bretherton et al., 2006; Wofsy and Kuang, 2012; Schwendike et al., 2014).  Previous studies have focused on observations (Walker, Bjerknes), theory (Geisler, 1981), 
  or modeling studies (Bretherton et al., 2006; Wofsy and Kuang, 2012).  The modeling studies have compared theoretical models with cloud
  resolving models.  One of the primary motivations of this paper is to compare cloud resolving model simulations of the Walker Circulation with 
  simulations from a GCM used on a doubly periodic domain.    What about Johnson et al.?}
The reality is that we 
  were also inspired by Schneider et al. 2017.  Is it possible to tie all that together in a sensible way?  
Johnson et al. discusses the trade inversion and the high statitic stability in the subtropics.  He contrasts 
the thermodynamic arguments of Kloesel and Albrecht (1989) 
  and Sun and Lindzen (1993) with the dynamic-thermodynamic argument of Schubert et al. 1995.  What say I?}
  \item{The Walker Circulation as way of imposing realistic large-scale circulations, as opposed to spontaneously generated large-scale
  circulations.  Previous studies of RCE have proven to be a 
  useful tool with which to explore the basic physics of climate and climate sensitivity, but they lack a large-scale circulation.  Many 
  studies have also shown the importance and prevalence of convective self aggregation (CSA).  A mock-walker circulation with an
  imposed SST gradient can be thought of as studying a partially constrained aggregating or organized system in a way that removes
  some of the ambiguity.  The mock-Walker circulation also allows for a simple way of studying in physical space, the connection
  between column relative humidity and circulation in a way that is directly tied to the physical state of the system rather than with
  a forced compositing of the data  
  \cite{Bretherton2005, Muller2012}.}
  \item{On the importance of reducing component, or sub-grid scale uncertainty through the use of both parameterized and explicit 
  convection.  This paragraph could also cite papers which have helped to validate RCE in general, and the channel RCE configuration
  in particular.  To what extent do RCE simulations of CSA agree between GCMs and CRMs?  Perhaps here I can hint at the utility of 
  this methodology for unifying our understanding.}
\end{itemize}

\textbf{Key Questions}
\begin{itemize}
  \item{How do clouds influence the Walker Circulation?}
  \item{To what extent are the deep convective clouds and the low-level clouds in the Walker Cell coupled?}
  \item{When simulating the Walker Circulation, how well does a GCM compare to a CRM?}  
\end{itemize}

We perform a series of sensitivity experiments that highlight the different ways in which these experiments can equilibrate.  To 
highlight both the robust features of the simulations and the sensitive nature of the distribution of the precipitation maximum 
we test the impact of convective parameterization, 
longwave radiation interactions with clouds, domain size, 
and the resolution, or grid-spacing.    Muller and Held, 2012 found that self-aggregation was sensitive to both domain size and resolution
because of the sensitivity of the low-level clouds to these parameters.  % the longwave radiative cooling.  

The results of our experiments are organized as follows.  First, the general structure and characteristics of the steady state solutions
will be described.   Then, the influence of parameterized convection and the longwave cloud radiative effect on the overturning 
circulation will be shown.  We will then briefly discuss how the circulation changes as a function of domain size.  The last part of the
results will describe and contrast the Walker circulation in a GCM-like configuration and a CRM-like configuration as the 
resolution changes.

\section{Experimental Details and Methods}

To develop the model configuration used for these experiments we started with the same code base as that of the 
recently developed atmospheric global climate model AM4.0 \cite{Zhao_etal18a, Zhao_etal18b}.
%(Zhao et al. a,b).  
This includes the GFDL finite-volume cubed-sphere dynamical core FV3 (Harris and Lin, 2013) 
which can solve either the hydrostatic primitive equations or the nonhydrostatic fully compressible Euler equations.  
The default AM4.0 physics we use includes interactive radiation, parameterized deep- and shallow-convection, 
a large-scale cloud scheme, and a boundary layer 
scheme as described in Zhao et al. a,b and the references therein.  The prognostic moisture variables are the specific 
humidity (q), liquid (ql) and frozen water (qi), and cloud fraction.  The top of the model domain is at 1 hPa, with 33 vertical 
levels and a sponge layer extending downward to 8 hPa.  The kilometer of atmosphere just above the surface is resolved by 
8 model levels.  The changes made to the default AM4.0 physics configuration are the following.  The cloud-aerosol 
interactions were turned off to focus on the interaction between clouds, radiation, and the circulation.  The gravity wave drag 
parameterization was turned off 
%(do_cg_drag set to .f.; Alexander-Dunkerton gravity wave drag) 
in order to reduce large oscillations which developed in the horizontal wind field near the top of the model domain.  
The convection, radiation, large-scale cloud, microphysics, and turbulence parameterizations all remain the same 
as in AM4.0.   Thus for the experiments with convective parameterization (grid-spacing of 25km and 100km), the 
physics are very similar to those of AM4.0.  This configuration can also be compared to the simulations described in 
\cite{Popp2017}, %Popp and Silvers, 2017, 
which used a developmental version of AM4 with almost the same changes to the physics
schemes.


\begin{table}
\begin{center}
\caption{Specifications of the simulations used most heavily in this study.  In the Convection
column, `prm'  indicates that convection is parameterized and `expl' indicates explicit convection.}
    \begin{tabular}{*{5}{c}}
    \hline
    \hline
    \\
 Name & Grid Spacing $(\mathrm{km})$ & Domain $ (\mathrm{km^2}) $& Length $(\mathrm{months}) $ & Convection     \\ \hline
  P100L &  100          &   800 $\times$ 16000    &  60              & prm                   \\ 
    \\
  P100 &  100                & 800 $\times$ 4000     & 60            & prm                     \\  
    \\
  P25L &  25             & 200 $\times$ 16000      & 60             & prm                     \\  
    \\
  P25  &  25             & 200 $\times$ 4000      & 60             & prm                   \\  
    \\
 E25  &   25          & 200 $\times$ 4000      & 60             & expl                \\  
    \\
 E2   &   2          & 100 $\times$ 4000      & 6             & expl                   \\ 
    \\
 E1   &   1          & 10 $\times$ 4000      & 6             & expl                 \\  \hline

    \end{tabular}\par
    %\bigskip 
    \label{tab:lambda}
\end{center}
\end{table}


All simulations use a nonhydrostatic dynamical core, with prescribed SSTs and a doubly periodic domain which is elongated in the zonal 
direction allowing for three dimensional simulations but with a reduced computational cost relative to the default global domain.  
The SST is prescribed as a Gaussian function which is 4K warmer in the center (301K/27.85C) of the domain then 
at the edges (297K/23.85C).  The GCM runs with 25km and 100km grid-spacing have been run for 5 years while the 
1km and 2km experiments were run for 6 months.  Experiments with parameterized convection will be labelled with a P prefix, followed
by a number indicating the grid-spacing in kms while the experiments with explicit convection and no parameterized convection will be 
labelled with an E prefix, followed by the appropriate number.  Thus P25 refers to an experiment with parameterized convection using a 
grid-spacing of 25km, while E25 refers to an experiment that is identical to P25 except that the convective parameterization is turned off.    

To illustrate the strong dependence of our results on domain size, as well as the fundamental role that the LW CRE plays in GCMs
we utilize fully parameterized experiments with grid-spacing of 25km (P25) and 100km (P100) on domains with a long dimension 
of 4000 km (`small') and
16000 km (`large').  To explore the mock-Walker circulation in the context of both a GCM and a CRM we utilize 
comparisons of the experiments with grid-spacings of 25km (E25), 2km (E2), and 1km (E1) all on a domain with the same long dimension 
of 4000 km.  The experiments with a grid-spacing of 25km serve as a link between the GCM-like configuration and the CRM-like 
configuration.  Domains with dimensions of 16000 km are too costly for the 1km and 2km experiments.

The E1 and E2 simulations are in many ways similar to the configuration of so-called cloud resolving models.  
In particular both the deep and shallow convective parameterizations are turned off, and the threshold of grid-cell mean relative humidity 
which triggers new clouds is changed from the default value of 0.8 to 1.0.  While a grid spacing of 1 or 2km is clearly not small enough
to resolve all clouds, it is small enough to resolve cloud-systems and medium to larger sized clouds.  The primary 
difference between the CRM simulations presented in this paper and those of more established CRMs such as SAM is that our model
has a much coarser vertical resolution and a prognostic large-scale cloud scheme inherited from the GCM.  The Tiedtke (1993) 
parameterization scheme for large-scale 
clouds was designed to be used with GCMs having a coarse grid-spacing.  However, there is not a fundamental problem that we are 
aware of in using the Tiedtke scheme for large-scale clouds in a model with 1km grid-spacing.   The advantage is retaining the 
identical cloud scheme as is used in the parent GCM; the disadvantage is the greatly increased complexity of the cloud computations 
relative to many other cloud resolving models.    

\begin{table}
\begin{center}
\caption{Domain mean precipitation ($\overline{\rm{P}}$), outgoing longwave radiation 
($\overline{\rm{OLR}}$), precipitable water ($\overline{\rm{PW}}$), and subsidence fraction (SF)  
or fraction of domain that is subsiding at the 532 hPa level.
Values in parenthesis correspond to experiments whith LWCRE is off.}
    \begin{tabular}{*{5}{c}}
    \hline
    \hline
    \\
 Name &   $\overline{\rm{P}} (\mathrm{mm/day})$ & $\overline{\rm{OLR}} (\rm{W \, m^-2})$ & $\overline{\rm{PW}} (\rm{mm})$ & SF   \\ \hline
  P100L   &  4.1 (3.5)   &  283.1 (286.9)  & 36.6 (31.3)  & 0.89 (0.61)      \\ 
    \\
  P100 &   3.9 (3.7)   &  283.2 (296.4)  & 28.0 (26.8) & 0.73 (0.71)           \\  
    \\
  P25L &   4.0 (3.8)  &  281.2 (290.7)   & 35.0 (32.9) & 0.78 (0.69)          \\  
    \\
  P25  &   3.8 (3.7)    & 282.9 (293.6)   & 27.4 (26.4) & 0.80 (0.72)          \\  
    \\
 E25  & 3.7 (3.5)    &   271.9 (286.8)   & 28.7 (27.3) & -            \\  
    \\
 E2   &  3.1 (3.4)   &  266.2 (285.5)    & 27.0 (25.2) & 0.83 (0.77)           \\ 
    \\
 E1   &  3.3 (3.7)   &  269.3 (289.2)    & 27.3 (26.5) & 0.78 (0.78)         \\  \hline

    \end{tabular}\par
    %\bigskip 
    \label{tab:lambda}
\end{center}
\end{table}


One tool that has been commonly used to infer the influence of clouds on a model simulation is to make the clouds invisible to the 
radiation.  This can tell us the extent to which the specific location of the clouds matters to the radiative interaction with the 
atmospheric state.   This method was originally pioneered by Slingo and Slingo, 1988 and Randall et al., 1989.  More recently, it has been
implemented as part of the CFMIP series of experiments (Stevens et al. 2012).
In the AM4 code, this is done separately for the longwave and shortwave radiation.  In this study we compare control experiments, 
in which clouds and radiation are fully interactive with experiments in which clouds are invisible to the longwave radiation.  These 
experiments are referred to as Longwave Cloud Radiative Effect Off (LWCRE off).  
For the LWCRE off experiments, both the longwave and shortwave radiation are present and interact with 
the atmospheric state, the clouds still interact with the shortwave radiation, and they still precipitate.   
Normally, turning off the longwave cloud radiative effect would have a large impact on the surface budget of a coupled model.  
However, because there is no land in our simulations and the SST is held fixed, the energetics of our experiments are not as
strongly effected as might be expected.  For this reason, experiments with only a water surface at the lower boundary and 
fixed SST are the ideal configuration to utilize the LWCRE off experiments as a way of diagnosing how clouds interact with 
the atmospheric state.  

\section{The organizing influence of LWCRE and the asymmetric results of parameterized convection.}

The mock-Walker Circulation that emerges from these simulations is shown in Figures \ref{fig:rh_psi_P25vsE25} and 
\ref{fig:precip_vertvel} to be characterized by a strong overturning circulation with precipitation focused over the warmer 
SSTs and a humid boundary layer across the full length of the domain.   For the GCM-like configuration (P25, left panels 
of Figure \ref{fig:rh_psi_P25vsE25}) with fully parameterized convection and with the longwave CRE turned on, there is 
a strong overturning circulation that results in a humid, condensate loaded troposphere near the middle of the domain 
and a dry ($< 10\% RH$) subsidence region with almost no condensate above $800 \rm{hPa}$.    
This atmospheric state has the same fundamental characteristics as the Walker circulation, the Hadley Cell, and 
experiments of radiative convective equilibrium which equilibrate to a state with deep-overturning circulations and 
convective aggregation.

To illustrate some of the sensitivities to convection and the interaction between clouds, radiation, and the large-scale 
circulation we compare the P25 experiment with analogous experiments in which the longwave CRE is turned 
off (P25 LWCRE off, middle 
panels of Figure \ref{fig:rh_psi_P25vsE25}) and in which the convection is made explicit by turning off the convective 
parameterization (E25, right panels of Figure \ref{fig:rh_psi_P25vsE25}).  
The mass streamfunction and vertical velocity both show the P25 experiment has a stronger overturning circulation 
than either E25 or P25 LWCRE off.  P25 also has higher humidity in the deep convective region, and lower humidity in 
the subsidence region over the 
cooler SST (Figure \ref{fig:rh_psi_P25vsE25}) than either E25 or P25 LWCRE off.  Averaged over the full domain, 
the parameterized convection leads to a dryer atmosphere with less 
condensate (both liquid and ice) and with weaker radiative cooling and condensational heating (below 800 hPa).  
However, in the updraft region the convective parameterization results in more moisture aloft.      

Superposing the circulation and relative humidity illustrates the subsidence driven 
drying and the strong moistening that results from ascent over the region of maximum latent heat flux.  
Figure \ref{fig:rh_psi_P25vsE25} shows two distinct circulation cells with one below, and one above $500 \rm{hPa}$.  
Deep-convective activity dominates the 
region over the warm pool, shallow convection is common over a wider range of SSTs, and for the most part stratocumulus 
clouds are absent (based on what?).  When the coupling between the circulation and clouds is broken, by turning off the longwave CRE, 
the atmospheric state is much more symmetric about the maximum SST.  We also find large differences among the 
experiments in the domain mean precipitation, condensate/clouds, and circulation (see Table 2).  In general the 
experiments with parameterized convection are much more erratic.   

One of the most prominent features of our GCM-like (particularly P25) simulations is an asymmetry relative to the 
symmetric SST distribution in both the time-dependent and 
steady-state solutions (Figures \ref{fig:rh_psi_P25vsE25},  \ref{fig:precip_vertvel}, \ref{fig:domdep} and  \ref{fig:conv_vs_ls}).  
This asymmetry is also present in P25 experiments on larger domains and at a resolution of $100 \, \rm{km}$. 
The steady-state precipitation maximum is located  not over the warmest SST but is
shifted to slightly cooler temperatures.  The asymmetry is apparent in the vertical velocity, mass circulation, relative humidity, 
specific humidity, and radiative heating.  
In the Hovmoller diagrams (Figures \ref{fig:domdep} and \ref{fig:domdep_lwoff}), the precipitation appears to be averse to 
residing over the SST maximum.  
For the P25 case shown in Figure \ref{fig:rh_psi_P25vsE25} a strong ($1 \, \rm{m \, s^-1}$) domain
mean shear develops above about 500 \textit{m} which shifts the precipitation and circulation off center for years at a time.
When the convective parameterization is turned off (E25), the overturning circulation becomes weaker and 
broader (w, and psi), and the precipitation, cloud fields, and 
circulation all consistently reside over the SST maximum (for about 1 year before going haywire).   
While the parameterized convection plays a large role in driving this asymmetry it appears to be not entirely a 
result of the convective parameterization, but also due to an interaction between the convective 
parameterization and the LW CRE.   The degree to which this asymmetry influences the comparison with other experiments is 
unclear.    Note: check the wind shear for P25large, P100small, and P100large (E25 has less shear than E1 or E2). 

Complex patterns of precipitation over a fixed sinusoidal or Gaussian SST distribution have been noted many times in 
previous literature \cite{Bretherton2006, Wofsy2012} (Ming Zhao (personal communication, see Jeevanjee et al. 2018))
but the irregularities have 
tended to be symmetric about the SST maximum.  This is broadly consistent with our simulations when the convection is 
entirely explicit (E25, E2, and E1, discussed further in section 4).      


%%%%%
% section 3.1 results
\begin{figure}
  \centering
      \includegraphics[width=0.96\columnwidth]{walkerfigs/rh_psi_cond_25km_6pan.eps}
      \caption{Comparison of the steady state relative humidity (shading of top panels) and mass 
      streamfunction (contours) for experiments
      with a grid-spacing of 25km on a domain of 200 $\times$ 4000 $\, \rm{km}^2$.  The lower panels 
      show the total (liquid + ice) condensate (g/kg).  
      An experiment with fully parameterized convection (P25) is on the left, the same 
      experiment but with the LWCRE off is in the center panel, and the experiment with LWCRE on, but the convective
      parameterization turned off is on the right (E25).  All three panels use the same contour interval for the mass 
      streamfunction (kg/s).}
  \label{fig:rh_psi_P25vsE25}
\end{figure}

"... the higher mean winds on the flanks of the SST maximum raise the latent heat fluxes and associated rainfall there, resulting in a 
double-peak rainfall distribution with a slight local minimum in rainfall at the SST maximum (just before eq. 8)" 
was noted by Bretherton et al., 2006.  On the page with equation 20, there is an excellent discussion of the dependence of the 
vertical structure of cumulus on two key factors.  First, temperature and moisture in the BL air where the convective updrafts originate.
Second, the moisture profile in the lower troposphere above the boundary layer.  This 'dictates the extent to which air
turbulently mixed into rising cumulus updrafts dries them out.'  I think most of our simulations with a grid-spacing of 25km or 
greater are being dominated by this second factor.    

The default experimental configuration includes radiation that is fully interactive with the clouds, water vapor, and 
temperature of the atmospheric state.  It can be seen in the left panels of Figure \ref{fig:rh_psi_P25vsE25} that this 
results in a stronger circulation with more condensate in the deep convective region and a dryer mid-troposphere
relative to the case with the LW CRE off.   A stronger and spatially concentrated circulation for cases when the 
clouds interact with the longwave radiation can also be seen in Figures 4, 5, 9 and 10 and is particular apparent in 
the cases with a large domain (16000 km long dimension).  This will be discussed further in the next section.  

Previous studies have shown that cloud radiative effects act to strengthen and contract an overturning circulation 
(e.g. Popp and Silvers, 2017, HH, and others (Muller and Held?), Albern et al., 2019).  This is 
due to an increased low-level flux
of moist static energy into the convective regions.  While our mock-Walker circulation is distinct from radiative
convective equilibrium and the resulting convective self-aggregation, there are obvious similarities between our 
region of persistent deep convection and a state of aggregation.   One of the simplest measures of convective 
aggregation and the large-scale circulation is the fraction of the domain in which the air is subsiding, the 
subsidence fraction (SF).  As convection becomes more organized, or aggregated, SF will increase.  We expect 
for an overturning circulation that a contraction of the convective region would result in a larger subsidence 
fraction.  This is precisely what we see in Table 2.  For each of our experiments, the cases with LWCRE on
have a larger SF (with the exception of E1, for which SF is constant).  Our prescribed SST warm patch ensures that our simulations will 
be `aggregated' to some degree.  However, the subsidence fraction is a useful metric even in this case which 
includes an additional constraint than RCE.  Given identical SSTs, the different SFs give a measure of variability
that is driven entirely by the atmosphere.  

%\subsection{The Asymmetric Influence of Convective Parameterization and the organizing tendency of LWCRE}

% LWCRE has a large impact on LS precip, not so much on convection
% LWCRE has a large impact on the low-level circulation (circulation is stronger when LWCRE is on)
% Are the low-level clouds a connecting point?  

% are the impacts of the LWCRE the same for GCMs as they are for CRMs?  

% Vertical Velocity and time averaged precipitation
\begin{figure}
  \centering
      \includegraphics[width=0.6\columnwidth]{walkerfigs/PrecipVertVelocity_LWCRE_onoff.eps}
  \caption{Precipitation (top) and vertical velocity (bottom) at approximately 530 hPa for P100, P25, E2, and E1 experiments.  
  The data has been averaged over the short horizontal dimension of the channel and over the 
  equilibrated part of the experiments.  Control configurations with normal model physics are on the 
  left.  On the right are the corresponding experiments with the longwave cloud radiative effect turned
  off (LWCRE off).}
  \label{fig:precip_vertvel}
\end{figure}
%
% below shows that the circulation extremes decrease when the lw cre is switched off.  the case with a 
% grid-spacing of 25 km shows the largest difference 
%
%% max/min values for the stream function with lwcre off:
%(0)	max/min of psiplot ensind  is: 0.3477345977293868 and: -0.2115510906212598   diff = 0.56
%(0)	max/min of psiplot1 is: 0.3195050101301776 and: -0.2795603759630235             diff = 0.60
%(0)	max/min of psiplot2 is: 0.3551683771471268 and: -0.3245916357825396             diff = 0.68
%
%% max/min values for the stream function with lwcre on:
%(0)	max/min of psiplot ensind  is: 0.1346705118433919 and: -0.6774837967222883    diff = 0.81
%(0)	max/min of psiplot1 is: 0.301707822836468 and: -0.3661587052771586                diff = 0.67
%(0)	max/min of psiplot2 is: 0.4634904466711686 and: -0.3170573366946793               diff = 0.78

\begin{figure}
  \centering
      \includegraphics[width=0.6\columnwidth]{walkerfigs/Precip_DMN_4models_1yr.eps}
  \caption{Domain mean precipitation as a function of time.  All data has been smoothed with a running mean filter.  
  Thick lines are the control experiments, thin lines have no LW CRE.  All data is shown for the E1 and E2 cases
  while the P25 and P100 cases show only the first out of five years of the time series.}
    \label{fig:precip_dom_mn}
\end{figure}

Despite the same boundary conditions and model base, the experiments documented here have a
large range of domain mean precipitation ($\rm{\overline{P}}$, Table 2) that varies by as much as 
0.6 $\rm{mm/d}$ (3.5-4.1 in parameterized experiments; 3.1-3.7 in explicit experiments).
This highlights how dominant the interaction between clouds and radiation can be in determining the 
characteristics of a system.
Because of the tight constraints that connect the domain mean precipitation, atmospheric condensational heating, and 
the total radiative cooling, the time evolution of the precipitation serves as a useful measure of whether a model has 
reached statistical equilibrium.  Figure \ref{fig:precip_dom_mn} demonstrates that this equilibrium is reached after about 30 
days for the E2, and E1 simulations, and after about 50 days for the P25 and P100 simulations.   After the initial adjustments
the simulations all oscillate about mean precipitation values which tend to increase with the grid-spacing 
(Table 2).   The fact that E1 and E2 reach equilibrium sooner than E25, P25, or P100 and that the period of oscillation 
about the domain mean precipitation is smaller helps to justify the 6 month simulation times for E1 and E2.  
%The period of oscillation is much larger for the GCM like simulations than it is for the CRM like oscillations.
The large oscillations in domain mean precipitation are similar to those noted in previous studies 
\cite{Silvers2016, Patrizio2019}
%(Silvers et al., 2016; Patrizio and Randall, 2019).   
%The experiments with a higher resolution have a lower value of domain
%mean precipitation, with the E2 simulation having the lowest (3.1 mm/day).  
Differences in $\rm{\overline{P}}$ can be understood as a consequence of the differences in upper level cloud fraction 
and the surface energy budget and will be discussed further in a later section.   



\section{The Influence of Domain Size on the LWCRE and the LS Parameterized Precipitation: Change Title?}

When designing numerical experiments the size of the domain and the grid-spacing that determines model resolution are 
two of the most fundamental choices that must be made.   One the goals of this study was to explore the results of 
changing grid-spacing over a wide range of values.  To simplify the analysis we have in most cases chosen to keep the 
long horizontal dimension fixed at $4000\, \rm{km}$.  However, previous studies 
\cite{Bretherton2005, Bretherton2006, Muller2012, Jeevanjee2013, Silvers2016, Patrizio2019}
(e.g. Bretherton et al., 2005; Bretherton et al., 2006; Muller and Held, 2012; Jeevanjee and Romps, 2013; 
Silvers et al., 2016; Patrizio and Randall, 2019) have 
documented sensitivities of the equilibrated state to domain size.   
The analysis of the previous, and of the next sections focus on results from 
experiments using a domain with a long domain length of $4000 \, \rm{km}$.  However we do find interesting 
sensitivities to the domain size that are noted here. 

The evolution in time of the precipitation field reveals important characteristics of these simulations, and 
illustrates how much the distribution can vary as a function of domain size, parameterization of convection, 
and the effect of the longwave radiation due to clouds. 
Shown in Figures {\ref{fig:domdep}} and {\ref{fig:domdep_lwoff}} are Hovmoller plots of precipitation after 
averaging along the short horizontal dimension.  The four panels show simulations with a grid-
spacing of $25\, \rm{km}$ on a domain with a horizontal width of $4000\, \rm{km}$ (far right) and $16000\, \rm{km}$ 
(middle right), and 
simulations with a grid-spacing of $100\, \rm{km}$ on a grid with a horizontal width of $4000\, \rm{km}$ (middle left) and 
$16000\, \rm{km}$ (far left).  Figure  {\ref{fig:domdep_lwoff}} shows the equivalent simulations with LW CRE off.    
 At all resolutions the Hovmoller plots show that the longwave CRE acts to concentrate the 
precipitation over a smaller geographic extent.   This is consistent with previous work showing that CREs 
strengthen an overturning circulation and
narrow the region of deep convection (see Popp and Silvers, 2017, HH, Dixit et al. 2019.  etc.).
The structure of the precipitation changes more as a function of domain size than it does as a function of resolution. 
On the large domains, the difference between experiments with and without LW CRE is extreme.  In contrast to the control 
experiments in Figure {\ref{fig:domdep}} which all show a narrow region of strong precipitation which meanders within 
500 km of the SST maximum at the center of the domain,  
the experiments without the LW CRE basically have no region with consistently strong precipitation.  Instead, 
smaller cells or lines of precipitation develop within an area that is roughly 8000 km wide.  As previously notes, these
GCM-like simulations equilibrate after approximately 50 days.  However, there is also a dramatic change in 
the distribution of precipitation on the $4000\, \rm{km}$ domain simulations after almost 2 years.   The domain mean 
precipitation does not significantly change in these cases, only the spatial structure.  

% new Hovmoller diagram
\begin{figure}
  \includegraphics[angle=90, width=0.95\columnwidth]{walkerfigs/hov_precip_4pan_vert_lwcreon.eps}
  \caption{Evolution of precipitation through 3 years of simulation for a grid spacing of 25km (2 panels on right) and
  100km (two panels on left).  For each resolution, domains with a width of 4,000 km and 16,000 km are shown.}
  \label{fig:domdep}
\end{figure}

\begin{figure}
  \includegraphics[angle=90, width=0.95\columnwidth]{walkerfigs/hov_precip_4pan_vert_lwcreoff.eps}
  \caption{Identical to previous figure, except that LW CRE is now off.}
  \label{fig:domdep_lwoff}
\end{figure}

\begin{figure}
  \centering
       \includegraphics[width=0.45\columnwidth]{walkerfigs/Cond_Walker_P100vsP25_lgvsm_lwoff.eps}
          \caption{Domain mean total condensate (liquid + ice) for P100 (red) and P25 (yellow) on the domain 
          with a long dimension of 16,000km (thick) and 4,000 (thin).  Solid lines show experiments with LWCRE on
          and dashed lines experiments with LWCRE off.}
  \label{fig:TotCond}
\end{figure}


Smaller domains were found to have a more focused ascent region and larger precipitation rates in Bretherton et al., 2006.
They also found less low-level clouds over the colder SSTs in the small domains, relative to a control domain size that was 4x
larger.   In contrast, here larger domains have larger precipitation rates (Table 2).  For the LWCRE off cases, in agreement 
with \citeA{Bretherton2006} we find a more 
focused ascent region in the smaller domain.   But when the LWCRE is on, the extent of the ascent region changes little between P25 
and P25L but for P100 and P100L the ascent region is much more focuses on the large domain.    
How do our low-level clouds over the colder SSTs compare with the findings of B06 (check the condensate for small vs. large 
domains)?   

% Fraction of Precip due to large-scale precip for large/small experiments with a 25km resolution
\begin{figure}
  \centering
      \includegraphics[width=0.8\columnwidth]{walkerfigs/Conv_vs_LS_hist.eps}
  \caption{Precipitation that is due to the large-scale(blue) cloud scheme and to the convective 
  parameterization (black).  Experiments with the LW CRE off are shown with thin lines.  
  Panels on the left show large domains with a width of 16000 km and
  panels on the right show domains with a width of 4000 km.}
  \label{fig:conv_vs_ls}
\end{figure}

The domain mean total precipitation is constrained by the radiative cooling of the atmosphere.  However, in 
models with the convection parameterized the 
total precipitation is composed of precipitation from the convection scheme and the large-scale cloud scheme. 
While the relative contribution of each component is not well constrained, it is thought to influence the 
atmospheric state (Held et al., what part of the state?). 
The distribution of convective and large-scale precipitation indicates how the condensational heating
in a GCM is being distributed among the parameterizations, and what is triggering the precipitation.  
Precipitation from each of these two components is
shown in Figure \ref{fig:conv_vs_ls} as a function of both resolution and domain size.  
In the regions of large-scale ascent, most of the 
precipitation derives from the large-scale cloud scheme.  We also see that the LW CRE (thick lines) dramatically increases
the large-scale precipitation.  The LW CRE has a much smaller effect on the magnitude of the convective precipitation 
but does act to spatially concentrate it.  With the exception 
of the P100L LW CRE off experiment, the convective precipitation produces relatively little of the total precipitation for 
both resolutions and on both domains.  The dramatic dependence on domain size of the precipitation field that is 
seen in Figure \ref{fig:domdep_lwoff} corresponds to a decrease in the large-scale precipitation of about 65\% in P25L case
and an almost complete elimination of the large-scale precipitation in the P100L case.    

%We hypothesize than an important difference between our simulations and previous studies which use GCMs
%is that although the deep and shallow convection is parameterized
%with the GFDL double-plume parameterization, the fraction of precipitation that is produced by the convective parameterization 
%is much smaller in our simulations than it is for many GCM-type experiments with parameterized convection.   At 25km grid-spacing, 
%the majority of condensational heating occurs through the large-scale cloud  scheme (Figure \ref{fig:conv_vs_ls}).
%%Overall, the fraction of precipitation that is 
%%due to the large-scale scheme is roughly 40\%, and in many places it is actually greater than the amount of precipitation 
%%produced by the convection scheme.   
%%In comparison, many GCMs, have between 70\%-90\% of their precipitation produced by 
%%the convective parameterization.  
%There are further surprises in the fraction of precipitation that is large-scale.  On our small 
%domains, with a bowling alley domain length of 4000 km, the simulations with 100km grid-spacing have values of $f_{ls}$ 
%of $75-80\%$.  Often, at GCM-like resolutions such as 100km the convective precipitation would be expected to control much of the
%precipitation rather than the large-scale parameterization.  However, when the domain length is increased to 16000 km
%$f_{ls}$ decreases to a range of $0-70\%$ that depends strongly on location.  Over the warmest region of the surface the 
%precipitation is still dominated by the large-scale scheme.   The large domain without the longwave CRE is the 
%only configuration in which the convective precipitation is dominant.    If the default, or baseline, configuration of a GCM is built 
%with the expectation that the convective parameterization scheme will dominate the production of tropical precipitation, then 
%perhaps a configuration that minimizes the convective precipitation will result in a balance between the parameterization 
%schemes that is less intuitive, such as an asymmetric response to symmetric forcing.  


%\section{The mock-Walker Circulation Dependence on Grid-Spacing: A transition from a 
%General Circulation Model to a Cloud Resolving Model}
\section{Dependence on Resolution: From a General Circulation Model to a Cloud Resolving Model}

\begin{figure}
  \includegraphics[width=0.96\columnwidth]{walkerfigs/precip_4mods_6mn.eps}
  \caption{Evolution of precipitation through the first 6 months of simulation for (left to right) the 25km control case (P25), 
  25km case with no parameterized convection (E25), 2km control (E2), and 1km control (E1).  
  Shading indicates mm/day.  The horizontal axis shows the entire 
  longitudinal domain of 4000 km with the center of the domain having a prescribed SST of 301 K and the edges 297 K.  
  Shown is the average over the small dimension of the channel domain. } 
  \label{fig:hov_4mods_6mn}
\end{figure}

% this paragraph should perhaps be moved to the introduction...
It is clear that the overturning circulation in the tropical Pacific encompasses dynamical motions at a host of scales and that the 
different cloud types interact with the circulation on scales ranging from meters to thousands of km's.  
The tropical Pacific also serves as an ideal location to study interactions between clouds and the 
circulation because it includes strong overturning circulations, abundant shallow cumulus, congestus, and cumulonimbus 
clouds \cite{Johnson1999}, as well as stratocumulus cloud decks along the eastern extremities of the basin.
Can we hope to simulate all of this in a single idealized modeling context?    
Our initial motivation for using the GFDL AM4 model on a doubly-periodic domain was to simulate 
a tropical Pacific-like region 
%that is rich in phenomena 
in the context of both a GCM and a CRM in the hopes that resolving more of the turbulent motions and circulations
would help us to better understand the physics and the mechanisms which are at work in the cloud-circulation
interactions of the tropical Pacific and improve our ability to model this region in a GCM.    
The dynamical core of the current generation GFDL models is ideally suited to this task because it can easily 
be used over a wide range of resolutions and can solve either the hydrostatic or nonhydrostatic 
governing equations.  

\begin{figure}
  \centering
      \includegraphics[width=0.96\columnwidth]{walkerfigs/rh_psi_cond_E25E2E1.eps}
      \caption{The equilibrated state of the Walker cell as a function of resolution.  
      Experiments shown are E25 (left), E2(center), and E1(right). Top panels show the 
      steady state relative humidity (shading) and mass streamfunction (black contours)
      while bottom panels show the total condensation (liquid + ice).  
      All panels use the same contour interval for the 
      mass streamfunction (kg/s).}
  \label{fig:rh_psi_P25E2E1}
\end{figure}

\begin{figure}
  \centering
      \includegraphics[width=0.96\columnwidth]{walkerfigs/rh_psi_cond_E25E2E1_lwoff.eps}
          \caption{Same as previous figure except with the longwave CRE turned off.}
  \label{fig:rh_psi_P25E2E1_lwoff}
\end{figure}


In this section we focus on simulations with grid-spacing of 1km, 2km, and 25km.   
A glance at Figure \ref{fig:hov_4mods_6mn} shows notable differences in the precipitation structure that result from the 
overturning circulation at different resolutions.  Shown are 180 days of precipitation from the P25, E25, E2, and 
E1 simulations.  As the 
resolution increases towards $1 \, \rm{km}$ the distribution of precipitation becomes more regular and consistently 
centered over the SST maxima in the middle of the domain.  Both the P25 and E25 simulations show more 
variability at later times compared to these first 180 days.  The simulations with explicit convection at resolutions 
typical of cloud-resolving models (E2, E1) show little aversion to the precipitation maximum occurring over 
the maximum in SST and relative to the P100, P25 and E25 simulations 
they are able to maintain a smoother distribution of  precipitation over a broader range of SST values.    


The steady state relative humidity and overturning circulation (Figures \ref{fig:rh_psi_P25E2E1} and 
\ref{fig:rh_psi_P25E2E1_lwoff}) show an interesting combination of similarities and differences
among the experiments with varying resolutions.  All experiments have a multicell vertical structure (consistent with 
Bretherton et al. 2006), in contrast to the single cell that results from simple theoretical models such as the SQTCM.   All 
show a similarly dry mid-tropospheric region where subsidence dominates.  However, the multicelled structure is quite irregular
in the P25 experiment, and a small third cell has developed in the boundary layer of the 1km experiment.  The mid-troposphere 
of the P25 experiment is more than $40\%$ dryer in the region with deep-convection than for the 2km and 1km experiments. 
The high resolution experiments also have higher amounts of condensate throughout the troposphere, and much higher
relative humidity above about 300 hPa. 

Despite a fairly regular distribution of precipitation around the SST maximum for experiments with increasing resolution, the 
surface enthalpy flux reveals large differences in the symmetry.   Figure \ref{fig:enthalpy} shows the surface enthalpy flux, 
the equivalent potential temperature, and the u-component wind field for E1,E2, and E25 both with (thick lines) and 
without (thin lines) the LW CRE.  
Over the SST maximum, E25 has a surface enthalpy flux that is 
60 W/m2 larger than that of the E1 experiment, and the E1 experiment has a very irregular pattern of enthalpy flux in the middle
half of the domain.  It appears that these differences in magnitude and regularity are due to the differing wind speeds among 
the three experiments.  The asymmetry in the enthalpy flux is also apparent in the lowest model level of the atmosphere
and the E1 experiment also has the largest domain mean wind shear of the three models.  This is consistent with the 
strong asymmetries of the P25 experiment being connected to a large domain mean wind shear in that experiment.  
It is also worth noting that the E1 experiment has a stronger domain mean u wind shear compared 
with the E2 and E25 experiments.  Figure \ref{fig:enthalpy} also shows that over the warm patch of SST, the surface 
enthalpy flux is decreased by the LWCRE for E25, E2, and E1 despite these experiments having stronger low-level winds.  
This decreased enthalpy flux also corresponds to an increased amount of specific humidity in the lowest atmospheric 
model level.        
%%%%%
% section 3 of results

% Hovmoller diagram of precipitation for the first 6 months of the 25km, 2km, and 1km experiments
% LW CRE on


Figure \ref{fig:cf_tdtlw} shows E2 to have the largest (about 17\%) upper level mean cloud
fraction in the subsidence region, with the E1 experiment having the next largest cloud fraction (10\%), followed by 
P100, E25, and P25 (3-5\%).   Less upper level clouds will allow more radiative cooling of the atmosphere and 
require a correspondingly larger amount of condensational heating and precipitation.    


\begin{figure}
  \centering
      %\includegraphics[width=0.3\columnwidth]{walkerfigs/MnTotCondProf.eps}
       \includegraphics[width=0.45\columnwidth]{walkerfigs/Cond_Walker_E25E2E1_lwoff.eps}
          \caption{Domain mean total condensate (liquid + ice) for P25 (thin yellow), E25 (yellow), 
          E2 (blue), and E1 (green).  Solid lines show experiments with LWCRE on
          and dashed lines experiments with LWCRE off.  All lines correspond to experiments with explicit 
          convection.}
  \label{fig:TotCond}
\end{figure}


The influence of resolution on the atmospheric state is clearly illustrated with the LW CRE off experiments shown 
in Figure \ref{fig:rh_psi_P25E2E1_lwoff}.
The two-dimensional structure of circulation and humidity shows that the amount of condensate aloft in the 
deep-convective region and the range of relative
humidity values between the ascent and descent regions increase with resolution.  It is also apparent in Figure
\ref{fig:rh_psi_P25E2E1_lwoff} that the condensate below 800 hPa decreases with increasing resolution.  
This is consistent with an overturning circulation that 
strengthens as the resolution increases.  Figure \ref{fig:rh_psi_P25E2E1_lwoff} also shows greater asymmetries 
and generally weaker circulations below about 500 hPa.  
When the clouds and radiation are allowed to directly interact with each other the E2 and E1 simulations 
appear much more similar (Figure \ref{fig:rh_psi_P25E2E1}).  The simulations with LW CRE also have a better
organized and stronger circulation below 500 hPa.     

The domain mean condensate is an important measure of the steady state reached by these experiments.  The vertical 
distribution of condensate is closely related to the distribution of clouds.  This in turn influences the distribution of 
energy throughout the atmosphere and the longwave radiation that is emitted to space.  It can also indicate the 
strength of convection 
and vertical mass transport.     Figure \ref{fig:TotCond} shows the domain mean condensate for P25, E25, E2, and E1
(solid lines) and the corresponding experiments with the LWCRE off (dashed lines).  There is a significant difference 
between the GCM-like P25/E25 experiments and the CRM-like E2/E1 experiments with the former having much 
higher values of low-level liquid condensate/cloud and the later having much higher values of upper level ice 
condensate.  The LWCRE off experiments influence the low-levels of the GCM-like experiments, and the upper-level
regions of the CRM-like experiments.     It is also interesting to note that the LWCRE results in less condensate 
above 700 hPa for all experiments, and more condensate below 700 hPa for the P25/E25 experiments.  

The E25 and P25 experiments have much less condensate aloft and a lower relative humidity in the mid-troposphere
where the differences in strength of the overturning circulation are most pronounced. 
(can we use this to understand the differences in LWCRE?).  

%\begin{figure}
%  \centering
%       %\includegraphics[width=0.96\columnwidth]{walkerfigs/hist_6pan_25vs2vs1_lwoff_lwon.eps}
%       \includegraphics[width=0.96\columnwidth]{walkerfigs/hist_3pan_25vs2vs1_lwoff_lwon_draft.eps}
%  \caption{Normalized probability distributions of the location of the maximum value of precipitation for experiments with the 
%  longwave CRE on.  All panels show cases with parameterized convection, the top panels are computed for the experiments 
%  having a domain width of 4000 km while the bottom panels are for experiments having a domain width of 16000 km.  The 
%  black curves show a Gaussian function with a standard deviation computed for each experiment and an expected location 
%  in the center of the domain where the SST is warmest. }
%\end{figure}

\begin{figure}
  \centering
      \includegraphics[width=0.96\columnwidth]{walkerfigs/CloudFractdtRad_lwcre.eps}
  \caption{Cloud fraction and temperature tendency due to longwave radiation.  Profiles were computed in the 
  subsidence regions and are shown for the control (LWCRE on) and LWCRE off experiments.}
  \label{fig:cf_tdtlw}
\end{figure}


An interesting fact that emerges from the domain mean values of precipitation (see Table 2) 
is that the sign of the response to LWCRE is not the same across the different grid-spacings.
When clouds are not allowed to interact with the longwave radiation the atmosphere can cool 
much more efficiently, as evidenced by larger values of OLR for all experiments when the LWCRE is
off.  Typically, the atmospheric cooling can be thought of as a proxy for the mean precipitation 
because the cooling is primarily offset by condensational heating.  This is clearly not the case 
for the E25, P25, and P100 cases for which the domain mean precipitation rates decrease despite
an increased amount of atmospheric cooling.  The implication is that the requisite atmospheric 
heating is coming from a process other than condensation.  The solution lies in the energy 
budget of the surface.  A fixed SST means that in all cases the upward flux of longwave energy
is constant, the upward flux of sensible heat flux will be mostly fixed (barring variations in surface
wind), and changes in downward shortwave flux will not be able to warm the surface.  However, 
the dramatic decrease of low-level clouds for the E25, P25, and P100 experiments when LWCRE is
off strongly influences the net flux of longwave energy at the surface (does the downwelling longwave
flux change becuase there are less clouds or because they are invisible to radiation?).  Under 
natural conditions with an interactive surface, the surface is warmed by a downward flux of longwave 
radiation due to the clouds
above.  When this source of atmospheric energy loss is removed in the LWCRE off experiments 
the upwelling longwave flux of radiation plays a larger role in warming the atmosphere that more 
than compensates for the increased OLR at the top of the atmosphere.  Thus, with an increase in 
atmospheric warming on 
the order of $20 \, \rm{W \,m^{-2}}$ when then LWCRE is off, there is less need for condensational 
heating.  The mean precipitation rate actually decreases.  The large decrease of low-level clouds also 
leads to a large increase of downward shortwave radiation.  This only slightly increases the
fluxes of reflected shortwave radiation (about $2 \, \rm{W \,m^{-2}}$) but does contribute to 
heating the atmosphere.

Turning the LW CRE off for the E1 and E2 experiments results in more precipitation while, 
for the E25, P25, and P100 experiments, less precipitation results from turning the LW CRE off.  
The primary method by which the longwave cloud radiative effect influences the atmosphere is by 
heating the atmosphere in the region between the clouds and the surface.  
Larger values of ice condensate and upper-level cloud fraction such as we see for the E2 and E1 experiment 
therefore imply a larger LW CRE relative to the E25, P25, and P100 experiments in which there are fewer clouds 
aloft that can emit downwelling longwave radiation.  When the warming effect of the upper level clouds 
in the E1 and E2 experiments is removed in the LWCRE off experiments the energy balance of the atmosphere 
must be maintained through an increase of latent heating and subsequent increase of precipitation.  This is 
reflected in Table 1.  The decrease of precipitation for LWCRE off experiments for the P25 and P100 experiments
is harder to explain because there is much less upper level cloud to influence the mid-tropospheric heating.  
However, the low-level clouds of the P25 and P100 experiments are dramatically reduced when the LWCRE is off.  

For GCM-like experiments with parameterized convection, the total precipitation is composed of one part from the 
convective parameterization, and one part from the large-scale cloud scheme.  Figure \ref{fig:conv_vs_ls} shows 
that when the LWCRE is turned off for the P25 and P100 experiments, the decrease in precipitation comes from 
a reduction in the large-scale cloud scheme precipitation and not from a decrease in the convective precipitation.       
This implies (is this connected to previous sentence?) there will be less atmospheric heating from 
longwave radiation (but doesn't LW just cool the atmosphere?) and a corresponding 
larger heating due to condensational heating which is consistent with higher values 
of domain mean precipitation for E25, P25, and P100 compared to E2 and E1.  Similarly, the E2 experiment has lower values 
of upper level condensate and higher values of mid-troposphere relative humidity compared to the E1 simulation, which ....??  
Water vapor is an efficient emitter of long-wave radiation and thus greatly facilitates the cooling of the troposphere.  
Popp and Silvers, 2017 showed that that there is dramatically less water vapor in the atmosphere and much less precipitation 
for LW CRE off experiments (see their figure 1b).  Our results here for the E25, P25, and P100 are consistent with 
those of Popp and Silvers, 2017 because the low-level condensate strongly decreases when the LWCRE is off.  This is not 
the case for E2 and E1.  Those experiments have very little low-level condensate to begin with and thus do not experience 
large changes in low-level condensate when the LWCRE is off.  They do however have a large increase of upper
level condensate for the LWCRE off experiments.   This is a large difference in the way that models with explicit convection 
respond to the LWCRE compared to more traditional large-scale models with parameterized convection.   


\begin{figure}
  \centering
    \includegraphics[width=0.96\columnwidth]{walkerfigs/LowLevelVars_E25E2E1.eps}
    \caption{Low-level structure and domain mean wind shear for simulations with explicit convection.  
    For all panels, thick lines show control experiments with the LW CRE on, thin lines show experiments
    with the LW CRE off.}
    \label{fig:enthalpy}
\end{figure}
  
%  \caption{Variables important to boundary layer energy fluxes.  Top left panel shows the surface enthalpy flux ($\rm{W m^{-2}}$) and top right panel shows the equivalent potential temperature on the model level closest to the surface.  The panel on bottom left shows the $u$ wind ($\rm{m s^{-1)}}$)
%  on the model level closest to the surface and the panel on bottom right shows the domain mean
%  $\rm{u}$ wind.}
  
  
%% surface enthalpy flux
%\begin{figure}
%  \centering
%  \begin{minipage}{0.4\textwidth}
%    \centering
%      \includegraphics[width=0.96\columnwidth]{walkerfigs/sfcenthalpyflx_3mods_lwcreOn.eps} 
%    \end{minipage}
%    \begin{minipage}{0.4\textwidth}
%    \centering
%      \includegraphics[width=0.96\columnwidth]{walkerfigs/sfcenthalpyflx_3mods_lwcreOff.eps}
%    \end{minipage}
%    \caption{Surface enthalpy flux (solid lines) for the three grid-spacings.  Dashed lines show the individual components 
%    (latent and sensible heat flux) of the enthalpy.
%    On the left are experiments with LWCRE on and on the right, the same experiments but with LWCRE off.}
%    \label{fig:enthalpy}
%\end{figure}


There are also differences in the lapse-rate below $700 \rm{hPa}$, in the upper-level clouds,
and the domain mean precipitation (Figure \ref{fig:precip_dom_mn}).
  
We can see clearly that the interactions between longwave radiation and clouds act to dry the troposphere in the regions of subsidence 
more than the cases when the longwave CRE is not active.  This is especially true for the simulations with a grid-spacing of 1 km 
and 2 km.  While the mid-tropospheric profiles of diabatic heating and cooling are similar between the LW CRE on/off simulations, 
there is a very strong response below about 850 hPa for the 25km model.  At that grid-spacing, with the 
LW CRE on, significant low-level clouds are formed, leading to a radiative cooling of almost 10K/day, 
and a radiative cooling of around -7K/day.   However, when the LW CRE is off, there is no appreciable 
difference in the low-level diabatic profiles among the E25, E2, and E1 experiments.   It 
is also clearly seen that the LW CRE decreases the upper-level ice-condensate for E25, E2, and E1 
experiments but dramatically increases the low-level liquid condensate for E25 and P25 (Figure 
\ref{fig:TotCond}).  

When the LW CRE is off, the simulations with resolutions of 1km, 2km, and 25km all show a similar 
distribution and magnitude of surface enthalpy flux (Figure \ref{fig:LowLevelVars}).  However, when the 
LW CRE is on, there are differences between the resolutions that can be as large as $80 \, \rm{W/
m^2}$.   These differences in surface enthalpy flux are due largely to the latent heat flux and are 
confined to the regions with ascending circulations over the warmer SST.  This manifestation of the strong interactions between clouds and the circulation 
would be entirely absent from high resolution simulations with gray radiation or fixed radiative cooling profiles.  

%% Domain mean vertical structure for 25km, 2km, and 1km resolutions
%% LW CRE on
%\begin{figure}
%  \includegraphics[width=0.8\columnwidth]{walkerfigs/Walker_VertStr_LWCRE_b.eps}
%  \caption{Domain mean vertical structure for simulations with a grid spacing of 25km (yellow, parameterized convection), 2km (blue), and 1km (green).  
%  All experiments have active LWCRE.  The dotted lines of RH show the mean over the cold pool regions.  Thin lines on the lapse rate panel show the moist adiabat for reference.}
%  \label{fig:vertst_lwcre}
%\end{figure}
%
%% Domain mean vertical structure for 25km, 2km, and 1km resolutions
%% LW CRE off
%\begin{figure}
%  \includegraphics[width=0.86\columnwidth]{walkerfigs/Walker_VertStr_LWCREoff.eps}
%  \caption{Domain mean vertical structure for simulations with a grid spacing of 25km (yellow, parameterized convection), 2km (blue), and 1km (green).  
%  All experiments have LWCRE off.  The dotted lines of RH show the mean over the cold pool regions.  Thin lines on the lapse rate panel show the moist adiabat for reference.}
%  \label{fig:vertst_lwcre_off}
%\end{figure}
%
%% RH and Psi for 1km, 2km, and 25km simulations			
%\begin{figure}
%  \centering
%  \begin{minipage}{0.49\textwidth}
%    \centering
%      %\includegraphics[width=0.5\columnwidth]{randomfigs/psi_rh_1km_lwoff.eps} 
%      \includegraphics[width=0.96\columnwidth]{walkerfigs/psi_rh_1km_ctl.eps} 
%    %\caption{{\bf default}}
%    \end{minipage}
%    \begin{minipage}{0.49\textwidth}
%    \centering
%      \includegraphics[width=0.96\columnwidth]{walkerfigs/psi_rh_1km_lwoff.eps}
%      %\caption{{\bf default}}
%    \end{minipage}
%    \begin{minipage}{0.49\textwidth}
%    \centering
%      \includegraphics[width=0.96\columnwidth]{walkerfigs/psi_rh_2km_ctl.eps}
%      %\caption{{\bf default}}
%    \end{minipage}
%    \begin{minipage}{0.49\textwidth}
%    \centering
%      \includegraphics[width=0.96\columnwidth]{walkerfigs/psi_rh_2km_lwoff.eps}
%      %\caption{{\bf default}}
%    \end{minipage}
%    \begin{minipage}{0.49\textwidth}
%    \centering
%      \includegraphics[width=0.96\columnwidth]{walkerfigs/psirh_noconv_ensind0_contint5.eps}
%      %\caption{{\bf default}}
%    \end{minipage}
%    \begin{minipage}{0.49\textwidth}
%    \centering
%      \includegraphics[width=0.96\columnwidth]{walkerfigs/psirh_noconv_lwoff_ensind3_contint5.eps}
%      %\caption{{\bf default}}
%    \end{minipage}
%    \caption{Relative humidity is shown by the shading while the mass streamfunction is given my the black contour lines.  The top row shows the 1km experiments, the middle row the 2km
%    experiments, and the bottom row the 25 km experiments.  Experiments with LWCRE are on the left and 
%    experiments without are on the right.}
%\end{figure}
%


\section{Conclusions}

The results presented in this paper show us that (scientific point learned).    Interactions between the clouds and radiation
act to spatially concentrate the circulation, strengthen the overturning circulation, and dry out the mid-troposphere where 
subsidence dominates (although domain mean condensate is less when LWCRE is on.).  

It is remarkable that the control simulations (LW CRE on) have a precipitation rate  that can vary by as much as 
20\% despite have have the same prescribed SST and incoming radiation.  The simulations also all have the same radiation,
turbulence, large-scale cloud and microphysics parameterizations and all use the same dynamical core.    

The longwave cloud radiative effect (LWCRE) has been shown to play a dramatic role in the organization of the precipitation
field and the low-level clouds.  This is primarily accomplished by decreasing the amount of precipitation that is produced 
by the large-scale precipitation scheme.  Two major differences between the GCM-like experiments and the CRM-like 
experiments are 4-5 times more low-level clouds and a much dryer mid-troposphere in the GCM-like experiments compared 
to the CRM-like experiments.  The LWCRE also highlights the impact of an increased vertical moisture transport 
in our CRM-like models.  

Several different configurations of a mock-Walker circulation have been used to analyze the response of clouds and the 
large-scale circulation to a simple Gaussian shaped prescribed SST pattern.  To better understand the role that clouds
and humidity play in determining and driving the circulation we have performed experiments with and without the radiative
effect of clouds, with and without the deep convective parameterization, across multiple domain sizes,  
and using a range of resolutions from 100 km to 1 km grid-spacing.  Our results imply that the convective 
parameterization and the longwave cloud radiative effect strongly interact with each and often lead to 
asymmetric results.   

Below is a list summarizing several of the changes that occur when the LW CRE is off.  Turning the LW CRE interactions off tends to: 

\begin{itemize}
  \item decrease the strength of the overturning circulation and the domain mean precipitation (P100,P25,E25)
  \item increase the midtropospheric RH by $5-15 \% $
  \item eliminate the difference of liquid condensate at 900 hPa between explicit and parameterized simulations.
  \item weaken the horizontally oriented low-level circulations.   
  \item spread out in geographic space the precipitation maxima
  \item homogenize the surface enthalpy flux over the warm pool, primarily through the latent heat flux
  \item decrease both the LTS and the EIS, over both the warm and cold regions
  \item decrease the lowest level $q, \theta_e,$ temperature $T$ and virtual temperature $T_v$.  
  \item eliminate large differences in the evaporation and lowest level temperature in the center of the domain between 
           the 1 and 2 \textit{km} simulations. 
  \item increase evaporation in the 1 and 2 \textit{km} cases, decrease it in the 25 \textit{km} case.
  \item amplify the larger values of ice condensate aloft for the explicit simulations.  
\end{itemize}

Relative to simulations with a grid-spacing of $25 \textit{km}$, the E2 and E1 experiments can be characterized as having: 
\begin{itemize}
  \item stronger overturning circulations (as measured by vertical velocity) which are more consistently centered over the maximum of SST.  
  \item higher relative humidity in the upwelling regions and aloft.   Between 600-800 hPa the explicit models can have a relative humidity 
  that is 40\% higher than in the parameterized, lower resolution simulations.  
  \item two to four times more ice condensate above 600 hPa but only about half as much liquid condensate below 700 hPa.
  \item less domain mean precipitation.  Values for 1 and 2 km simulations are in the 3.2-3.5 mm/d range, while those for the 25 km 
  simulations are 10-20\% higher (3.5-3.9 mm/d).  
  \item a weaker radiative cooling rate (about 2 K/d) and weaker heating rate (about 2 K/d).
  %\item much less vertical shear.  
\end{itemize}


There is a rich literature on tropical overturning circulations.  Much of it is focused on the Hadley circulation and while some
work has looked at the Walker circulation, it has received less attention.  Part of our motivation in using the framework of the 
Walker circulation comes from the fact that all of the major types of tropical cloud types occur regularly within this circulation
clouds and the circulation.  While this study has interpreted the experiments in the context of the Walker Circulation, our results 
are also relevant to the deep overturning circulations and meridional SST gradients that define the ITCZ and the Hadley 
Circulation.  In that context, our results are consistent with those of several recent studies (e.g. Harrop and Hartmann, 2016, 
Popp and Silvers, 2017, Dixit et al. 2018, Flaschner et al. 2018).  These studies, as well as the present one, all show that the 
LWCRE acts to constrain the deep convective region.  This results from an increased atmospheric energy uptake and 
strengthening of the overturning circulation where the deep convective clouds occur (Popp and Silvers, 2017).  Consistent with 
these papers, the present work also shows that the LWCRE has a strong influence on the low-level circulation.   When 
the LWCRE is turned off, the low-level circulations shift upward and are not as well organized (reference psi figures).
There is a corresponding change in the low-level cloud fields, longwave radiative cooling, and the domain mean precipitation.
For the experiments with a GCM-like configuration, the LWCRE strongly influences the precipitation from the large-scale
cloud scheme while leaving the precipitation from the convective parameterization scheme largely unchanged.  This leads 
to a much stronger response of the GCM-like experiments to the LWCRE, especially in the low-levels of the troposphere.  
Because we expect the fraction of precipitation that is due to the convective parameterization to be model dependent 
this could explain why GCMs have a large spread in their response to turning off the LWCRE (reference some 
of Aiko's work?). 

We have used the tropical overturning circulation to compare the multi-scale interactions between circulation patterns, cloud 
systems, and interactive radiation across experiments with grid-spacing ranging from 100km to 1km.  The flexible modeling
system at GFDL has allowed us to use a single code base in a GCM-like configuration with physics parameterizations 
that are very close to the AM4.0/CM4.0 models as well as in a CRM-like configuration with explicit convection.  While 
there are significant differences between the CRM presented in this paper and more conventional CRMs (e.g. vertical 
grid spacing and threshold based `binary' cloud scheme), the prospect of so easily converting a GCM into something 
like a CRM allows for an enticing testbed for future model development.
These comparisons have highlighted some of the unexpected behaviors of a GCM-like configuration when used with 
highly idealized boundary conditions.  For example, the consistent asymmetry of the circulation and precipitation relative 
to the fixed SST pattern and the dominance of the large-scale precipitation over the convective precipitation.  
The comparisons have also illustrated some 
of the challenges that arise when dramatically increasing the resolution of a GCM.  For example, the lack of shallow 
clouds (both convective and stratocumulus) and the difficulty of comparing clouds in this CRM to other CRMs due to the 
prognostic large-scale cloud scheme used at GFDL.

The goal in developing and using idealized models is to capitalize on their simplicity and learn something about the 
parent system as a result.  Idealized studies focusing on particular parts of the Earth system have shown that the danger of 
such idealized models is that they will not actually be less complex or easier to understand than the Earth system.  

The only difference between our simulations and radiative convective equilibrium is a gradient of SST at the lower boundary.
The addition of this small difference creates an undeniable link with the observed tropical atmosphere.  The idealized 
mock -Walker cell configuration serves as an important step between models of pure radiative convective equilibrium
and models which are applicable to a wider range of Earth like conditions.  


Although the radiative cooling and condensational heating
 are in many ways similar...   The 1 and 2 \textit{km} experiments transport much more condensate aloft and maintain a 
 relative humidity over the warm pool region that is about 40\% higher than the experiments with parameterized convection.

References: 

Bretherton et al. 2006
Jeevanjee et al. 2018
Kuang and Wofsy
Muller and Held, 2012
Popp and Silvers, 2017
Patrizio and Randall, 2019
Randall et al., 1989
Silvers et al. 2016
Slingo and Slingo, 1988
Stevens et al. 2012
Zhao et al. 2018a
Zhao et al. 2018b



% PDF for location of precip maximum of precipitation for experiments at both 25km and 100km resolutions 
% LW CRE on
%\begin{figure}
%  \centering
%      %\includegraphics[width=0.96\columnwidth]{walkerfigs/hist_maxP_4mods_lwcreon.eps}
%       \includegraphics[width=0.96\columnwidth]{walkerfigs/hist_4pan_25vs100_lwcre.eps}
%  \caption{Normalized probability distributions of the location of the maximum value of precipitation for experiments 
%  with the longwave CRE on.  All panels show cases with parameterized convection, the top panels are computed 
%  for the experiments having a domain width of 4000 km while the bottom panels are for experiments having a 
%  domain width of 16000 km.  The black curves show a Gaussian function with a standard deviation computed for 
%  each experiment and an expected location in the center of the domain where the SST is warmest. }
%\end{figure}


\appendix

%A couple of appendices with extra details which may not be included in the final paper.  

\section{Streamfunction}

For a two-dimensional flow in the $\lambda,p$ plane we can write the mass conservation equation in terms of a 
mass streamfunction $\psi$ as: 
\begin{equation}
u =  -\frac{\partial \psi}{\partial p}   \,\, {\rm and} \,\, \omega =\frac{\partial \psi}{a \rm{cos} \phi \partial \lambda}
\end{equation}
with $a$ as the radius of Earth.
This satisfies the continuity equation: 
\begin{equation}
\frac{\partial}{a \rm{cos} \phi\partial\lambda}\left(-\frac{\partial \psi}{\partial p}\right)+\frac{\partial}{\partial p}\left(\frac{\partial \psi}{a \rm{cos}\phi \partial \lambda}\right)=0.
\end{equation}
The streamfunction is therefore defined by two equations and can be solved with either.  The choice is often made based on the 
boundary conditions.  Dimensionally, $\psi$ must have units of velocity multiplied by pressure, or $\rm{kg/s^3}$.  Often, the streamfunction
is computed as a mass streamfunction with units of $\rm{kg/s}$.  Density does not appear above
because pressure is being used as the vertical coordinate.   With height as the vertical coordinate, mass is given by $\rho dxdydz$, 
assuming a hydrostatic atmosphere and rewriting in terms of pressure mass is given by $-(1/g) dxdydp$.
Scaling the equations above by $a/g$ results in $\psi$ having units of $\rm{kg/s}$.  

Solving for $\psi$ by integrating the vertical velocity is possible, but requires a boundary condition along one edge of the domain between the surface and the 
top of the atmosphere.  Solving for $\psi$ by integrating the horizontal velocity allows us to set $\psi=0$ along the upper edge of the domain.  
This is both simple, and physically motivated.  

% computed with matlab, see StreamFunNew.m script
\begin{equation}
\psi_{i,j+1}= \psi_{i,j}-\frac{a}{g}\sum_{j=32}^1 \left(u_{i,j+1}(p_{j+1}-p_{j})\right).
\end{equation}

In the previous equation, $w, u,$ and $\rho$ have all been averaged in space and time and $p$ is the pressure on full model levels.  The 
radius of Earth is $a$ and the acceleration due to gravity by $g$.  The density, $\rho$ is computed as $ \rho_{i,j}=\frac{p_j}{R T_{v;i,j}}  $  where $R$ and $T_v$ are the gas constant for dry air ($287 \rm{J/kg K}$) and the virtual temperature ($T_v=T(1+q/\epsilon)/(1+q)$), respectively.  The specific humidity is given by $q$ and $\epsilon = 0.622$.


\section{Cloud Water and Cloud Fraction}

The last several generations of the GFDL atmospheric models have used the parameterization developed by Tiedtke (1993) to prognostically compute the grid-cell averaged cloud water ($l$) and cloud fraction ($a$).  The formulation for the local time rate of change from Tiedtke is 
\begin{equation}
  \frac{\partial l}{\partial t} = A(l)+S_{cv}+S_{BL}+C-E-G_{p}-\frac{1}{\rho}\frac{\partial}{\partial z}(\rho\overline{w'l'})_{entr}
\end{equation}
and 
\begin{equation}
  \frac{\partial a}{\partial t} = A(a)+S(a)_{cv}+S(a)_{BL}+S(a)_C-D(a).
\end{equation}
Transport of $l$ or $a$ through the boundaries of a grid is given by $A(l)$ or $A(a)$, the terms ($S_{cv}, S(a)_{cv}, S_{BL}, S(a)_{BL}$, and $S(a)_C$) are the sources of cloud water or cloud fraction from convection, boundary layer turbulence, and condensation, respectively.  The condensation/sublimation rate is given by $C$, $E$ is the evaporation of cloud water, $G_{p}$ is the rate of generation of precipitation by microphysical processes and the $\overline{w'l'}$ term is the flux divergence from entrainment at the top of the cloud layer, and lastly, $D(a)$ is the sink of cloud fraction due to evaporation.

Often, cloud resolving models, or cloud system resolving models, do not predict partially cloudy grid cells, but rather consider a cell to be either cloudy or clear.  If the goal is to more directly compare to cloud resolving models then cloud fraction should not be prognostically predicted but simply tied to a certain value of total condensate in a grid-cell.  However, in the Tiedtke system, the prognostic equation for cloud water depends on the cloud fraction.  As given by equations 16, 24, 26,28, 30, and 33 in Tiedtke's paper, this dependence is present in the $S_{BL}$, $C$, $E$, and $G_{p}$ terms. 

\acknowledgments
Funding for this work came from both GFDL and Stony Brook.  Data and scripts are available from author upon request

\bibliography{Silvers_WalkerCell}

%\bibliography{references}
%\begin{thebibliography}{}
%
%\bibitem[{\textit{Bretherton et~al.}(2005)\textit{Bretherton, Blossey, and
%  Khairoutdinov}}]{Bretherton_etal_2005}
%Bretherton, C.~S., P.~N. Blossey, and M.~Khairoutdinov (2005), An
%  energy-balance analysis of deep convective self-aggregation above uniform
%  sst, \textit{J. Atmos. Sci.}, \textit{62}, 4273--4292.
%
%%\bibitem[{\textit{Jeevanjee and Romps}(2013)}]{Jeevanjee_Romps_2013}
%%Jeevanjee, N., and D.~M. Romps (2013), Convective self-aggregation, cold pools,
%%  and domain size, \textit{Geophys.\ Res.\ Lett.}, \textit{40}, 994--99,
%%  \doi{10.1002/grl.50204}.
%  
%\end{thebibliography} 
  
%%% Acknowledgements
%\begin{acknowledgments}
%(Levi's acknowledgements)
%\end{acknowledgments}

\end{document}  
