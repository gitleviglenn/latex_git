\documentclass[11pt]{amsart}   	% use "amsart" instead of "article" for AMSLaTeX format
\usepackage{geometry}                		% See geometry.pdf to learn the layout options. There are lots.
\geometry{letterpaper}                   		% ... or a4paper or a5paper or ... 
%\geometry{landscape}                		% Activate for for rotated page geometry
%\usepackage[parfill]{parskip}    		% Activate to begin paragraphs with an empty line rather than an indent
\usepackage{graphicx}				% Use pdf, png, jpg, or eps� with pdflatex; use eps in DVI mode
								% TeX will automatically convert eps --> pdf in pdflatex		
\usepackage{amssymb}
\usepackage{gensymb}

\title{Double and Single ITCZs with and without Clouds}
\author{Max Popp and Levi G. Silvers}
%\date{2016}							% Activate to display a given date or no date

\begin{document}
\maketitle
%\section{}
%\subsection{}

\section{Abstract}
A major bias in tropical precipitation over the Pacific in climate simulations stems from the models' tendency to produce two strong distinct intertropical convergence zones (ITCZs) too often. Several mechanisms have been proposed that may contribute to the emergence of two ITCZs, but current theories cannot fully explain the bias. This problem is tackled by investigating how the interaction between atmospheric cloud- radiative effects (ACREs) and the large-scale circulation influences the ITCZ position in an atmospheric general circulation model. Simulations are performed in an idealized aquaplanet setup and the longwave and shortwave ACREs are turned off individually or jointly. The low-level moist static energy (MSE) is shown to be a good predictor of the ITCZ position. Therefore, a mechanism is proposed that explains the changes in MSE and thus ITCZ position due to ACREs consistently across simulations. The mechanism implies that the ITCZ moves equatorward if the Hadley circulation strengthens because of the increased upgradient advection of low-level MSE off the equator. The longwave ACRE increases the meridional heating gradient in the tropics and as a response the Hadley circulation strengthens and the ITCZ moves equatorward. The shortwave ACRE has the opposite effect. The total ACRE pulls the ITCZ equatorward. This mechanism is discussed in other frameworks involving convective available potential energy, gross moist stability, and the energy flux equator. It is thus shown that the response of the large-scale circulation to the shortwave and longwave ACREs is a fundamental driver of changes in the ITCZ position.

\section{Key Points}
\begin{itemize}
  \item{Experiments investigate how the location of the ITCZ depends on the interaction between ACREs and the large-scale circulation}
  \item{Low-level moist static energy is shown to be a good predictor of the ITCZ position}
  \item{Longwave cloud radiative effects drive the ITCZ equatorward while shortwave cloud radiative effects drive it poleward}  
\end{itemize}


\begin{figure}
  \centering 
    \includegraphics[width=2.5in]{poppsilvers_fig1.jpg}
   \caption{Zonal-mean quantities in steady state: (a) the total (convective + large-scale) precipitation, (b) the vertically integrated total (liquid + frozen) water path, (c) the atmospheric energy uptake, and (d) the ACRE. Colors represent the individual experiments ACREoff (black), ACREonLW (red), ACREonSW (blue), and ACREon (orange). The filled circles show the global and the crossed circles the tropical means (from 30�S to 30�N) of precipitation in (a). The horizontal location of the dots is only for visual separation and does not have any particular meaning. Note that the horizontal axes are scaled with the cosine of the latitude.}
   \label{fig:lambdaecs}
\end{figure}




\end{document}  