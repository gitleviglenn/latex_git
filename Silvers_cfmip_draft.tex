\documentclass[11pt]{article}   	% use "amsart" instead of "article" for AMSLaTeX format
\usepackage{geometry}                		% See geometry.pdf to learn the layout options. There are lots.
\geometry{letterpaper}                   		% ... or a4paper or a5paper or ... 
%\geometry{landscape}                		% Activate for for rotated page geometry
%\usepackage[parfill]{parskip}    		% Activate to begin paragraphs with an empty line rather than an indent
\usepackage{graphicx}				% Use pdf, png, jpg, or eps� with pdflatex; use eps in DVI mode
								% TeX will automatically convert eps --> pdf in pdflatex		
\usepackage{amssymb}
\usepackage{gensymb}
\usepackage{epstopdf}
%\usepackage{pdflatex}
%\usepackage{epsfig}

			% Activate to display a given date or no date
%\title{Using the Walker Circulation to connect low-level cloud changes to deep convective entrainment}
\title{Clouds and Sensitivities Across a Hierarchy of GFDL CMIP6 Models}
\author{Levi G. Silvers et al.}

\begin{document}
\maketitle

\begin{abstract}
The newest atmospheric climate model at GFDL, AM4, succeeded at significantly reducing the TOA radiative flux biases as compared to CERES observations.  Despite a relatively low top-of-atmosphere sensitivity to uniform warming of SSTs (Cess warming experiments), the corresponding coupled climate model, CM4, has high transient and equilibrium climate sensitivities.  

We will present a systematic picture of the modeled clouds across a hierarchy of model configurations which utilize this atmospheric model.  This hierarchy includes the CFMIP Aquaplanet and AMIP experiments, fully coupled model experiments (using GFDL's CM4 model) as well as additional AMIP-like experiments with particular SST patterns.  This demonstrates the large range of sensitivities that are possible from a single atmospheric climate model.   Looking at the global mean radiative feedbacks across the different model configurations as well as in the context of CMIP5 and CMIP6 models will allow us to assess to what extent the cloud feedbacks in the idealized experiments relate to the fully coupled experiments and to observed clouds.  
\end{abstract}

%\section{Key Questions}

\textbf{Key Points}
\begin{itemize}
  \item{At 25 km, simulations with parameterized shallow- and deep-convection exhibit asymmetries in the circulation, 
  precipitation, and humidity fields resulting in a steady-state overturning circulation that is not centered on the SST
  maximum.}
  \item{The longwave cloud-radiative effect is a fundamental factor in establishing these asymmetries and leads to dramatic
  changes in the low-level clouds and boundary layer structure.}
  \item{Cloud resolving simulations result in stronger overturning, more condensate aloft, and a RH in the deep convective 
  region that is 30\% higher than for the coarser cases with parameterized convection.}
  \item{The mock-Walker Circulation is RCE plus an imposed gradient in surface temperature.  This is a simple design that 
  incorporates additional complexity which is critical for the connection to observed atmospheric phenomena.}
\end{itemize}

\section{introduction}

This paper has a double motivation.  One of the motivations is to illustrate the wide range of sensitivities to perturbations 
that can be achieved with a single base model.  For this purpose we utilize several models from the latest suite of GFDL 
models.  This constitutes a hierarchy of configurations that ranges from an atmosphere only aquaplanet to a fully coupled 
climate model.  Previously the range of climate sensitivities and feedbacks among the large ensemble of climate models 
the world has to offer have been documented largely by examining the multi-model mean results with little attention to 
the variability of sensitivity within a single modeling framework (IPCC; Andrews et al., 2012; Ringer et al., 2014).  By 
focusing on the models from a single modeling center we hope to bring out fresh insight that is lost within a large ensemble.

An additional motivation for this paper is to document the simulation of clouds in the newly developed GFDL models that 
have participated in CMIP6.  This will largely be done in the context of the amip experiment and comparisons to observations
that overlap with that time period (1979-2014).  The experiments and diagnostics used are taken from the latest 
round of CFMIP which has provided a useful framework to guide experimental design and diagnostic outputs.       

\section{Clouds in AM4.0: Comparison between amip and observations}

\begin{figure}
% figure generated with calip25_isccp_am4_totcld.ncl, figurenumber = 1
  \includegraphics[width=0.86\columnwidth]{cfmipfigs/clt_isccp_calipso_am4_4pan.eps}
  \caption{Comparison of total cloud fraction from AM4 with two observational data sets: ISCCP (left) and CALIPSO (right).}
  \label{fig:clt_isccp_calipso}
\end{figure}

\begin{figure}
% figure generated with calip25_isccp_am4_totcld.ncl, figurenumber = 3
  \includegraphics[width=0.86\columnwidth]{cfmipfigs/clt_calipso_obs_sim_am4.eps}
  \caption{Comparison of total cloud fraction.  Top shows observations from CALIPSO, processed for GCM comparison.  Middle 
  shows output from the CALIPSO simulator, bottom shows raw model output of clt without a simulator.}
  \label{fig:calipso_sim_vs_mod}
\end{figure}

\begin{figure}
% figure generated with cloud_table_AM4fig.ncl
  \includegraphics[width=0.86\columnwidth]{cfmipfigs/ctp_vs_tau_isccp_modis_am4_obs.eps}
  \caption{Joint histograms showing the variations of cloud fraction as a function of cloud top pressure and tau.}
  \label{fig:calipso_sim_vs_mod}
\end{figure}

\begin{figure}
% figure generated with cloud_table_AM4fig.ncl
  \includegraphics[width=0.86\columnwidth]{cfmipfigs/calip_9pan_amip_obs.eps}
  \caption{Calipso scumstash}
  \label{fig:calipso_9pan}
\end{figure}


\section{The Role of Clouds in AM4.0/CM4.0/Aquaplanets in Determining and Influencing Climate Sensitivity}

In an attempt to compare to the values of feedback that were reported in Ringer et al., 2014, we use the same method 
to compute the radiative feedbacks for the GFDL models.  That is, for the abrupt $4xCO_{2}$ experiment, feedbacks are 
inferred to be the slopes of a linear regression calculation using annual, global-mean top-of-atmosphere flux anomalies (Gregory et al., 2004).  From the atmosphere only experiments (amip-p4K, amipFuture, and aqua-p4K), the feedbacks are calculated as 
differences in top-of-atmosphere flux anomalies between the perturbation and control experiments.   

Brient and Bony (2012) proposed what they call the $\beta$ feedback as a way to quantify whether or not ' models have 
changes in low-cloud cloud radiative effects which are proportional to the strength of the cloud radiative effects of their
low clouds.'  See $Stevens_etal_Cookie_overview$ for details.   That paper asks the question of whether single 
model relationships between low-cloud amounts and the sensitivity of low clouds to warming can be 
masked by changes in multi-model ensembles.  This should be checked for the GFDL model in this paper.  

The cloud radiative effect (CRE) is computed as the difference between the all-sky and clear-sky TOA fluxes.  

\begin{table}
\begin{center}
\caption{Global mean radiative feedbacks.  Compare to Ringer et al. 2014.}
    \begin{tabular}{*{5}{c}}
    \hline
    \hline
 Feedback & Aqua p4K & Amip p4K & Amip Future & Abrupt 4xCO2    \\ \hline
    Net          &   -2.2    &  -1.6              & -1.8           &    -0.84 (from Tim)         \\ 
    \\
    Net CRE      & 0.3      & -0.1            & 0.1           &?            \\  
    \\
    LW CRE       & 0.4       & 0.1            & 0.1           & ?              \\  
    \\
    SW CRE      & -0.7       & -0.1          & -0.2          & ?              \\  
    \\
    LW CLR       & 2.0       & 2.0             & 2.0           & ?             \\  
    \\
    SW CLR      & -0.1      & -0.3            & -0.3          & ?                   \\  \hline

    \end{tabular}\par
    %\bigskip 
    \label{tab:lambda}
\end{center}
\end{table}

\begin{figure}
% figure generated with cloud_table_AM4fig.ncl
  \includegraphics[width=0.86\columnwidth]{cfmipfigs/amip_p4k_future_cldfrac_calip.eps}
  \caption{Calipso scumstash}
  \label{fig:calipso_amip_comparison}
\end{figure}

\end{document}
