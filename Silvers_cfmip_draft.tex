\documentclass[11pt]{article}   	% use "amsart" instead of "article" for AMSLaTeX format
\usepackage{geometry}                		% See geometry.pdf to learn the layout options. There are lots.
\geometry{letterpaper}                   		% ... or a4paper or a5paper or ... 
\usepackage{graphicx}				% Use pdf, png, jpg, or eps§ with pdflatex; use eps in DVI mode
								% TeX will automatically convert eps --> pdf in pdflatex		
\usepackage{amssymb}
\usepackage{gensymb}
\usepackage{epstopdf}

			% Activate to display a given date or no date
%\title{Using the Walker Circulation to connect low-level cloud changes to deep convective entrainment}
\title{Clouds and Sensitivities Across a Hierarchy of GFDL CMIP6 Models}
\author{Levi G. Silvers et al.}

\begin{document}
\maketitle

\begin{abstract}
Clouds observed from multiple satellites are compared to the clouds simulated by GFDL's new AM4 climate model.  
The new atmospheric climate model succeeded at significantly reducing the TOA radiative flux biases as compared 
to CERES observations.  Despite a relatively low top-of-atmosphere sensitivity to uniform warming of SSTs 
(Cess-type warming experiments), the corresponding coupled climate model, CM4, has high transient and 
equilibrium climate sensitivities.  

We will present a systematic picture of the modeled clouds across a hierarchy of model configurations which utilize this atmospheric model.  This hierarchy includes the CFMIP Aquaplanet and AMIP experiments, fully coupled model experiments (using GFDL's CM4 model) as well as additional AMIP-like experiments with particular SST patterns.  This demonstrates the large range of sensitivities that are possible from a single atmospheric climate model.   Looking at the global mean radiative feedbacks across the different model configurations as well as in the context of CMIP5 and CMIP6 models will allow us to assess to what extent the cloud feedbacks in the idealized experiments relate to the fully coupled experiments and to observed clouds.  
\end{abstract}


\textbf{Key Points}
\begin{itemize}
  \item{The AM4.0 TOA cloud radiative effect shows small biases compared with CERES-EBAF observations, both versions 
  2.8 and 4.1.}
  \item{Too few clouds are simulated by AM4.0 at high, middle, and low-levels, as compared to observations from CALIPSO.}
  \item{AM4.0 low-level clouds have a negative bias (relative to MISR) that is larger for thin clouds, relative to thick.}
  \item{Various configurations of a single global climate model lead to climate sensitivities that range from 2.1K (Cess) to 
  about 5K (Winton's ECS).}
\end{itemize}

\section{Introduction}

% thoughts from Zhao et al. 2018 part 1, section 4
% Attention to the observed TOA radiative fluxes were given a high priority during the development of AM4.0 first 
% of the relatively high confidence in the observations, and second because it was anticipated that these fluxes
% are an important component in the coupled simulations of CM4.0.   Figure 2 shows that AM4.0 does a better 
% job simulating RLUT and RSUT than any other of the compared CMIP5 models.   Not only was the global 
% mean bias of (-0.77 W/m2) an improvement over AM3 and AM2.1 (-4.11 and -3.16 W/m2), but the spatial pattern 
% was improved over almost the entire world ocean.   Fluxes at the TOA for AM4.0 largely corrected over reflection
% throughout most of the oceans and an over absorption of solar radiation in the Southern Hemisphere ocean 
% south of ~60 degrees.   
 

From Poster
It has been known for decades that  the spread of cloud feedbacks among models contributes to a large portion of the spread in climate sensitivity among those same models (Cess et al., 1989; Bony and Dufresne, 2005).  It also now clear that the climate sensitivity estimated from constant temperature perturbation experiments is not a good indicator of the climate sensitivity that is estimated from regression techniques of abrupt 4xCO2 warming experiments (e.g. Andrews et al., 2012), and that both of these measures of sensitivity underestimate the equilibrium sensitivity that is attained when a coupled climate model reaches equilibrium after multiple millennia (Paynter et al., 2018).      

Variation of the climate sensitivity within individual models has been tied to the influence of particular surface temperature patterns.  This work assesses the fidelity of the clouds in AM4.0/CM4.0 with observations and compares clouds from experiments with differing sensitivities. 
end from Poster

This paper has a double motivation.  First, to illustrate the wide range of sensitivities to perturbations 
that can be achieved with a single base model.  For this purpose we utilize several models from the latest suite of GFDL 
models.  This constitutes a hierarchy of configurations that ranges from an atmosphere only aquaplanet to a fully coupled 
climate model.  Previously the range of climate sensitivities and feedbacks among the large ensemble of climate models 
the world has to offer have been documented largely by examining the multi-model mean results with little attention to 
the variability of sensitivity within a single modeling framework (IPCC; Andrews et al., 2012; Ringer et al., 2014).  By 
focusing on the models from a single modeling center we hope to bring out fresh insight that is lost within a large ensemble.

A second motivation for this paper is to document the simulation of clouds in the newly developed GFDL models that 
have participated in CMIP6.  This will largely be done in the context of the amip experiment and comparisons to observations
that overlap with that time period (1979-2014).  The experiments and diagnostics used are taken from the latest 
round of CFMIP which has provided a useful framework to guide experimental design and diagnostic outputs.       

Can we shed lite on which particular component of the cloud feedbacks lead to large changes in the sensitivity?  

\section{Clouds in AM4.0: Comparison between AMIP and observations}



\begin{figure}
% figure generated with calip25_isccp_am4_totcld.ncl, figurenumber = 1
  \includegraphics[width=0.95\columnwidth]{/Users/silvers/Research/cfmip_paper/cf_misr_cldfrac_allheights.eps}
  \caption{Observed and modeled clouds for all heights.  Clouds shown are divided into thick (optical depth between 3.6 and 23) 
  and thin (optical depth between 0.3 and 3.6) 
  clouds.  The observations (bottom) are from the MISR satellite (2000-2013) and the modeled clouds (top) were 
  derived with the MISR simulator.  Mean cloud fraction values for MISR 
  (AM4) are 32\% (24\%) for thick clouds and 22\% (12\%) for thin clouds.}
  \label{fig:misr}
\end{figure}

\begin{figure}
% figure generated with calip25_isccp_am4_totcld.ncl, figurenumber = 3
  \includegraphics[width=0.86\columnwidth]{/Users/silvers/Research/cfmip_paper/cf_misr_cldfrac_below3km.eps}
  \caption{As in previous figure except that here only clouds below 3km are shown.  Clouds shown are divided into thick 
  (optical depth between 3.6 and 23) and thin (optical depth between 0.3 and 3.6) 
  clouds.  The observations (bottom) are from the MISR satellite (2000-2013) and the modeled clouds (top) were derived 
  with the MISR simulator (amip experiment). Mean cloud fraction values for MISR 
  (AM4) are 21\% (17\%) for thick clouds and 17\% (6\%) for thin clouds}
  \label{fig:calipso_sim_vs_mod}
\end{figure}

Before discussing the climate sensitivities of AM4/CM4 and how they are influenced by cloud feedbacks we assess
the clouds which are simulated by AM4.0 over the recent observational period of 1979-2014.  This is done by comparing
satellite based observations with the clouds in the AMIP experiment that is part of GFDLs contribution to CMIP6.  In 
order to compare the observed and modeled clouds as consistently as possible across multiple platforms we have 
used a selection of the satellite simulators which are included in COSP as part of the efforts of the CFMIP community 
(Bodas-Salcedo et al. 2011).  This paper highlights simulators of the algorithms for three satellites based datasets: 
ISCCP (the International Satellite Cloud Climatology Project), MODIS (the Moderate Resolution Imaging Spectroradiometer), 
MISR (the Multiangle Imaging Spectro-Radiometer), and CALIPSO (the Cloud-Aerosol Lidar and Infrared Pathfinder 
Satellite Observation; Chepfer et al., 2010; Cesana and Chepfer, 2013).  By using multiple complimentary satellite 
datasets and simulators we can benefit from 
the different strengths of each satellite while largely circumventing the known problems and discrepancies of the 
observational data sets.      

Discrepancies among the many observational data sets has been extensively discussed in the literature and are 
beyond the scope of this paper.  The curious reader can find detailed discussions in Kay et al., 2012; 
Marchand et al. 2010 and?; Pincus et al., 2012; Cesana et al. ?; Stephens et al., 2018.  However, we note that ISCCP is useful due to 
the longevity of the record (1983 - 2008 in Z18 figure) and MODIS has a detection algorithm that is an improvement upon ISCCP's and
thus does a better job of detecting mid-level clouds and excluding partly filled pixels.   Of the three passive satellite products 
(ISCCP, MODIS, and MISR), MODIS is considered to have the most accurate readings of high-topped clouds.  
The height of clouds is determined with a stereo height retrieval method in MISR that is distinct from that used by
both ISCCP and MODIS.  Because of this we have more confidence in the low and mid-level cloud readings 
from MISR, as well as the cumulus clouds (Marchand et al. 2010; Zhang et al., 2019).   The CALIPSO observations
are derived from a Lidar that allows for a direct measure of vertical structure of the clouds.  

\begin{figure}
% figure generated with cloud_table_AM4fig.ncl
  \includegraphics[width=0.86\columnwidth]{cfmipfigs/ctp_vs_tau_isccp_modis_am4_obs.eps}
  \caption{Joint histograms showing the variations of cloud fraction as a function of cloud top pressure and tau.
  Figure taken from Zhao et al., 2018}
  \label{fig:calipso_sim_vs_mod}
\end{figure}

\begin{figure}
% figure generated with cloud_table_AM4fig.ncl
  \includegraphics[angle=90,width=0.86\columnwidth]{cfmipfigs/calip_9pan_amip_obs_vs_AM4.eps}
  \caption{Cloud Fraction observed by CALIPSO (2007-2016) and modeled with the CALIPSO satelite simulator 
  (amip experiment).
  Dashed line shows meridional profiles of AM4.0 cloud fraction and solid line shows
  the meridional profile of observed cloud fraction.  With only a few exceptions in the polar regions, AM4 underestimates
  clouds in all three vertical ranges with the largest underestimate being of low-level clouds between +/-30.}
  \label{fig:calipso_9pan}
\end{figure}

As a starting point we look at the total cloud fraction in Figure \ref{fig:clt_isccp_calipso}.  The ISCCP and 
CALIPSO products agree fairly well on the 
global mean cloud fraction (65 and 67\%, respectively).   By this measure there are too few clouds in AM4.0 
(biases of -15 and -12 \%).   The ISCCP and CALIPSO data agree well over the oceans, with most of the differences
occuring over land (particularly north of 30).    

\begin{figure}
% figure generated with calip25_isccp_am4_totcld.ncl, figurenumber = 1
  \includegraphics[width=0.75\columnwidth]{cfmipfigs/ISCCP_CALIPSO_clt_correctedBiasPattern_nogrid.eps}
  %\includegraphics[width=0.75\columnwidth]{cfmipfigs/clt_isccp_calipso_am4_4pan.eps}
  \caption{Comparison of total cloud fraction from AM4 with two observational data sets: ISCCP (left) and CALIPSO (right).}
  \label{fig:clt_isccp_calipso}
\end{figure}

The MISR satellite is thought to provide good estimates of low-level clouds heights and amounts.  For this reason
and for comparison with both Kay et al., 2012, and Zhang et al., 2019 we compare observations from MISR
and the AM4.0 simulator output in Figure \ref{fig:misr}.  The optical depth is used to distinguish
between the thick and thin clouds.  Consistent with comparisons with ISCCP and CALIPSO we see that AM4.0
has fewer clouds than observed.  The pattern of thick clouds observed by MISR is reproduced fairly well, except in 
the tropical Indian Ocean.  The pattern of thin clouds is not reproduced as well, particularly in the southern hemisphere 
tropics.   

The global structure of the cloud fields are shown for the upper level, mid level, and low levels in 
Figure \ref{fig:calipso_9pan}.  The longitudinally averaged cloud fraction is shown as a function of latitude in the 
left most panels.  There is a large under-estimation of cloud fraction by (AM4.0: dashed; CALIPSO: solid) between 
plus/minus 30 
degrees for both mid and low-level clouds.   AM4.0 does a much better job with the upper level tropical clouds 
although there is still a negative bias.   The global maps detail where the biases are occurring.  In the upper 
levels the clouds are underestimated roughly equally among the deep-tropical and stratocumulus regions.  In the
mid-levels the bias tends to be larger over tropical land and in the region of the inter-tropical convergence zone.  
In the mid-levels there are large biases over much of the tropical Pacific ocean, as well as in the stratocumulus
zones and the southern tropical Atlantic ocean.  

Paragraph summarizing the basic characteristics of clouds in AM4.0 as stated by Ming's papers.  We could also summarize 
what David finds about clouds in AM4.0 in his paper.

During the developmental of the AM4.0 GFDL model particular care was given to ensuring that the energy budget at 
the top of the atmosphere compared well with observations.  The data set that was most heavily leaned on to ensure 
this was the CERES-EBAF TOA radiative fluxes, version 2.8 (Loeb et al., 2009).  Based on AM4.0 data that was used for 
the CMIP6 AMIP
experiment, the total TOA flux bias is 989898, the bias in CRE 
due to longwave radiative fluxes is $-2.4 \rm{W m^{-2}} $ and the bias due to CRE of the shortwave radiative fluxes   
is $-1.1 \rm{W m^{-2}} $.  About the same time the AM4.0 development was frozen and the process of documentation 
underway, the CERES-EBAF was updated to version 4.1.  When comparing the same AM4.0 AMIP experiment to the 
newer version of CERES-EBAF data we find biases of $-2.2 \rm{W m^{-2}} $ for the longwave CRE and 
$-2.9 \rm{W m^{-2}} $ for the shortwave CRE.   Should these biases be computed using only the AMIP years that 
correspond to the CERES years?   As previously discussed (e.g. Kay et al., 2012) these longwave and shortwave CRE 
biases are difficult to interpret because model developers often tune models to match particular CERES-EBAF data 
sets despite the well known fact that different versions of the CERES-EBAF data lead to changes in the observed CRE of
several $\rm{W m^{-2}}$.   Our aim in discussing this for AM4.0 is transparency in the model development process, as 
well as to point out that both versions of the CERES-EBAF data show similar shortcomings of the spatial patters of the 
simulated clouds  from AM4.0 in the AMIP experiment.   



\section{The Role of Clouds in AM4.0/CM4.0/Aquaplanets in Determining and Influencing Climate Sensitivity}

% A paragraph, either here or earlier that clearly states to what degree and why we think that changes in clouds
% are a critical element in determining why different models have different sensitivities.  Papers that should probably
% be cited are Bony and Dufresne, 2005, Webb et al. (the convection off paper), the Bretherton review on cloud 
% feedbacks, the CFMIP paper?  It would be  nice to cite more of the original papers and less of the review papers.  
% Probably some of Zelinka's papers.  

\begin{figure}
  \includegraphics[width=0.86\columnwidth]{cfmipfigs/cf_clim_amip_hist.eps}
  \caption{Cloud fraction between 1979-2014 for the amip and historical experiments.  The CALIPSO
  observations range from January 2007 through December 2016.  Note that the regions poleward of
  50 degrees are the only regions in which AM4/CM4 produce more clouds than observed.  It is 
  also interesting that having an interactive surface only leads to a minimally improved CF bias.}
  \label{fig:calipso_amip_comparison}
\end{figure}


The recent model development efforts at GFDL, and participation in CMIP6/CFMIP have provided a 
unique/valuable opportunity to analyze a group of models that are built on the same code base while 
spanning a wide range of model configurations.  The model types include (should we include any of the
efforts from the RCE or Walker Cell configurations?) an Aquaplanet, an AGCM (AM4.0), 
and an AOGCM (CM4.0).  When combined with a variety of boundary conditions this creates a hierarchy of
models which are all derived from or dependent on the same atmospheric code base (AM4.0).  
%It is a bit like a group of siblings sharing the same gene pool but at various stages of intelligence or education.      
Our hypothesis for this study is that the large range of climate sensitivities that are found within this 
group of models is in large part due to variations in the clouds (both their climatology and their feedbacks).
The clouds in these simulations respond to a variety of different conditions including different
patterns of prescribed SST, the presence or absence of land/ice, and in the case of the aquaplanet
experiments, a lack of cloud-aerosol interactions which are present in the amip and coupled experiments.  

Due to a combination of changing computing power and differing priorities, the sensitivity of Earth's climate to 
perturbations has been estimated in many different ways.  Three very common methods are documented 
here.  First, the climate sensitivity can be estimated from the change in radiative flux at the top of the 
atmosphere after imposing a prescribed change of surface temperature.  This method of using SST changes
as a proxy for climate change was pioneered by Robert Cess (e.g. 1989, 1990).  Second, the climate
sensitivity can be estimated from a fully coupled experiment in which the CO2 concentration is instantaneously 
quadrupled relative to the pre-Industrial values.  Linear regression can then be used to estimate the 
sensitivity of the model (Gregory et al., 2004, Andrews we al., 2012, IPCC).  This is the origin of the 
so-called equilibrium climate sensitivity in Andrews et al., 2012 and the IPCC.  Third, the climate sensitivity
can be computed from simply measuring the change of surface temperature after the CO2 concentration
has been doubled in a fully coupled climate model following an increase of 1\% per year in the CO2 concentration.
This defines the transient climate response.  
    
The DECK experiments of CMIP6 can be used to provide a measure of the climate sensitivity based on 
three different fully coupled experiments.  The methods used to determine the sensitivity are distinct, and the resulting 
differences in the measures of sensitivity reveal important details of the physics that underly the response
of the climate system to a perturbation such as increasing concentrations of carbon dioxide.  The three 
experiments are the historical simulation (historical: 1850-2014),  the abrupt quadroupling of 
$\rm{CO}_2$ concentration ($abrupt-4xCO2$: 300 years), and the experiment in which the 
concentration of $\rm{CO}_2$ is increased by 1\% 
 per year from the pre-industrial value of 284.262 ppm(?units?) until the concentration is doubled ($1pctCO2$: 150 years, doubling at year 70).  
The historical simulation is our best attempt at using our coupled GCM to reproduce the historical 
record found in observations.  This has frequently been used to estimate the climate sensitivity using the 
energy budget method (Otto et al., 2013, Armour et al., Andrews et al., Winton et al., 2020).  The 
$abrupt-4xCO2$
experiment has been utilized by the IPCC to derive measures ECS among the CMIP5 and CMIP6 models
using a linear regression of the change in TOA radiative fluxes and surface air temperature over the first 
150 years of simulation (Gregory et al., 2004; Andrews et al., 2012).  In contrast to the abrupt change of 
$\rm{CO}_2$ concentration, the $1pctCO2$ experiment allows the Earth system to adjust to 
a 1 \% per year increasing of $\rm{CO}_2$ concentration until they are doubled relative to the 
pre-Industrial values.  The sensitivity at this point is defined 
as the Transient Climate Response (TCR).  The TCR is the relevant measure of climate sensitivity for 
near term climate change because it is expected that the atmospheric concentrations of $\rm{CO}_2$ 
will reach (insert 2xCO2 value) around the middle of this century.         

% Ringer et al 2014 comparison
In order to compare to the values of feedback that were reported in Ringer et al., 2014, we use the same method 
to compute the radiative feedbacks for the GFDL models.  That is, for the abrupt $4xCO_{2}$ experiment, feedbacks are 
inferred to be the slopes of a linear regression calculation using annual, global-mean top-of-atmosphere flux anomalies 
(Gregory et al., 2004).  From the atmosphere only experiments (amip-p4K, amipFuture, and aqua-p4K), the feedbacks 
are calculated as 
differences in top-of-atmosphere flux anomalies between the perturbation and control experiments.   

\begin{figure}
% figure generated with cloud_table_AM4fig.ncl
  \includegraphics[width=0.86\columnwidth]{cfmipfigs/HML_CF_cfmip.eps}
  \caption{Cloud fraction from several of the CMIP6 experiments.}
  \label{fig:cf_hml_manyexps}
\end{figure}

\begin{figure}
% figure generated with cloud_table_AM4fig.ncl
  \includegraphics[width=0.86\columnwidth]{cfmipfigs/HML_CF_cfmip_bias.eps}
  \caption{Cloud fraction bias relative to the AM4.0 amip experiment.}
  \label{fig:cf_hml_bias}
\end{figure}


Plotting the cloud fields as a function of latitude and height highlights some of the similarities and differences 
between the various experiments.  In general the clouds in the $amip$ and historical simulations are similar to 
each other, as we would expect.  The differences between the clouds of these experiments are due to the 
different SST pattern that the two experiments have.  We can thus see how the clouds change as a result 
of SST biases over the observational period.  Two warming experiments have been plotted, amip p4K and 
the coupled abrupt 4xCO2 experiment.  The cloud fields in these experiments are similar to each other, with 
the differences again likely coming from the different SST patterns that result from a fully coupled system.  The 
cooling experiment is rather unique and tends to have cloud fraction changes in the opposite direction as 
the warming experiments.    

%% Beta feedback??  relavent?
%Brient and Bony (2012) proposed what they call the $\beta$ feedback as a way to quantify whether or not ' models have 
%changes in low-cloud cloud radiative effects which are proportional to the strength of the cloud radiative effects of their
%low clouds.'  See $\it{Stevens etal, Cookie Overview}$ for details.   That paper asks the question of whether single 
%model relationships between low-cloud amounts and the sensitivity of low clouds to warming can be 
%masked by changes in multi-model ensembles.  This should be checked for the GFDL model in this paper.  
%
%The cloud radiative effect (CRE) is computed as the difference between the all-sky and clear-sky TOA fluxes.  

\begin{table}
\begin{center}
\caption{Global mean radiative feedbacks ($\rm{W\, m^{-1} K^{-1}}$).  The cloud radiative effect (CRE) is computed as
 the clear sky minus all sky flux. The values for the Abrupt 4xCO2 experiment
have been estimated from a linear fit of the years 51-300.  Compare to Ringer et al. 2014.
Values in parenthesis are most recent values.  Why are they different than the Amip p4K 
original values?}
    \begin{tabular}{*{7}{c}}
    \hline
    \hline
 Feedback & Aqua p4K & Amip m4K & Amip p4K &  Amip p8K & Amip Future & Abrupt 4xCO2    \\ \hline
    Net          &   -2.2 (-2.2) & ?? (-1.9)  &  -1.6 (-1.8)      & ?? (-1.9)      & -1.8 (-2.1)    &    -0.46 (-0.84; Tim)         \\ 
    \\
    Net CRE      & 0.3 (0.3)  & ?? (0.0)  & -0.1(-0.1)    & ?? (-0.0)         & 0.1 (0.1)          & 0.65   \\  
    \\
    LW CRE       & 0.4 (0.4)  & ?? (0.2)  & 0.1 (0.2)  & ?? (0.2)        & 0.1 (0.1)          & 0.14    \\  
    \\
    SW CRE      & -0.7 (-0.7) & ?? (-0.2) & -0.1 (-0.1)  & ?? (-0.1)      & -0.2 (-0.2)         & -0.5              \\  
    \\
    LW CLR       & 2.0 (2.0)  & ?? (2.3) & 2.0 (2.3)   & ??  (2.3)       & 2.0 (2.4)         & 1.7             \\  
    \\
    SW CLR      & -0.1 (-0.1)  & ?? (-0.4) & -0.3 (-0.4)    & ?? (-0.4)        & -0.3 (-0.4)          & -0.6                   \\  \hline

    \end{tabular}\par
    %\bigskip 
    \label{tab:lambda}
\end{center}
\end{table}

% SST patterns, the pattern effect.  Should my discussion paragraph of the pattern effect be 
% the same paragraph/discussion as when I discuss Cess experiments?  
The Cess sensitivity for AM4.0 ($0.57 \, \rm{K\, W^{-1} m^2}$, Z18b, pg 26) is closer to the Cess sensitivity 
% see page 39 of Zhao et al. 2018 part 1 for discussion of Cess sensitivity.  
of AM2.1 ($0.54 \, \rm{K\, W^{-1} m^2}$) than AM3.  This is partly due to the method of converting convective cloud condensate to precipitation 
in the convection scheme.   Z18b on the same page describes a Cess sensitivity with AM4.0 when the drop number
is fixed of ($0.52 \, \rm{K\, W^{-1} m^2}$).  As discussed in Z18b and Winton et al. 2019, despite a Cess sensitivity similar to AM2.1, 
the TCR and ECS of CM4.0 are larger than those of CM2.1, and close to those of CM3.   This clearly 
shows that using the Cess sensitivity to infer the TCR or ECS is not reliable (does not work).   Is the ratio of TCR to
Cess sensitivity useful?  

\begin{figure}
% figure generated with cloud_table_AM4fig.ncl
  \includegraphics[width=0.86\columnwidth]{cfmipfigs/amip_p4k_future_cldfrac_calip.eps}
  \caption{Cloud fraction for amip related experiments.  Observations from CALIPSO are shown for reference.}
  \label{fig:calipso_amip_comparison}
\end{figure}

\begin{figure}
  \includegraphics[width=0.86\columnwidth]{cfmipfigs/cf_warming.eps}
  \caption{Cloud fraction for experiments with warming.  For reference cloud fraction from amip is also shown.}
\end{figure}

\begin{figure}
  \includegraphics[width=0.86\columnwidth]{cfmipfigs/cf_warming_amipbias.eps}
  \caption{Cloud fraction bias for experiments with warming.  The bias is computed relative to the amip experiment. 
  For the low-level clouds note that the coupled experiments have CF maximum (5-6\%) at -20 annd -75 where the prescribed uniform warming
  does not.  The coupled model experiments produce much less CF in the latitudes between 30-70.  All warming experiments have more 
  arctic low-level clouds than the amip experiment.  Keep in mind the amip experiment has too many arctic low-level clouds based
  on the CALIPSO data.}
  \label{fig:calipso_amip_comparison}
\end{figure}

The previous section showed that the AM4.0 amip simulations underestimates clouds relative to the CALIPSO observations.  
Simulations over the historical period with the fully coupled model do not improve on this bias, and in the case of the abrupt
4xCO2 and gradual 1\% CO2 experiments the bias is worse at almost all latitudes except at 20 degrees south.  The two coupled
warming experiments (abrupt and ramp) have some interesting differences from both the \textit{amip} and the
 \textit{historical} simulations.   These warming experiments have a distinct peak (about 5\%) of low-level cloud fraction at about 20N that 
 both \textit{amip} and \textit{historical} lack (\textit{historical} does have a small but less distinct peak).  Additionally, 
 while the low-level clouds of the \textit{amip} and \textit{historical} experiments are almost identical between 10-60N, the 
 \textit{abrupt} and \textit{ramp} experiments show a large decrease of cloud fraction between about 30-70N.  Is this decrease
 of low-level cloud in the NH responsible for a stronger sw cloud feedback relative to the amip p4K and historical simulations?  Of the 
 two experiments, the abrupt has the least amount of low-level cloud.  Does it also then have the strongest/largest SW CRE?  
 When comparing these experiments to the CERES0-EBAF observations from the observational period the \textit{abrupt} and 
 \textit{ramp} experiments match the CERES-EBAF CRE better than the amip or historical experiments.  All of the AM4.0 
 simulations shown here compare best with the CERES-EBAF v2.8.  This is likely because this is the version of the data that 
 was used as a reference during the model development.  For comparison we also show the CERES-EBAF v4.1 data.  Although 
 differences between the two versions are clear, the AM4.0 models compare relatively well to both.  The differences between 
 the two observational version are mostly contained to the latitudes between +/- 30.  This excellent agreement with observations 
 of the CRE in the NH for the warming experiments was surprising, but does not in general indicate better observational 
 agreement of the warming experiments relative to the \textit{historical} and \textit{amip} experiments.   These warming 
 experiments have a very strong arctic and southern ocean response that differs significantly from the historical observations 
 and simulations.   As these are signature CRE changes in a warming climate it is not surprising that they are not 
 present in the historical data.  

\begin{figure}
  \includegraphics[width=0.86\columnwidth]{cfmipfigs/meridional_CRE_CERES2p8.eps}
  \caption{.}
  \label{fig:CRE_multiexperiments_ceres}
\end{figure}

\section{What are we learning?}

How much do clouds contribute to our uncertainty in the climate sensitivity?  
Which clouds (high, mid, low, thick, thin, geographic regions) contribute to this uncertainty?
Why do we do a poor job with these clouds in models?  

Experiments using Cess-like uniform SST warming as a proxy for climate change have successfully predicted that much of the differences 
among models is due to the cloud radiative affect.  However, we also know that the sensitivity derived from Cess-like experiments is not
a good predictor of the climate sensitivity for a parent, coupled model.  Comparison of Cess-sensitivities and coupled model sensitivities 
combined with patterned prescribed SST experiments using AGCM have demonstrated the significant role of a pattern-effect in determining 
the sensitivity of a model.  

We also know that a method which uses linear regression of a constant forcing experiment to estimate the climate sensitivity only 
provides an accurate measure of the climate sensitivity when the initial 20-50 years of simulation are excluded from the analysis.  
The climate sensitivities of models that have been run to equilibrium with millenial-scale simulations show a large discrepency in the
different time periods used to estimate this sensitivity.  

Clouds in a climate model are highly maleable while also being a dominant factor controlling what the climate sensitivity is for that model.  
Yet we do not have satisfactory clouds constraints to impose on the clouds of a climate model.  If improving the representation of low-level
clouds in the GFDL model was the first priority, it could be done.  However it is not the first priority and doing so degrades other elements of 
the climate.  

One of our goals was to highlight and learn from the influence of clouds feedbacks across a range of GFDL based models.  Our hope 
is that by looking at models from a single institution rather than a larger ensemble of CMIP models we will avoid inadvertantly 
masking important relationships by a multi-model ensemble mean.  

\end{document}
