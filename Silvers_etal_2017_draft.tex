%%%%%%%%%%%%%%%%%%%%%%%%%%%%%%%%%%%%%%%%%%%%%%%%%%%%%%%%%%%%%%%%%%%%%%%%%%%%
% AGUtmpl.tex: this template file is for articles formatted with LaTeX2e,
% Modified July 2014
%
% This template includes commands and instructions
% given in the order necessary to produce a final output that will
% satisfy AGU requirements.
%
% PLEASE DO NOT USE YOUR OWN MACROS
% DO NOT USE \newcommand, \renewcommand, or \def.
%
% FOR FIGURES, DO NOT USE \psfrag or \subfigure.
%
%%%%%%%%%%%%%%%%%%%%%%%%%%%%%%%%%%%%%%%%%%%%%%%%%%%%%%%%%%%%%%%%%%%%%%%%%%%%
%
% All questions should be e-mailed to latex@agu.org.
%
%%%%%%%%%%%%%%%%%%%%%%%%%%%%%%%%%%%%%%%%%%%%%%%%%%%%%%%%%%%%%%%%%%%%%%%%%%%%
%
% Step 1: Set the \documentclass
%
% There are two options for article format: two column (default)
% and draft.
%
% PLEASE USE THE DRAFT OPTION TO SUBMIT YOUR PAPERS.
% The draft option produces double spaced output.
%
% Choose the journal abbreviation for the journal you are
% submitting to:

% jgrga JOURNAL OF GEOPHYSICAL RESEARCH
% gbc   GLOBAL BIOCHEMICAL CYCLES
% grl   GEOPHYSICAL RESEARCH LETTERS
% pal   PALEOCEANOGRAPHY
% ras   RADIO SCIENCE
% rog   REVIEWS OF GEOPHYSICS
% tec   TECTONICS
% wrr   WATER RESOURCES RESEARCH
% gc    GEOCHEMISTRY, GEOPHYSICS, GEOSYSTEMS
% sw    SPACE WEATHER
% ms    JAMES
% ef    EARTH'S FUTURE
% ea    EARTH AND SPACE SCIENCE
%
%
%
% (If you are submitting to a journal other than jgrga,
% substitute the initials of the journal for "jgrga" below.)

%\documentclass[draft,grl]{agutex2015}
\documentclass[grl]{agutex2015}
% To create numbered lines:

% If you don't already have lineno.sty, you can download it from
% http://www.ctan.org/tex-archive/macros/latex/contrib/ednotes/
% (or search the internet for lineno.sty ctan), available at TeX Archive Network (CTAN).
% Take care that you always use the latest version.

% To activate the commands, uncomment \usepackage{lineno}
% and \linenumbers*[1]command, below:

\usepackage{graphicx}
\usepackage{epstopdf}

% \usepackage{lineno}
% \linenumbers*[1]
%  To add line numbers to lines with equations:
%  \begin{linenomath*}
%  \begin{equation}
%  \end{equation}
%  \end{linenomath*}
%%%%%%%%%%%%%%%%%%%%%%%%%%%%%%%%%%%%%%%%%%%%%%%%%%%%%%%%%%%%%%%%%%%%%%%%%
% Figures and Tables
%
%
% DO NOT USE \psfrag or \subfigure commands.
%
%
%  Uncomment the following command to include .eps files
%  (comment out this line for draft format):
%  \usepackage[dvips]{graphicx}
%
%  Uncomment the following command to allow illustrations to print
%   when using Draft:
  \setkeys{Gin}{draft=false}
%
% Substitute one of the following for [dvips] above
% if you are using a different driver program and want to
% proof your illustrations on your machine:
%
% [xdvi], [dvipdf], [dvipsone], [dviwindo], [emtex], [dviwin],
% [pctexps],  [pctexwin],  [pctexhp],  [pctex32], [truetex], [tcidvi],
% [oztex], [textures]
%
% See how to enter figures and tables at the end of the article, after
% references.
%
%% ------------------------------------------------------------------------ %%
%
%  ENTER PREAMBLE
%
%% ------------------------------------------------------------------------ %%

% Author names in capital letters:
\authorrunninghead{SILVERS ET AL.}

% Shorter version of title entered in capital letters:
\titlerunninghead{Cloud Feedbacks in long-AMIP}

%Corresponding author mailing address and e-mail address:
%\authoraddr{Corresponding author: A. B. Smith,
%Department of Hydrology and Water Resources, University of
%Arizona, Harshbarger Building 11, Tucson, AZ 85721, USA.
%(a.b.smith@hwr.arizona.edu)}

\begin{document}

%% ------------------------------------------------------------------------ %%
%
%  TITLE
%
%% ------------------------------------------------------------------------ %%


\title{Decadal variations of cloud feedback in long-AMIP experiments with three GFDL climate models.}
%
% e.g., \title{Terrestrial ring current:
% Origin, formation, and decay $\alpha\beta\Gamma\Delta$}
%

%% ------------------------------------------------------------------------ %%
%
%  AUTHORS AND AFFILIATIONS
%
%% ------------------------------------------------------------------------ %%


%Use \author{\altaffilmark{}} and \altaffiltext{}

% \altaffilmark will produce footnote;
% matching \altaffiltext will appear at bottom of page.
\authors{Levi G. Silvers, David Paynter, and Ming Zhao}
% \authors{A. B. Smith,\altaffilmark{1}
% Eric Brown,\altaffilmark{1,2} Rick Williams,\altaffilmark{3}
% John B. McDougall\altaffilmark{4}, and S. Visconti\altaffilmark{5}}

%\altaffiltext{1}{Department of Hydrology and Water Resources,
%University of Arizona, Tucson, Arizona, USA.}

%\altaffiltext{2}{Department of Geography, Ohio State University,
%Columbus, Ohio, USA.}

%\altaffiltext{3}{Department of Space Sciences, University of
%Michigan, Ann Arbor, Michigan, USA.}

%\altaffiltext{4}{Division of Hydrologic Sciences, Desert Research
%Institute, Reno, Nevada, USA.}

%\altaffiltext{5}{Dipartimento di Idraulica, Trasporti ed
%Infrastrutture Civili, Politecnico di Torino, Turin, Italy.}

%% ------------------------------------------------------------------------ %%
%
%  KEYPOINTS
%
%% ------------------------------------------------------------------------ %%

% Key points are 1 to 3 points that the author provides,
% that are 100 characters or less, that are ultimately published
% with the article.
%% for example:
% \keypoints{\item Here is the first keypoint. what happens if it is a
% long keypoint, like this one. We want to see this wrap please.
% \item This is the second.
% \item And here is the third keypoint
% }


\keypoints{\item Three models show the Eastern Tropical Pacific as a key location where decadal trends in SST and low-level clouds strongly influence the global energy budget
\item Changes in low-level clouds can be linearly decomposed but with differing magnitudes of dependencies on EIS and SST
\item The cloudy response of the models differs markedly between twentieth century periods with high and low climate sensitivities}
%\item AM2.1, AM3, and AM4g10r8 all show strong twentieth
%century decadal variability of the global feedback parameter.
%\item All three models show increased low-level clouds in the East Pacific
%over recent decades.
%\item Low-level cloud trends depend on both EIS trends 
%and omega trends.}

%% Keypoints will print underneath the abstract.


%% ------------------------------------------------------------------------ %%
%
%  ABSTRACT
%
%% ------------------------------------------------------------------------ %%

% >> Do NOT include any \begin...\end commands within
% >> the body of the abstract.

\begin{abstract}

Significant decadal variability of the global feedback parameter is shown to be clearly linked to the pattern of surface warming through the evolution of low-clouds and the estimated inversion strength (EIS).  We examine the atmospheric response to changing SST patterns between 1870-2004 using the amip-piForcing experimental configuration with three GFDL atmospheric global climate models (AM2, AM3, and AM4).  There is a notable strengthening of the negative global feedback from the middle period of the 20th century until the present.  We contrast the variability of the cloud response among the models between these two periods.  The changing low-cloud cover is analyzed within a framework that includes trends of the EIS and the mid-tropospheric vertical velocity.  

Recently it has been argued that decadal changes in low-level cloud cover over the eastern Pacific ocean basin strongly influence the global net feedback parameter.  These decadal changes of low-level cloud cover have been tied to corresponding changes in the observed sea surface temperatures.  We examine these claims with AM2, AM3, and AM4.  The dominant configuration difference among the models is the parameterization of the convective clouds.  The amip-piForcing experiment provides an ideal configuration to test the range of atmospheric responses to a prescribed but evolving SST pattern.

All three GFDL models confirm the importance of the relationship between the global climate sensitivity and the trends in the eastern Pacific of SST and low-level clouds.  However, there are noteworthy differences in the cloud response of the models.  Despite all three models being driven by observed SSTs, sea ice concentration, and constant preindustrial forcing agents distinct net global feedbacks are simulated.  


%Recently it has been argued that decadal changes in low-level cloud cover over the eastern Pacific ocean basin strongly influence the global net feedback parameter.  These decadal changes of low-level cloud cover have been tied to corresponding changes in the observed sea surface temperatures (SST).  We examine this claim in three generations of GFDL atmospheric global climate models $(AM2.1, AM3, and AM4_g10r8)$.  The dominant configuration difference among the models is the parameterization of convective clouds.  Atmospheric experiments with observed SSTs (atmospheric forcing held at constant preindustrial values) between 1870-2004 are examined to elucidate the source of differences between the models.  
%
%All three models are shown to have strong decadal variability in their net feedback parameter, the global mean cloud-radiative effect, and the low-level cloud fields.  All three models confirm the importance of the relationship in the eastern Pacific region between decadal SST patterns and low-level cloud changes.  However, there are noteworthy differences in the cloud response of the models.  
%
%The correlation between changes in low-level cloud cover and the lower tropospheric stability is used to examine the impact of the alternate cumulus parameterizations used by the models.  We approximate the change of low-level cloud with a linear combination of changes in lower tropospheric stability and SST.  The usefulness of this approximation differs among the three models examined.   
%
%Despite all models being driven by observed SSTs, sea ice concentration, and having constant preindustrial atmospheric forcing agents distinct net global feedbacks are simulated. 
%
\end{abstract}

%% ------------------------------------------------------------------------ %%
%
%  BEGIN ARTICLE
%
%% ------------------------------------------------------------------------ %%

% The body of the article must start with a \begin{article} command
%
% \end{article} must follow the references section, before the figures
%  and tables.

\begin{article}

%% ------------------------------------------------------------------------ %%
%
%  TEXT
%
%% ------------------------------------------------------------------------ %%

\section{Introduction}

The global energy budget of Earth is often considered in a mean, equilibrated context.  However, fluctuations in the TOA net radiative flux are observed on monthly, yearly, and decadal scales.  The decadal variability of the energy budget is also present in the global mean cloud radiative effect.  This paper elucidates the nature of changes to the cloud fields in simulations covering 1871-2005.  By clarifying key geographical regions and particular types of clouds that play a prominent role in the decadal variability of the TOA net radiative flux we hope to increase our confidence in the equilibrated energy budget analysis of Earth.    

    \begin{enumerate}
        \item Decadal variations of global feedback in AMIP-like experiments offer an opportunity for 
        mechanistic attribution
        	\begin{enumerate}
	    \item SST patterns have been shown to be important for variations of global 
	    feedback (Gregory and Andrews, 2016).  
	    \item Estimated Inversion Strength (EIS)
	    
	    The lower-tropospheric stability, and the related estimated inversion strength (EIS) has been used as a predictor of low-level clouds in select stratocumulus dominated regions (Klein and Hartmann, 1993, Wood and Hartmann, 2006).  The tropical marine EIS was found by Zhou et al. 2016 to be well correlated with the evolution of tropical tropical low-cloud change.  The EIS can be approximated by a linear combination of the surface temperature and the temperature on the 700hPa pressure level.  In the tropics, this provides a glimpse of one influence of the convection parameterization because the mid-tropospheric temperature is largely determined by the lapse rate in the warm pool regions with strong deep convection.
	    
	    \item Analyzing alternate atmospheric climate models with the same SST 
	    patterns and atmospheric forcing throughout the twentieth century highlights
	    both atmospheric internal variability and the impact of different parameterization 
	    configurations among the models.
	    \item Why AM2.1, AM3, and AM4g10r8?
	\end{enumerate}
	\item Relationship between decadal feedback, equilibrium climate sensitivity, and effective climate sensitivity
	\item Review of Gregory and Andrews 2016, Zhou et al. 2016, Qu et al. 2014,2015, and Paytner et al. 2017
    \end{enumerate}

\section{Method}
%    \begin{enumerate}
 %       \item Description of amipPiForcing experiments
\subsection{Description of amipPiForcing experiments}  
        
        The primary experiment used in this work corresponds to the amip-piForcing experiment.  For this experiment the AGCMs are run using observed SST and sea-ice, but with the forcing due to GHG, aerosol, and solar forcing set to assumed pre-industrial levels and held constant throughout the experiment.  The time simulated is from 1870 to 2005.  The amip-piForcing experiments are sometimes referred to as a `long-AMIP' experiments.  The amip-piForcing experiment has been included as one of the tier-II experiments for CFMIP3 (Webb et al. 2017), for additional details and examples see Gregory and Andrews (2016) and Zhou et al. (2016). 
        
\subsection{Description of AM2.1, AM3, and AM4g10r8}
        
        The three GFDL atmospheric models analyzed in this paper have many differences, but the those that are thought to be of primary importance to the top-of-the-atmosphere radiation budget are relatively few.  The AM3 and AM4g10r8 models include an aerosol indirect effect by including the cloud drop number as a prognostic variable.  The AM2.1 model diagnosed the cloud drop number.  The parameterization schemes used to represent cumulus convection differs significantly among the three models.  The AM2.1 model uses the Relaxed-Arakawa-Schubert (RAS) scheme, AM3 uses the Donne-Deep scheme for deep convection and the University of Washington shallow convection scheme (UWShCu, Bretherton et al. 2004) for shallow convection.  AM4g10r8 uses a version of the UWShCu scheme that has been modified to include two bulk plumes which can account for both shallow and deep convection.  In addition to these differences, the scheme which computes the radiative transfer was updated from SEALW99/ESFSW99 in AM2.1 and AM3 to SEALW16/ESFSW16 in AM4g10r8.    
        
\subsection{Methodology of Analysis (here or in results?)}        
%        \begin{enumerate}
%            \item Global feedback parameter via Gregory regression sliding window
            Because our interest is in the covariation of R and T rather than the change relative to a control state we compute the differential climate feedback parameter in the same was as in Gregory and Andrews [2016].  That is, linear regression is used to compute the trend between R and T.  The regression is applied to the global yearly mean values over 30 windows.  The values plotted in Figure 1 represent the mid-point of each window.  
 %           \item Global maps and global mean time series of trends per years
            To determine the spatial pattern of changes to particular variables linear regression was used at each grid point.              
 %           \item Estimated Inversion Strength
            
            The estimated inversion strength can be thought of as a correction to the LTS based on the moist adiabatic profile.  The LTS is defined to be $\rm{LTS}=\theta_{700}-\theta_{sfc}$.  
            To compute the estimated inversion strength (EIS) we follow the approximation made by Wood and Bretherton [2006]: 
            \begin{equation}
              \rm{EIS} = \rm{LTS} - \Gamma^{850}_m (z_{700}-\rm{LCL})
            \end{equation}
            with the lifting condensation level (LCL) calculated as $\rm{LCL}=(20+(T_{sfc}-273.15)/5)(1-RH_{sfc})$.
            The potential temperature gradient for moist-adiabatic processes is calculated as
            \begin{equation}
              \Gamma_m(T,p)=\frac{g}{c_p} \left[ 1-\frac{1+L_v q_s(T,p)/R_a T}{1+L^2_v q_s(T,p)/c_p R_v T^2} \right].
            \end{equation}
            To compute $\Gamma^{850}_m$ we used the temperature on the $850 \rm{hPa}$ pressure level and computed the saturation vapor pressure on the 
            $850 \rm{hPa}$ level using $q_s=610.8 \, \rm{exp}[17.27(T_{850}-273.15)/(T_{850}-35.85)] (\rm{hPa})$.  When the pressure surfaces of interest for these calculations intersect topography 
            the variables we are interested in are not defined.  To account for this, missing vallues of $\rm{T}_{850}$ are set to 50 $K$, $\rm{T}_{sfc}$ is not allowed to be lower than $200$ K, and $\rm{RH}_{sfc}$
            is not allowed to be lower than $5 \%$.            
     
%    \end{enumerate}

\section{Results}

%\begin{enumerate}
 %   \item General Results
\subsection{General Results}
    
We have shown that all three models exhibit strong decadal variability of the feedback parameter.  The close connection between the global feedback and the CREs is clear based on their respective time series.  Correlations between the global feedback parameter and CRE for each model are: 0.93, 0.93, and 0.81 for AM2.1, AM3, and AM4g10r8, respectively.

Over the second half of the twentieth century AM4g10r8 and AM3 show a strong decrease of the global feedback parameter and the CRE response (Fig. \ref{fig:alphacre}).  Both of these models also have an increasing trend of EIS during this time period (Fig. \ref{fig:alphacre}).  In contrast, AM2.1 has a weak decrease of global feedback parameter and a nearly constant CRE trend during this time period.  The trend of EIS for AM2.1 is relatively flat and less than that for AM4g10r8 and AM3.  
%If the change of LLC can be linearly decomposed into a component due to the EIS and one due to the change of SST I would then expect this to indicate a poorer correlation for AM2.1 compared to AM3 in the tropics.  
Between 1925 and 1955 the average global feedback parameter for AM4, AM3, and AM2.1 changes from average values of (-1.35, -1.02, -1.80) to (-2.22, -2.42, -2.20) between 1975 and 1989.  This represents a dramatic change in the global feedback parameter and determining the cause will be an important part of our ability to understand how clouds influence the global energy budget on decadal time scales. 

%Changes in low-cloud cover where approximated by Zhou et al. 2016 as $\Delta \rm{LCC} = \delta \Delta \rm{EIS} - \gamma \Delta \rm{SST}$.  If this holds then to the extent that different models use the same SST fields, the change of LCC should follow the change of EIS.  
%The importance of different SST patterns with the same mean value has been recognized (Andrews et al. 2015).  Is there a similar importance of the EIS pattern?  In this context the hypothesis (Qu et al. 2014,2015) that changes of EIS can be approximated as 
%$\Delta \rm{EIS} \approx \alpha \Delta T_{700}-\beta \Delta T_s $ indicates that the pattern of EIS change is entirely (for the case of the same SST) determined by the change of temperature on the 700 hPa level.  This is an easier quantity to diagnose from GCMs than EIS.

%All three models show an increase of the anomalous tropical mean low-level clouds and EIS between about 1975 and 2005, with the increase in AM2.1 covering the smallest period of time.   
%The tropical marine low-cloud changes in AM4g10r8 are well explained (r=0.96) by a linear combination of the changes in the EIS and the SST.  
%To the extent that the MAC LWP from Norris et al. 2016 represents a realistic trend of LWP the AM4g10r8 LWP trend shows a more realistic trend than AM2.1.  
%Because of the large amount of the anomalous change in low-cloud cover that can be explained by the linear combination of EIS and SST, as in CAM5.3, it is tempting to conclude that AM4g10r8 is doing the best job at reproducing observed relationships between EIS and low-cloud cover.  
%However, AM4 needs to be tested for its representation of the cloud controlling factors which have been discussed by Bretherton, and are shown in Figure 1 of the supplementary material of Zhou et al. 2016.

The period after about 1975 is found by all three models to have a small climate sensitivity (large, negative feedback parameter, Fig. \ref{fig:alphacre}) and a negative CRE.  This period also corresponds in all three models to an increase of tropical mean low-level clouds  (Fig. \ref{fig:lccts}).  Computing the trend of LCC change between 1975-2005 reveals much of this change to be concentrated in the eastern Pacific ocean basins (Fig. \ref{fig:lcceis_late}) in agreement with the findings of Zhou et al. 2016.  Further analysis of the changes to low-level clouds is discussed in the next section.    

A strong relationship between LCC and EIS in many regions of the globe is shown by the trends of LCC and EIS between 1975-2005 (Figure \ref{fig:lcceis_late}) and between 1925-1955 (Figure \ref{fig:lcceis_early}).  Both time periods have a large increase of both EIS and LCC in the equatorial eastern Pacific ocean.  However, during the later period there is an increase of LCC throughout much of the Southern Hemispheric oceans while the early period the LCC decreases in much of the Southern Hemispheric oceans and northern midlatitudes.  This is consistent with a positive cloud feedback/large climate sensitivity during the early period and a small cloud feedback/climate sensitivity during the later period (Fig. \ref{fig:alphacre}).

%    \item Multiple Linear Regression 
\subsection{Multiple Linear Regression}    

    The anomalous changes of tropical marine low-cloud cover were shown to be well explained by a linear combination of the change in EIS and SST in Zhou et al. 2016.  We explore the linear dependence of low-cloud changes on EIS and SST using multiple linear regression to determine the relative contributions of each component.  The coefficients $\gamma$ and $\delta$ can thus be determined for particular experiments or models in the following assumed relationship: 
    \begin{equation} 
    \Delta \rm{LCC} \approx \gamma \Delta \rm{EIS} - \delta \Delta \rm{SST}.
    \label{eqLCC}
    \end{equation}  
Based on high correlations between changes in low-cloud cover and the linear combination of changes in EIS and SST the above relationship holds fairly well.  However, we find that the relative contributions of EIS and SST to the low-cloud cover is model dependent.  Because the three models examined here all have observationally based SST prescribed, this difference likely indicates differences in the relative importance of the EIS fields which are in turn driven by the temperature on the $700 \rm{hPa}$ surface.  The importance of the low-level and mid-tropospheric temperature field is not surprising.  The direct connection between the temperature field at particular levels with parameterization schemes (convection, turbulence) provides an encouraging link between details of the parameterizations and the low-level cloud field.  
    
Several noteworthy points can be drawn from Table 1.  As shown by the correlation coefficients, the decomposition used by Zhou et al. 2016 to explain the variance in the change of the low-cloud cover is not directly applicable to other models.  Using multiple linear regression to approximate the best fit for each model explains a significantly higher proportion of the variance in the changing low-cloud cover.  Also interesting are the varying coefficients computed for Eq. ref{eqLCC}.  This indicates that each of the models has a different relative sensitivity of low-clouds to EIS and SST.   

Several noteworthy points can be drawn from Table 1.  As shown by the correlation coefficients, the decomposition used by Zhou et al. 2016 to explain the variance in the change of the low-cloud cover is not directly applicable to other models.  Using multiple linear regression to approximate the best fit for each model explains a significantly higher proportion of the variance in the changing low-cloud cover.  Also interesting are the varying coefficients computed for Eq. ref{eqLCC}.  This indicates that each of the models has a different relative sensitivity of low-clouds to EIS and SST.   

    
We find some agreement with previously published work, but also some interesting differences.  Our results, as well as those of Grise and Medeiros, 2016, and Paytner et al. 2017 indicate that low-level clouds, and thus the TOA net CRE, strongly depend on both EIS and the large-scale vertical velocity, omega.     

\subsection{Linear approximation of the Estimated Inversion Strength}

Changes in the EIS where shown to be well approximated as a linear combination of the surface temperature and the temperature on the $700 \rm{hPa}$ pressure level by Qu et al. 2014.  Specifically,
\begin{equation}
  \Delta \rm{EIS} \approx \alpha \Delta T_{700}-\beta \Delta T_{sfc},
  \label{eq:eis}
\end{equation}
where $T_{sfc}$ and $T_{700}$ are the temperatures on the surface and the $700 \rm{hPa}$ pressure surface, respectively.
This provides an elegant simplification of the EIS and shows its close relation to the lower troposheric stability (different of potential temperature between 700 hPa and the surface).  In the atmosphere the relative influence of the $T_{sfc}$ and $T_{700}$ on the EIS varies depending on location.  One of the strengths of long-AMIP experiments is the use of observed SST so modeled differences in the marine EIS fields will be largely due to modeled differences in $T_{700}$.  Approximations of EIS using eq. \ref{eq:eis} are compared with the simulated change of EIS in Figures \ref{fig:lcceisapp_am4}-\ref{fig:lcceisapp_am2}.  Overall the approximation captures the spatial patterns of changes in the EIS field, but the magnitude is overestimated.  

%\end{enumerate}

\section{Conclusions}

The eastern Pacific has been shown to be a key location where decadal trends in SST and low-level clouds can strongly influence the global energy budget.  

These changes in low-level clouds were shown to be well modeled by a linear combination of anomalous EIS and SST.  However, the coefficients of the decomposition vary among the models.  This implies that the models do not weight the influence of EIS and SST in the same way.  The response of low-level clouds thus depends to differing degrees on the changes in the 700 hPa temperature.  This provides a relatively simple connection between differing model parameterizations and the response of the clouds to changing SST patterns.  

We have also shown that the although the three models respond similarly to the SST patterns between 1975-2005, their response to the SST patterns throughout the twentieth century differs dramatically.  The contrast in response between the period of 1975-2005 and 1925-1955 is emphasized.  

Using the observed sea surface temperature and ice concentrations as the lower boundary for the atmosphere eliminates the feedback processes between the atmosphere, land, and ocean in nature.  However, using these observations ensures that the atmospheric models are driven by the the same patterns and magnitudes of warming.  This provides an excellent test of how much freedom of response the atmosphere alone is capable of given similar boundary conditions.  

%%% End of body of article:

%%%%%%%%%%%%%%%%%%%%%%%%%%%%%%%%
%% Optional Appendix goes here
%
% \appendix resets counters and redefines section heads
% but doesn't print anything.
% After typing \appendix
%
\appendix
%\section{Here Is Appendix Title}
% will show
% Appendix A: Here Is Appendix Title
%
\section{A few of Levi's questions}
\begin{enumerate}

\item The differential climate feedback parameter in Fig. 1 shows a relatively smooth transition from the small negative values during the period 1925-1955 and the larger negative values between 1975-2005.  Why does the time series of tropical change in LCC in Fig. 6 look so different?     It looks almost like there is a 30 year oscillation in the tropical LCC.  

\item How much do the differences in the surface temperature data matter?  How much do differences over land matter?  

\item Should we incorporate the vertical velocity into the paper?  

\item What is the relationship between these results and the 5 typical strato-cumulus regions that are often studied? 

\item According to Xu et al. 2014, GFDL-CM2.0, GFDL-CM2.1, GFDL-CM3, GFDL-ESM2G, and GFDL-ESM2M (H,I,f,g,h) all use cloud-top radiative cooling to drive boundary layer turbulence.  Is this also the case for AM4?  They list all three as having a 'K profile' PBL scheme type and a Prognostic cloud scheme type.  

\end{enumerate}
This study does not address the known spread of LCC change due to differing sensitivities of models to changing SST.  Because all three of our models are driven with observed SST the different responses are due to differences of the atmospheric response to the boundary conditions (to what extent is the large-scale circulation tied to the SST?).  

%%%%%%%%%%%%%%%%%%%%%%%%%%%%%%%%%%%%%%%%%%%%%%%%%%%%%%%%%%%%%%%%
%
% Optional Glossary or Notation section, goes here
%
%%%%%%%%%%%%%%
% Glossary is only allowed in Reviews of Geophysics
% \section*{Glossary}
% \paragraph{Term}
% Term Definition here
%
%%%%%%%%%%%%%%
% Notation -- End each entry with a period.
% \begin{notation}
% Term & definition.\\
% Second term & second definition.\\
% \end{notation}
%%%%%%%%%%%%%%%%%%%%%%%%%%%%%%%%%%%%%%%%%%%%%%%%%%%%%%%%%%%%%%%%
%
%  ACKNOWLEDGMENTS

%\begin{acknowledgments}
%(Text here)
%\end{acknowledgments}

%% ------------------------------------------------------------------------ %%
%%  REFERENCE LIST AND TEXT CITATIONS
%
% Either type in your references using
% \begin{thebibliography}{}
% \bibitem{}
% Text
% \end{thebibliography}
%
% Or,
%
% If you use BiBTeX for your references, please use the agufull08.bst file (available at % ftp://ftp.agu.org/journals/latex/journals/Manuscript-Preparation/) to produce your .bbl
% file and copy the contents into your paper here.
%
% Follow these steps:
% 1. Run LaTeX on your LaTeX file.
%
% 2. Make sure the bibliography style appears as \bibliographystyle{agufull08}. Run BiBTeX on your LaTeX
% file.
%
% 3. Open the new .bbl file containing the reference list and
%   copy all the contents into your LaTeX file here.
%
% 4. Comment out the old \bibliographystyle and \bibliography commands.
%
% 5. Run LaTeX on your new file before submitting.
%
% AGU does not want a .bib or a .bbl file. Please copy in the contents of your .bbl file here.

%\begin{thebibliography}{}

%\providecommand{\natexlab}[1]{#1}
%\expandafter\ifx\csname urlstyle\endcsname\relax
%  \providecommand{\doi}[1]{doi:\discretionary{}{}{}#1}\else
%  \providecommand{\doi}{doi:\discretionary{}{}{}\begingroup
%  \urlstyle{rm}\Url}\fi
%
%\bibitem[{\textit{Atkinson and Sloan}(1991)}]{AtkinsonSloan}
%Atkinson, K., and I.~Sloan (1991), The numerical solution of first-kind
%  logarithmic-kernel integral equations on smooth open arcs, \textit{Math.
%  Comp.}, \textit{56}(193), 119--139.
%
%\bibitem[{\textit{Colton and Kress}(1983)}]{ColtonKress1}
%Colton, D., and R.~Kress (1983), \textit{Integral Equation Methods in
%  Scattering Theory}, John Wiley, New York.
%
%\bibitem[{\textit{Hsiao et~al.}(1991)\textit{Hsiao, Stephan, and
%  Wendland}}]{StephanHsiao}
%Hsiao, G.~C., E.~P. Stephan, and W.~L. Wendland (1991), On the {D}irichlet
%  problem in elasticity for a domain exterior to an arc, \textit{J. Comput.
%  Appl. Math.}, \textit{34}(1), 1--19.
%
%\bibitem[{\textit{Lu and Ando}(2012)}]{LuAndo}
%Lu, P., and M.~Ando (2012), Difference of scattering geometrical optics
%  components and line integrals of currents in modified edge representation,
%  \textit{Radio Sci.}, \textit{47},  RS3007, \doi{10.1029/2011RS004899}.

%\end{thebibliography}

%Reference citation examples:

%...as shown by \textit{Kilby} [2008].
%...as shown by {\textit  {Lewin}} [1976], {\textit  {Carson}} [1986], {\textit  {Bartholdy and Billi}} [2002], and {\textit  {Rinaldi}} [2003].
%...has been shown [\textit{Kilby et al.}, 2008].
%...has been shown [{\textit  {Lewin}}, 1976; {\textit  {Carson}}, 1986; {\textit  {Bartholdy and Billi}}, 2002; {\textit  {Rinaldi}}, 2003].
%...has been shown [e.g., {\textit  {Lewin}}, 1976; {\textit  {Carson}}, 1986; {\textit  {Bartholdy and Billi}}, 2002; {\textit  {Rinaldi}}, 2003].

%...as shown by \citet{jskilby}.
%...as shown by \citet{lewin76}, \citet{carson86}, \citet{bartoldy02}, and \citet{rinaldi03}.
%...has been shown \citep{jskilbye}.
%...has been shown \citep{lewin76,carson86,bartoldy02,rinaldi03}.
%...has been shown \citep [e.g.,][]{lewin76,carson86,bartoldy02,rinaldi03}.
%
% Please use ONLY \citet and \citep for reference citations.
% DO NOT use other cite commands (e.g., \cite, \citeyear, \nocite, \citealp, etc.).

%% ------------------------------------------------------------------------ %%
%
%  END ARTICLE
%
%% ------------------------------------------------------------------------ %%
\end{article}
%
%
%% Enter Figures and Tables here:

\begin{table}
    \caption{Coefficient terms (Eq. \ref{eqLCC}) in the decomposition of $\Delta \rm{LCC}$ for AM2.1, AM3, AM4p0, and CAM5.3.  Also shown are the correlation coefficients between the GFDL models and the decomposition from Zhou et al. 2016, and between each model and the $\Delta \rm{LCC}$ simulated for that model.  All values are computed as tropical means ($\pm 30^{\circ}$).}
    \begin{center}
    %\begin{tabular}{| l | l | l | l | p{5cm} |}
    %\begin{tabular}{l*{3}{c}r}
    \begin{tabular}{*{5}{c}}
    \hline
    %\multicolumn{5}{|c|}{Experimental Specifications} \\
    \hline
    Model      & $\gamma$                                      & $\delta$    &     Zhou et al.  CC       & $\Delta \rm{LCC}$ CC         \\ \hline
    AM2.1           &   4.7  &    -0.1            &  0.59    & 0.88                                          \\
    \\
 %                         &                                                       &                         &  301 K    &  10 years                                         \\ \hline
    AM3                & 5.6   &  -0.5           &   0.59   & 0.77                                          \\ 
    \\
 %                         &                                                              &                          &  301 K    &   10 years                                               \\ \hline
    AM4p0              & 3.5 &  -0.3          & 0.70    &   0.70                                                     \\ 
    \\
  %                        &                                                 &                       & 301K       &       4 years                                                \\ \hline
    CAM5.3               & 3.7 &  -0.9     & 1       &    ??                                                     \\ 
    \\
   %                       &                                                 &                                     & 301 K     &         2 years                                              \\ \hline
    %                      &                                                 &                                     &  301 K               &     1 year                        \\ \hline
     \end{tabular}\par
     \label{LCCtable}
\end{center}
\end{table}
%
%
% DO NOT USE \psfrag or \subfigure commands.
%
% Figure captions go below the figure.
% Table titles go above tables; all other caption information
%  should be placed in footnotes below the table.
%
%----------------
% EXAMPLE FIGURE
%
 %\begin{figure}
 %\noindent\includegraphics[width=20pc]{samplefigure.eps}
 %\caption{Caption text here}
 %\label{figure_label}
 %\end{figure}
%

%\begin{figure}
 % \includegraphics[width=6.0in]{figures/alpha_ts_am4am3am2_longamip_30yrwind_phat.pdf}
%  \caption{Time evolution of the global mean feedback parameter ($\rm{W/m^2 K}$) for AM4g10r8(black), AM3(red), and AM2.1(blue).  
%  Time series have been computed by linear regression of the global, yearly mean TOA net radiative imbalance and the surface 
%  temperature over a sliding 30 window.}
%\end{figure}

\begin{figure}
 % \includegraphics[width=6.0in]{figures/alpha_2panel_ts_am4am3am2_longamip_30yrwind_allensmbrs.eps}
  \includegraphics[width=6.0in]{figures/alpha_ts_am4am3am2_longamip_30yrwind_mostensmbrs.eps}
  \caption{Upper panel: Time evolution of the global mean feedback parameter ($\rm{W/m^2 K}$) for AM4g10r8(black), AM3(red), and AM2.1(blue).  
  Time series have been computed by linear regression of the global, yearly mean TOA net radiative imbalance and the surface 
  temperature over a sliding 30 window.  Lower panel: Time evolution of the global mean Cloud Radiative Effect ($\rm{W/m^2 K}$) for 
  AM4g10r8(black), AM3(red), and AM2.1(blue).}
  \label{fig:alphacre}
\end{figure}

%\begin{figure}
%  \includegraphics[width=6.0in]{figures/alpha_cre_ts_am4am3am2_longamip_30yrwind_phat.pdf}
%  \caption{Time evolution of the global mean Cloud Radiative Effect ($\rm{W/m^2 K}$) for AM4g10r8(black), AM3(red), and AM2.1(blue).}
%\end{figure}

\begin{figure}
  \includegraphics[width=6.0in]{figures/eistrend_ts_am432_longamip_tr_bold_30yrwind_new.pdf}
  \caption{Time evolution of the global mean Estimated Inversion Strength ($\rm{K/30 yr}$) for AM4g10r8(black), AM3(red), and    
                AM2.1(blue).  Global mean shown as thin solid lines and tropical mean values as thick solid lines.}
  \label{fig:eis}
\end{figure}

\begin{figure}
  \includegraphics[width=6.0in]{figures/lcceis_late.eps}
  \caption{Trends of the low-cloud cover (left) and estimated inversion strength (right) during the period 1975-2005 for AM4 (top), AM3 (middle), and AM2(bottom).  Trends were computed with linear regression at each grid point for the variable of interest.}
  \label{fig:lcceis_late}
\end{figure}

\begin{figure}
  \includegraphics[width=6.0in]{figures/lcceis_early.eps}
  \caption{As in previous figure but for the period 1925-1955.}
  \label{fig:lcceis_early}
\end{figure}

\begin{figure}
  \includegraphics[width=6.0in]{figures/delLCC_AM2AM3AM4_pm30.eps}
  \caption{Time evolution of the tropical mean ($\pm 30^{\circ}$) change in low-cloud cover ($\Delta \rm{LCC}$) for AM2.1(blue), AM3(red), AM4p0(black).  Thick lines show the       simulated $\Delta \rm{LCC}$ while thin lines show $\Delta \rm{LCC}$ approximated using Eq. \ref{eqLCC} with coefficients determined by multiple linear regression (see Table \ref{LCCtable}).  A nine-year running mean has been used to smooth the data series.}
  \label{fig:lccts}
\end{figure}


\begin{figure}
  \includegraphics[width=6.0in]{figures/hccomega_late.eps}
  \caption{Trends of the high-cloud cover (left) and vertical velocity (omega) on the 500 hPa level (right) during the period 1975-2005 for 
                AM4 (top), AM3 (middle), and AM2(bottom).  Trends were computed with linear regression at each grid point for the variable of 
                interest.}
  \label{fig:hccomega_late}
\end{figure}

\begin{figure}
  \includegraphics[width=6.0in]{figures/hccomega_early.eps}
  \caption{As in previous figure but for the period 1925-1955.}
  \label{fig:hccomega_late}
\end{figure}

\begin{figure}
  \includegraphics[width=6.0in]{figures/tref_t700_am4am3am2_late.eps}
  \caption{Trends of the Tref (left) and temperature on the 700hPa level (right) during the period 1975-2005 for AM4 (top), AM3 (middle), and AM2(bottom).  Trends were computed with linear regression at each grid point for the variable of interest.}
  \label{fig:tref_t700_late}
\end{figure}

\begin{figure}
  \includegraphics[width=6.0in]{figures/tref_t700_am4am3am2_early.eps}
  \caption{As in previous figure but for the period 1925-1955.}
  \label{fig:tref_t700_early}
\end{figure}

\begin{figure}
  %\includegraphics[width=6.0in]{figures/am4g10r8_eis_t700_tsurf_trend_endtime1620.eps}
  \includegraphics[width=6.0in]{figures/glb_eis_llc_t700_tsurf_trend_6pan_am4late.pdf}
  \caption{Trends during the period 1975-2005 for AM4g10r8.  Trends were computed with linear regression at each grid point for the variable of interest.}
  \label{fig:lcceisapp_am4}
\end{figure} 

\begin{figure}
  %\includegraphics[width=6.0in]{figures/am3p9_eis_t700.pdf}
  \includegraphics[width=6.0in]{figures/glb_eis_llc_t700_tsurf_trend_6pan_am3late.pdf}
  \caption{Trends during the period 1975-2005 for AM3.  Trends were computed with linear regression at each grid point for the variable of interest.}
  \label{fig:lcceisapp_am3}
\end{figure}

\begin{figure}
  \includegraphics[width=6.0in]{figures/glb_eis_llc_t700_tsurf_trend_6pan_am2late.pdf}
  \caption{Trends during the period 1975-2005 for AM2.  Trends were computed with linear regression at each grid point for the variable of interest.}
  \label{fig:lcceisapp_am2}
\end{figure}


%\begin{figure}
%  \includegraphics[width=6.0in]{figures/am4g10r8_ts_anom_zhouetal_fig2c_trop_glob}
%  \caption{Global (top) and tropical (bottom) mean time series of low-level cloud anomalies (thick, solid line) computed relative to the 
%  mean over full time series for AM4g10r8.  
%  Also shown (thick dashed) is a an approximation of the low-level cloud anomalies computed as the linear combination of the EIS (thin solid) and the 
%  temperature at 2m (Tref: thin dashed).  Time series have been smoothed with a 9 year running mean filter.}
%\end{figure}
%%
%\begin{figure}
%  \includegraphics[width=6.0in]{figures/am3p9_ts_anom_zhouetalfig2c_glob_trop}
%  \caption{Global (top) and tropical (bottom) mean time series of low-level cloud anomalies (thick, solid line) computed relative to the 
%  mean over full time series for AM3.  
%  Also shown (thick dashed) is a an approximation of the low-level cloud anomalies computed as the linear combination of the EIS (thin solid) and the 
%  temperature at 2m (Tref: thin dashed).  Time series have been smoothed with a 9 year running mean filter.}
%\end{figure}
%%
%\begin{figure}
%  \includegraphics[width=6.0in]{figures/am2_1_ts_anom_zhouetalfig2c_trop_gl}
%  \caption{Global (top) and tropical (bottom) mean time series of low-level cloud anomalies (thick, solid line) computed relative to the 
%  mean over full time series for AM2.1
%  Also shown (thick dashed) is a an approximation of the low-level cloud anomalies computed as the linear combination of the EIS (thin solid) and the 
%  temperature at 2m (Tref: thin dashed).  Time series have been smoothed with a 9 year running mean filter.}
%\end{figure}
%
\begin{figure}
  \includegraphics[width=6.2in]{figures/glb_rad_trends_late.eps}
  \caption{TOA radiative fluxes $\rm{W/m^2 per 30yr}$ for AM4g10r8 (left), AM3 (center), and AM2.1(right).  Fields are trends computed between 1975 and 2005.}
\end{figure}

%\begin{figure}
%  \includegraphics[width=6.2in]{figures/glb_trend_am4am3am2_early.pdf}
%  \caption{TOA radiative fluxes $\rm{W/m^2 per 30yr}$ for AM4g10r8 (left), AM3 (center), and AM2.1(right).  Fields are trends computed over a 30 year period near the middle of the twentieth century.}
%\end{figure}

\begin{figure}
  \includegraphics[width=6.2in]{figures/glb_rad_trends_early.eps}
  \caption{TOA radiative fluxes $\rm{W/m^2 per 30yr}$ for AM4g10r8 (left), AM3 (center), and AM2.1(right).  Fields are trends computed between 1925 and 1955.}
\end{figure}

%\begin{figure}
%  \includegraphics[width=6.0in]{}
%  \caption{}
%\end{figure}

% ---------------
% EXAMPLE TABLE
%
%\begin{table}
%\caption{Time of the Transition Between Phase 1 and Phase 2\tablenotemark{a}}
%\centering
%\begin{tabular}{l c}
%\hline
% Run  & Time (min)  \\
%\hline
%  $l1$  & 260   \\
%  $l2$  & 300   \\
%  $l3$  & 340   \\
%  $h1$  & 270   \\
%  $h2$  & 250   \\
%  $h3$  & 380   \\
%  $r1$  & 370   \\
%  $r2$  & 390   \\
%\hline
%\end{tabular}
%\tablenotetext{a}{Footnote text here.}
%\end{table}

% See below for how to make sideways figures or tables.

\end{document}

%%%%%%%%%%%%%%%%%%%%%%%%%%%%%%%%%%%%%%%%%%%%%%%%%%%%%%%%%%%%%%%

More Information and Advice:

%% ------------------------------------------------------------------------ %%
%
%  SECTION HEADS
%
%% ------------------------------------------------------------------------ %%

% Capitalize the first letter of each word (except for
% prepositions, conjunctions, and articles that are
% three or fewer letters).

% AGU follows standard outline style; therefore, there cannot be a section 1 without
% a section 2, or a section 2.3.1 without a section 2.3.2.
% Please make sure your section numbers are balanced.
% ---------------
% Level 1 head
%
% Use the \section{} command to identify level 1 heads;
% type the appropriate head wording between the curly
% brackets, as shown below.
%
%An example:
%\section{Level 1 Head: Introduction}
%
% ---------------
% Level 2 head
%
% Use the \subsection{} command to identify level 2 heads.
%An example:
%\subsection{Level 2 Head}
%
% ---------------
% Level 3 head
%
% Use the \subsubsection{} command to identify level 3 heads
%An example:
%\subsubsection{Level 3 Head}
%
%---------------
% Level 4 head
%
% Use the \subsubsubsection{} command to identify level 3 heads
% An example:
%\subsubsubsection{Level 4 Head} An example.
%
%% ------------------------------------------------------------------------ %%
%
%  IN-TEXT LISTS
%
%% ------------------------------------------------------------------------ %%
%
% Do not use bulleted lists; enumerated lists are okay.
% \begin{enumerate}
% \item
% \item
% \item
% \end{enumerate}
%
%% ------------------------------------------------------------------------ %%
%
%  EQUATIONS
%
%% ------------------------------------------------------------------------ %%

% Single-line equations are centered.
% Equation arrays will appear left-aligned.

Math coded inside display math mode \[ ...\]
 will not be numbered, e.g.,:
 \[ x^2=y^2 + z^2\]

 Math coded inside \begin{equation} and \end{equation} will
 be automatically numbered, e.g.,:
 \begin{equation}
 x^2=y^2 + z^2
 \end{equation}

% IF YOU HAVE MULTI-LINE EQUATIONS, PLEASE
% BREAK THE EQUATIONS INTO TWO OR MORE LINES
% OF SINGLE COLUMN WIDTH (20 pc, 8.3 cm)
% using double backslashes (\\).

% To create multiline equations, use the
% \begin{eqnarray} and \end{eqnarray} environment
% as demonstrated below.
\begin{eqnarray}
  x_{1} & = & (x - x_{0}) \cos \Theta \nonumber \\
        && + (y - y_{0}) \sin \Theta  \nonumber \\
  y_{1} & = & -(x - x_{0}) \sin \Theta \nonumber \\
        && + (y - y_{0}) \cos \Theta.
\end{eqnarray}

%If you don't want an equation number, use the star form:
%\begin{eqnarray*}...\end{eqnarray*}

% Break each line at a sign of operation
% (+, -, etc.) if possible, with the sign of operation
% on the new line.

% Indent second and subsequent lines to align with
% the first character following the equal sign on the
% first line.

% Use an \hspace{} command to insert horizontal space
% into your equation if necessary. Place an appropriate
% unit of measure between the curly braces, e.g.
% \hspace{1in}; you may have to experiment to achieve
% the correct amount of space.


%% ------------------------------------------------------------------------ %%
%
%  EQUATION NUMBERING: COUNTER
%
%% ------------------------------------------------------------------------ %%

% You may change equation numbering by resetting
% the equation counter or by explicitly numbering
% an equation.

% To explicitly number an equation, type \eqnum{}
% (with the desired number between the brackets)
% after the \begin{equation} or \begin{eqnarray}
% command.  The \eqnum{} command will affect only
% the equation it appears with; LaTeX will number
% any equations appearing later in the manuscript
% according to the equation counter.
%

% If you have a multiline equation that needs only
% one equation number, use a \nonumber command in
% front of the double backslashes (\\) as shown in
% the multiline equation above.

%% ------------------------------------------------------------------------ %%
%
%  SIDEWAYS FIGURE AND TABLE EXAMPLES
%
%% ------------------------------------------------------------------------ %%
%
% For tables and figures, add \usepackage{rotating} to the paper and add the rotating.sty file to the folder.
% AGU prefers the use of {sidewaystable} over {landscapetable} as it causes fewer problems.
%
% \begin{sidewaysfigure}
% \includegraphics[width=20pc]{samplefigure.eps}
% \caption{caption here}
% \label{label_here}
% \end{sidewaysfigure}
%
%
%
% \begin{sidewaystable}
% \caption{}
% \begin{tabular}
% Table layout here.
% \end{tabular}
% \end{sidewaystable}
%
%

